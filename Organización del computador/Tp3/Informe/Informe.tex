\documentclass[a4paper]{article}
\usepackage[spanish]{babel}
\usepackage[utf8]{inputenc}
\usepackage{charter}   % tipografia
\usepackage{graphicx}
\usepackage[]{algorithm2e}
\usepackage{makeidx}
\usepackage{paralist} %itemize inline

\usepackage{multirow}

\usepackage{float}
\usepackage{amsmath, amsthm, amssymb}
\usepackage{amsfonts}
\usepackage{sectsty}
\usepackage{charter}
\usepackage{wrapfig}
\usepackage{listings}
\lstset{language=C}


\input{codesnippet}
\input{page.layout}
\setcounter{secnumdepth}{2}
\usepackage{underscore}
\usepackage{caratula}
\usepackage{url}


% ******************************************************** %
%              TEMPLATE DE INFORME ORGA2 v0.1              %
% ******************************************************** %
% ******************************************************** %
%                                                          %
% ALGUNOS PAQUETES REQUERIDOS (EN UBUNTU):                 %
% ========================================
%                                                          %
% texlive-latex-base                                       %
% texlive-latex-recommended                                %
% texlive-fonts-recommended                                %
% texlive-latex-extra?                                     %
% texlive-lang-spanish (en ubuntu 13.10)                   %
% ******************************************************** %



\begin{document}


\thispagestyle{empty}
\materia{Organizaci\'on del Computador II}
\submateria{Segundo Cuatrimestre de 2014}
\titulo{Trabajo Pr\'actico III}
\subtitulo{Programación de Sistemas, Lo de Charly Bondiola}
\integrante{Federico De Rocco}{403/13}{fede.183@hotmail.com}
\integrante{Fernando Abelini}{544/09}{ferabelini@outlook.com}
\integrante{Manuel Casanova}{355/05}{casanova_manuel@yahoo.com}
\maketitle
\newpage


\thispagestyle{empty}
\vspace{3cm}
\tableofcontents
\newpage


%\normalsize


\section{Introducción}

En el siguiente informe se mostrara y explicara la forma en que se lograron resolver los ejercicios donde se utilizan los conceptos de System Programming. El sistema busca poder correr 16 tareas al mismo tiempo a nivel de usuario. Para esto se implementara un juego sencillo de dos jugadores que correrá usando este sistema. Cada jugador podrá enviar un zombie para atacar a su oponente. Por zombie nos referimos a tareas del sistema que pueden desplazarse a través de la memoria buscando a otras tareas para devorar sus cerebros. 
Ambos jugadores tienen un máximo de 8 zombies en juego de tres clases, Guerrero, Mago y Clérigo. 

\section {Ejercicio 1}

\subsection{Tabla de Descriptores Globales}

También llamada GDT es una estructura que nos permitirá gestionar y direccionar segmentos de memoria(Tanto de datos como de programas). En el caso puntual del tp, está contara con 4 segmentos, tendremos dos de ellos para el código y los otros para datos, cuyos niveles serán 0 y 3 respectivamente. Estos segmentos direccionan los primeros 623MB de memoria. Una restricción impuesta para este trabajo práctico nos dice que las primeras 7 posiciones no pueden utilizarse y a los ojos del enunciado están ocupadas. Entonces el primer índice usado para declarar estos segmentos es el número 8, exceptuando el descriptor nulo que está en la posición 0. Este último será usado para cargar la GDT.

En la GDT se construyen 6 segmentos. El primero es la dirección que se encuentra vacía. Los de código y datos poseen en como límite 623MB-1(osea los primeros 623Mb direccionados comenzando desde el 0). De estos la única diferencia es el type(Este lógicamente las distingue entre código y datos). Entre estos el primer, tanto en código como en datos, tiene el dpl marcado para que solo pueda usarlo el kernel. Las otras pueden ser usadas por los procesos o tareas. El último segmento corresponde a la pantalla. La misma tiene un tamaño de 80x50 píxeles. Por esto, su límite debe ser de 80*50*2 -1 y su base está marcada en la dirección 0xB8000. También lo pusimos como datos, usando type. Para poder verlo en bytes usamos que la granularidad es 0(bit g del segmento).

\subsection{Pila}

Se debe setear la pila en la posición 0x27000(Pasado por enunciado) para pasar a modo protegido. Esto se hace asignandole este valor a nuestros registros base y tope de pila(ebp y esp). 

\subsection{Pintar Pantalla}

El enunciado nos pidió hacer una rutina que se encargue de dibujar la pantalla con color verde. Para resolver esto se construyó una macro en el archivo pantalla.mac. Dicha macro escribe es cada píxel de la pantalla el valor 0x2020. El primer byte de valor 20 tiene asociado el color, el cual es verde, el segundo tiene asociado el valor del carácter espacio. El primer ciclo recorre toda la pantalla. El segundo recorre solo la primer columna de esta. En el mismo se asigna el valor 0x4020. Este representa nuevamente al carácter espacio pero de color rojo. Con azul es equivalente, solo que tomamos la última columna y la pintamos de azul(0x1020). Algo importante de señalar es como se accede a la pantalla. Esto se realiza usando el selector de segmento fs.

\subsection{Pasar a modo protegido}

Para poder realizar esta acción tendremos que completar la GDT como se vio en la sección anterior y después se deben realizar los siguientes pasos:

\begin{enumerate}
\item Deshabilitar interrupciones(Usando CLI).
\item Cargar el registro GDTR con la dirección base de la GDT.
\item Setear el bit PE del registro CR0.
\item Hacer un FAR JUMP a la siguiente instrucción.
\item Cargar los registros de segmento(DS, ES, GS, FS y SS).
\end{enumerate}

En esencia solo se debería realizar el paso 3 pero hay que deshabilitar las interrupciones para que no caiga una justo en medio del proceso. También se deberá hacer el paso 2 para poder estar seguro que tenemos la gdt. Usando el FAR JUMP ponemos el límite de lo que se ejecuta en modo real y después de este estamos ejecutando en modo protegido. Por último establecemos los selectores de segmentos enumerados en el punto 5. 

Hay algunas cosas que debemos hacer antes de comenzar el pasaje a modo protegido. Esto es habilitar A20 la cual se define como una linea de control para el controlador de memoria, esta nos habilita el acceso a direcciones superiores a los 2 a la 20 bits. Este pin se debe habilitar para poder direccionar sobre el MB de memoria.

\section{Ejercicio 2}

\subsection{IDT}

Para completar la estructura utilizamos la macro IDT\_ENTRY. La cual recibe un número que corresponde a la interrupción en la tabla de interrupciones. La función idt\_inicializar se encarga de usar esta macro para cada una de las excepciones de la tabla que nos interese poner. Estas son únicamente las que están contempladas por INTEL. Las demás están reservadas. Estas nos quedan de la siguiente forma:
\\
    $IDT\_ENTRY(0)$\\
    $IDT\_ENTRY(2)$\\
    $IDT\_ENTRY(3)$\\
    $IDT\_ENTRY(4)$\\
    $IDT\_ENTRY(5)$\\
    $IDT\_ENTRY(6)$\\
    $IDT\_ENTRY(7)$\\
    $IDT\_ENTRY(8)$\\
    $IDT\_ENTRY(9)$\\
    $IDT\_ENTRY(10)$\\
    $IDT\_ENTRY(11)$\\
    $IDT\_ENTRY(12)$\\
    $IDT\_ENTRY(13)$\\
    $IDT\_ENTRY(14)$\\
    $IDT\_ENTRY(16)$\\
    $IDT\_ENTRY(17)$\\
    $IDT\_ENTRY(18)$\\
    $IDT\_ENTRY(19)$\\
\\

Para poder atender las excepciones se fabricó un conjunto de mensajes para cada una de las ellas. Dichos mensajes se mostraran en pantalla usando la macro ISR la cual consiste en usar el número de excepción y llamar a la función imprimir\_texto\_mp pasándole como dato el mensaje adecuado para esta excepción. Además de la parte superior izquierda de la pantalla, el color del texto y fondo. 

También debemos completar la isr.h con cada una de las funciones pertinentes de las excepciones anteriores. Estas serán posteriormente definidas en isr.asm usando la macro anteriormente mencionada.

\section{Ejercicio 3}

\subsection{Pantalla}

En esencia se hace lo mismo que en el Pintar Pantalla del ejercicio 1. Lo único que se agrega es la parte negra de la parte baja y los cuadros azul y rojo de cada jugador que indican el puntaje. También imprimimos los números que van por encima de los relojes de los zombies.

\subsection{Inicializar directorio de páginas y tabla de páginas del kernel y activar paginación}

El directorio de páginas se inicia en la posición 0x27000 y la tabla de páginas en la dirección 0x28000. El valor de la base del directorio de páginas se define con el registro cr3, solo que teniendo en cuenta solo los primeros 20 bits(Tiene que ser múltiplo de 4kb). Todos los demás bits de atributos los pusimos en 0. Simplemente cargamos ese valor. PCD y PWT tienen que estar desactivados para poder activar paginación. Completamos el directorio de páginas con páginas en blanco(todos los valores nulos). Excepto la primera que corresponde a la de la tabla de páginas. La tabla de páginas fue completada con páginas con todos sus atributos en 0, excepto el de presente y el de lectura escritura que van en 1. En la base ponemos valores ascendentes del 0 al 1023. Con la intención de mapear los 4mb de memoria.

Para poder activar la paginación solo nos resta setear el bit PG del cr0. Esto se debe hacer solo si se realizaron los pasos anteriores, osea armar el directorio de páginas y tabla de páginas.

\section{Ejercicio 4}

\subsection{Inicializar mmu}

Para esta función utilizaremos la variable global indice\_pagina\_libre la cual, como el nombre lo indica, nos dirá la posición de una página libre en la tabla de páginas. Usaremos la función inicializar\_mmu para cargar en esa variable la posición de la primer página libre. Esta se encuentra en 0x00100000. 

\subsection{Inicializar un directorio de páginas y tablas de páginas para una tarea}

Usaremos las estructuras offset\_pagina\_zombie\_A y offset\_pagina\_zombie\_B las cuales nos servirán para saber que posición ocupa cada página de la tabla en la pantalla. 

Para poder realizar lo pedido debemos tener como parámetros el jugador y la posición en pantalla. Comenzaremos guardando en nuestra variable cr3 la dirección de una nueva página, posteriormente la usaremos para mapear las páginas. Obtendremos esto usando la función obtener\_pagina\_libre la cual aumenta el índice de página libre y lo devuelve. Usando esta variable cr3 mapeamos las páginas con idetity mapping las direcciones de 0x00000000 a 0x003fffff.

Dependiendo de que jugador sea, mapearemos las direcciones desde 0x08000000 a 0x08008000 usando la función mmu\_mapear\_pagina y las estructuras offset\_pagina\_zombie\_A y offset\_pagina\_zombie\_B. Para cada caso debemos conseguirnos la posición inicial de mapa.
\\
Para el jugador 1:
$pos\_inicial\_mapa=((posicionY*78 + 1)* 0x1000) + base\_mapa$\\
Para el jugador 2:
$pos\_inicial\_mapa=(( posicionY*78 + 77)* 0x1000) + base\_mapa$\\

Siendo posicionY la pasada como parámetro y base\_mapa el valor 0x00400000 que hace referencia a la primer posición libre después de la memoria del kernel. Utilizaremos esta posición inicial para poder obtener la dirección física. Sumaremos este valor con cada una de las ocho posiciones de las estructuras offset para obtener las direcciones físicas a mapear en la pantalla. Y pondremos en cada una de ellas el valor de atributos en 1, para setear el bit de Usuario Supervisor, y como dirección virtual tomaremos las posiciones 0x08000000 a 0x08008000. También usaremos el cr3 obtenido anteriormente.

\subsection{Mapear página}

Haremos esto tomando una dirección física, una virtual, un cr3 y los atributos. Tomamos la dirección y la descomponemos en sus partes y solo usaremos el índice del directorio y de la tabla. Nos conseguimos la base del directorio de páginas con el cr3. Después usaremos la parte alta de la dirección virtual para obtener el la correspondiente PDE. Tenemos que asegurarnos que la página este presente. Si no lo está, seteamos el bit p, ponemos en base una página libre y le asignamos los atributos pasados. Después obtendremos la base de la tabla de páginas de ese directorio y usaremos la parte de table de virtual para obtener la PTE buscada. Seteamos el bit de presente, le asignamos en base la dirección física y le ponemos los atributos. Concluiremos llamando a la función tlbflush para invalidar la cache de traducción de direcciones. Como aclaración tenemos que mencionar que atributos solo contiene información sobre el us(Usuario/Supervisor).

\subsection{Unmapear página}

Para esta función haremos lo mismo que en la anterior. Osea buscar la PDE y, posteriormente, la PTE correspondiente usando el cr3 y la dirección virtual. Después tenemos que poner el bit de presente de esa tabla en 0. Si esta página resulta ser la última página presente en la tabla de páginas, también pondremos en 0 el bit p del directorio de páginas correspondiente a esta tabla.

\section{Ejercicio 5}

En este punto, agregamos las entradas 32, 33 y 102(0x66) en la IDT. Estas se utilizan para clock, teclado y software respectivamente.

En los 3 casos se realizan los siguientes pasos:
\begin{enumerate}
  \item  Deshabilitar las interrupciones.
\item Preservar los registros.
\item Comunicar al PIC que ya se atendió la interrupción(salvo la 102 que no se comunica con el pic).
\item Realizar la tarea correspondiente a la interrupción.
\item Restaurar los registros.
\item Habilitar las interrupciones.
\item Retornar de la interrupción.
\end{enumerate}

\subsection{Reloj}

En cada interrupción de clock se ejecuta la función proximo\_reloj. Dicha función es la que se encarga de imprimir el reloj en la esquina inferior derecha de la pantalla de juego.

\subsection{Teclado}

Cada vez que se pulsa una tecla de las habilitadas para el juego, el pic nos informa que se produjo una interrupción. Mediante el puerto 0x60 obtenemos el scan code de la tecla que fue presionada. En el archivo pic.c creamos la función que convierte un scan code(de los 10 habilitados para el juego), en un carácter para luego ser impreso en pantalla. En caso de ser una tecla que no es del juego, se ignora.

La macro imprimir\_texto\_mp recibe la dirección de memoria del mensaje. Con lo cual en el código de la atención de interrupción del teclado, pusheamos el carácter en la pila y utilizamos la dirección de ESP como parámetro.  

\section{Ejercicio 6}

\subsection{GDT de tareas}

Se pide describir dos gdt para tareas. Una para la tarea inicial y otra para la tarea idle. En el caso de la tarea inicial:\\
$limit=No$ $importa$\\
$base=No$ $importa$\\
$type=0x09$\\
$s=0x00$\\
$dpl=0x00$\\
$p=0x01$\\
$avl=0x00$\\
$l=0x00$\\ 
$db=0x00$\\
$g=0x01$\\   

No nos importa que tenga en base o límite porque no lo vamos a usar. En el caso de base la vamos a inicializar desde tss.c. El tipo tiene que ser 9 ya que debe poderse ejecutar y además acceder. El s debe estar en 0 ya que esto pertenece al sistema y no a código o datos. El dpl debe estar en 0 porque, como ya dijimos, esto es usado por el sistema entonces debe tener nivel de privilegio kernel. El uso de esta tarea es basicamente servir para respaldar el contexto de una tarea cuando cambie a otra. Obviamente en ambos casos debe tener el bit presente seteado. Lo demás queda como los casos de datos y código anteriores, granularidad en 1, avl en 0, etc. 

Para la tarea idle, la gdt quedaría asi:\\
$limit=0x000067$\\
$base=0x00000000$\\
$type=0x09$\\
$s=0x00$\\
$dpl=0x00$\\
$p=0x01$\\
$avl=0x00$\\
$l=0x00$\\
$db=0x00$\\
$g=0x01$\\   

Todos los atributos son iguales a los de la tarea inicial. La única diferencia es que el límite ahora si nos va a importar, ya que esta tarea hace algo mas que solo servir para inicializar las tareas. Como en el anterior el valor de la base va a ser pasado posteriormente.

\subsection{Inicializar la TSS}

El código en C nos provee de dos variables que usamos para definir la tss. Estas son tss\_inicial y tss\_idle. Lo que debemos hacer con estas es completar los espacios base de las entradas de las gdt para las tareas del mismo nombre. Simplemente tomamos las direcciones de memoria de las variables y las asignamos como se ve a continuación.\\
$unsigned$ $int$ $base = \&tss\_inicial\;$\\
$base = base << 16\;$\\
$base = base >> 16\;$\\
$gdt[TAREA\_INICIAL].base\_0\_15 = base \;$\\
$base =  \&tss\_inicial << 8\;$\\
$base = base >> 24\;$\\
$gdt[TAREA\_INICIAL].base\_23\_16 = base\;$\\
$base = ( unsigned int ) \&tss\_inicial >> 24\;$\\
$gdt[TAREA\_INICIAL].base\_31\_24 = base\;$\\

La esencia es tomar la dirección de memoria de la tss correspondiente y ponerla en base de la gdt. Lo mismo se hace para la tarea idle. Con la diferencia que en el caso de esta última tendremos que completar los atributos que en el caso de la inicial no nos interesaba como quedaban. Pasaremos a mostrar que asignamos en los campos de la tss\_idle, los que no mostramos los dejamos en 0 porque no nos importaban.\\
    $tss\_idle.esp0 = 0x17000$\\
    $tss\_idle.ss0 = 0x50$\\
    $tss\_idle.cr3 = 0x27000$\\ 
    $tss\_idle.eip = 0x00016000$\\
    $tss\_idle.eflags = 0x00000202$\\ 
    $tss\_idle.esp = 0x27000$\\ 
    $tss\_idle.ebp = 0x27000$\\
    $tss\_idle.es = 0x50$\\
    $tss\_idle.cs = 0x40$\\
    $tss\_idle.ss = 0x50$\\
    $tss\_idle.ds = 0x50$\\
    $tss\_idle.fs = 0x50$\\
    $tss\_idle.gs = 0x50$\\
    $tss\_idle.iomap = 0xFFFF$\\

El cr3, es, cs, ss, ds, fs, gs, esp y ebp son los mismos que el kernel. La eip y iomap están pasadas por enunciado. El eflags tiene solamente seteado lo necesario para tener las interrupciones habilitadas.    

\subsection{Completar TSS libre}

Pasando como parámetros el índice en la gdt, un jugador, la posición del jugador y una tss libre. Lo que haremos es similar a lo que hicimos en los enunciados anteriores. Primero obtenemos la base usando la tss libre y la asignamos a la gdt señalada por ese índice. Después completaremos los atributos de la tss.\\
    $t->esp0 = obtener\_pagina\_libre()+0x1000$\\
    $t->ss0 = 0x40$\\
    $t->cr3 = mmu\_inicializar\_dir\_zombie(posicionY, jugador)$\\
    $t->eip = 0x00800000$\\
    $t->eflags = 0x00000202$\\
    $t->esp = 0x00801000$\\
    $t->ebp = 0x00801000$\\
    $t->es = 0x5B$\\
    $t->cs = 0x4B$\\
    $t->ss = 0x5B$\\
    $t->ds = 0x5B$\\
    $t->fs = 0x5B$\\
    $t->gs = 0x5B$\\
    $t->iomap = 0xFFFF$\\

El ss0 es el stack de nivel 0 y lógicamente debe ser el mismo que el del kernel. En esp0, que es el segmento para entrar en la pila ss0, la obtenemos como una nueva página. La eip debe ser entre 0x00800000 porque son de las tareas zombie. Los registros esp y ebp tienen que usar la parte del final de su página de código. El eflags está igual que el anterior, con las interrupciones habilitadas. Los selectores de código y datos tienen por valor las posiciones de código y datos 2 de la gdt. El cr3 es el que obtenemos para asignarle una tarea zombie usando mmu\_inicializar\_dir\_zombie. Para esto usamos el jugador y la posición del mismo, pasadas como parámetros. 

\section{Ejercicio 7}

\subsection{Estructuras}

Las estructuras de datos que necesitaremos para el scheduler serán las siguientes:

\begin{enumerate}
\item  Un entero de nombre sched\_jugador que será 0 si el jugador actual es el primero, 1 si es el segundo, -1 si se está corriendo la tarea idle.

\item Un entero sched\_indice\_tarea que indique el índice del zombie vivo, actualmente corriendo.

\item Un entero sched\_indice\_tarea\_A que nos dirá el índice del último zombie del primer jugador.

\item Un entero sched\_indice\_tarea\_B ídem anterior para el segundo jugado

\item Un entero sched\_indice\_actual que nos dice que índice de la gdt que se está ejecutando.
\end{enumerate}

Para el juego usamos una estructura jugador que tiene los siguientes datos:

\begin{enumerate}
    \item  Un entero tipo\_zombie que nos dirá cual es el tipo de zombie que está marcando en el lanzador. Si vale 0 es guerrero, 1 es mago y 2 es clérigo.

    \item Un entero de nombre pos que nos dirá la posición donde se encuentra el zombie en el lanzador. Como solo hay dos jugadores, al saber que jugador es ya sabemos la posición de la columna y esta variable nos indica la fila. 

    \item Un entero zombies\_lanzados que nos indicará la cantidad de zombies que fueron lanzados por ese jugador. El valor nunca es superior a 20 por como está implementado game\_lanzar\_zombie. 

    \item Un arreglo de zombie\_descriptor nombrado como zombies. El arreglo tiene tamaño 8(Cantidad máxima de zombies de un jugador en juego). Esta estructura se describirá posteriormente.

    \item Un entero colinicial que nos indicará cual es la columna donde se lanzará el zombie del jugador. Si es el primer jugador será 1, 76 para el segundo.

    \item Un entero puntaje el cual tendrá la cantidad de veces que al jugador contrario le entró un zombie en su área(Anota un punto).
\end{enumerate}

Agregamos 16 descriptores para la gdt que nos servirán para cada una de las tareas de los zombies. La razón por la cual son 16 es porque cada jugador puede tener una cantidad máxima de 8 tareas en juego.

Para el zombie usamos la estructura zombie\_descriptor que tiene los siguientes datos:
\begin{enumerate}
    \item Un carácter de nombre tipo. Que tiene como valor 'G' si el zombie es guerrero, 'M' si es mago y 'C' en el caso que sea clérigo.

    \item Un entero de nombre vivo, que tendrá como valor 1 si el zombie está activo, 0 en caso contrario.

    \item Un short de nombre gdt\_index que, como el nombre índica, será el índice de la gdt de la tarea del zombie.

    \item Un entero de nombre x, que será el índice de columna en el cual se encuentra el zombie en la pantalla.

    \item Un entero de nombre y, ídem anterior para fila.

    \item Un entero pos\_reloj. Este dato se utiliza para imprimir los relojes en cada zombie y se actualiza por cada interrupción de reloj. Este tendrá de valor: 0 si el reloj está en -, 1 está en /, en 2 es $|$ y en 3 está \backslash. 
\end{enumerate}

\subsection{Inicializar Scheduler}

Debemos inicializar las estructuras del sched mencionadas anteriormente. Para esto les asignamos los siguientes valores:
\begin{enumerate}
    \item $sched\_jugador = -1$\;
    \item $sched\_indice\_tarea\_A = 0$\;
    \item $sched\_indice\_tarea\_B = 0$\;
    \item $sched\_indice\_tarea = 0$\;
    \item $sched\_indice\_actual = (GDT\_IDX\_TAREA\_IDLE * 8)$\; (Comenzamos con la tarea idle).  
\end{enumerate}


Como se dijo en la sección anterior, cuando sched\_jugador está en -1 significa que se está corriendo la tarea idle. Por esto mismo ponemos sched\_indice\_actual en el índice de la gdt de la tarea idle.

\subsection{Scheduler próximo índice}

Esta función es la encargada de devolver el próximo índice a ser ejecutado. Devuelve un selector de segmento en caso de tener que saltar a una tarea, 0 si no hay que realizar un cambio de tarea. El selector devuelto se guarda en la variable global sched\_indice\_actual. el selector devuelto apunta siempre un tss\_descriptor, y este puede pertenecer a un zombie o la tarea idle.\\

Esta función realiza los siguientes pasos:
\begin{enumerate}
\item Ordena a los jugadores, dependiendo de quien movió último. Esto lo determina utilizando la variable sched\_jugador.
\item Busca dentro de los zombies del siguiente jugador el siguiente zombie vivo, si encuentra alguno, marca en las variables que va a ser el próximo zombie a correr. Para determinar cual es el siguiente empieza a buscar a partir del siguiente índice, dentro del vector de zombies, que corrió último para ese jugador. Para esto utiliza las variables sched\_indice\_tarea\_A o sched\_indice\_tarea\_B según corresponda.
\item Si no encuentra alguno vivo, realiza la misma búsqueda dentro del vector de zombies del otro jugador, y realiza las modificaciones correspondientes. 
\item Si no encuentra algún zombie vivo, determina que la tarea que debe correr es la tarea idle, no modifica ninguna variable.
\item Por último, compara que el dato que determinó debe ser el siguiente. Para esto usa el valor de la tarea que se está ejecutando ( sched\_indice\_actual ), y si son iguales devuelve 0, indicando que no es necesario un cambio de tarea. 
\end{enumerate}

\subsection{Salto de tareas}

El salto de tareas se realiza en la interrupción de reloj. Modificamos esta interrupción agregando la función anterior y nos aseguramos de modificar los relojes de las tareas. Si la función sched\_proximo\_indice nos devuelve un valor distinto a 0 entonces se debe realizar el salto a la tarea que devuelve como resultado. En caso contrario no se realiza el salto y sale de la interrupción. 

\subsection{Interrupción 0x66}

Debemos modificar esta interrupción para que implemente el servicio mover. Como sabemos la syscall nos pasará en eax un dato que nos dirá para donde debe desplazarse el zombie actualmente en juego. Utilizamos una función en C llamada zombie\_mover para poder resolver este problema. Dicha función recibe el dato en eax y hace lo siguiente:
\begin{enumerate}
  \item Primero que nada nos creamos cuatro variables: adelante, atras, izquierda y derecha. Que contendrán los valores en que se deberá desplazar el zombie en pantalla para cada caso, estas serán usadas para cambiar los valores de x ó y de los zombies. Tomamos en cuenta también los casos borde.
  \item Obtenemos la posición en pantalla actual del zombie y dibujamos la baba en él.
  \item Vemos a donde nos tenemos que desplazar y copiamos las páginas de forma correcta. También modificamos los valores de x ó y del zombie según corresponda:

  \begin{algorithm}[H]
    $mmu\_copiar\_pagina((unsigned int *) 0x8000000, (unsigned int *) 0x8005000)$\;
    $zombie\_actual.y = zombie\_actual.y + izquierda$\;
    \caption{Caso de izquierda}
    \end{algorithm} 

    \begin{algorithm}[H]
    $mmu\_copiar\_pagina((unsigned int *) 0x8000000, (unsigned int *) 0x8004000)$\;
    $zombie\_actual.y = zombie\_actual.y + derecha$\;
    \caption{Caso de derecha}
    \end{algorithm} 

    \begin{algorithm}[H]
    $mmu\_copiar\_pagina((unsigned int *) 0x8000000, (unsigned int *) 0x8001000)$\;
    $zombie\_actual.x = zombie\_actual.x + adelante$\;
    \caption{Caso de adelante}
    \end{algorithm} 

    \begin{algorithm}[H]
    $mmu\_copiar\_pagina((unsigned int *) 0x8000000, (unsigned int *) 0x8006000)$\;
    $zombie\_actual.x = zombie\_actual.x + atras$\;
    \caption{Caso de atras}
    \end{algorithm}

    \item Dibujamos el zombie usando los nuevos x e y.
    \item Revisamos si el valor de x es 0 o 77. Ya que en estos dos casos significa que el zombie a llegado al área de uno de los jugadores, el primero en el caso de 0 y el segundo en el caso de 77. En estos casos se debe aumentar el puntaje del jugador que corresponda e imprimirlo. Además se debe matar a la tarea zombie actual.
    \item Remapear
    \end{enumerate}

mmu\_copiar\_pagina copia de dirección lógica a dirección lógica una página entera. Los valores de origen y destino se especifican en el enunciado.

\subsection{Debug}

Para poder completar el modo debug lo que hacemos es agregar las siguientes variables globales:
\begin{enumerate}
  \item Una variable booleana debug\_on que cuando sea true nos indicará que el modo debugger está activado, desactivado en el caso contrario.
  \item Una estructura datos\_debug. Esta contendrá la información necesaria para mostrar en el debugger. Lo que contendrá será lo siguiente: eax, ebx, ecx, edx, esi, edi, ebp, esp, eip, cs, ss, ds, fs, gs, es y eflags.
  \item Un arreglo de short de tamaño 4000 para respaldar la pantalla de nombre respaldo\_pantalla.
  \item Otro booleano de nombre pantalla\_debug\_activa que dirá si la pantalla de debug se está mostrando ó no.
  \item Un último booleano de nombre no\_jump que es usado solamente dentro de la rutina de atención de excepciones para poder facilitar el código.
\end{enumerate}

Agregamos la letra 'Y' en el convertir\_scancode y lo que hará sencillamente es invertir el valor de debug\_on en el caso que pantalla\_debug\_activa este en falso. En el caso contrario, pintará la pantalla para que quede igual a como era antes de caer la excepción y continuará las tareas. Para esto último solamente pondremos un condicional en la rutina de atención de interrupción de reloj para evitar que salte a otra tarea si la pantalla de debug está activada. 

Utilizaremos una función de nombre debug\_zombie que se encargará de dibujar en la pantalla, previamente respaldada y los datos que pide el ejercicio. Modificaremos la macro que utilizamos para atender las excepciones para que, si debug\_on está activado, completará datos\_debug y llamará a debug\_zombie. También se debió tener en cuenta cuales de las excepciones devuelven el código de error y cuales no. Esto tiene que ver con como se obtienen los datos. Muchos de los cuales debemos conseguir de la pila como por ejemplo el cs ó eip. 


\subsection{Juego}

Para el juego agregaremos dos archivos game.c y game.h. Estos nos servirán para moldear el movimiento de los jugadores, el cambio de tipo de zombies y su lanzamiento. Para esto usaremos las siguientes variables:

\begin{enumerate}
  \item Un entero jugA\_pos que sirve para indicar en que posición de la columna roja esta el lanzador del primer jugador. Su valor predeterminado es el 0 (parte superior de la columna).
  \item Un entero jugB\_pos ídem anterior para el segundo jugador.
  \item Un entero jugA\_ztype que indica que tipo de zombie tiene el lanzador actual. Si es 0 entonces es guerrero, 1 es mago y 2 es clérigo. Su valor predeterminado es el 0 (tipo guerrero).
  \item Un entero jugB\_ztype ídem anterior para el segundo jugador.
\end{enumerate}

Usando la función game\_inicializar podremos asignar los valores predeterminados a estas variables. Además de inicializar el sched y los jugadores, pintarlos en pantalla y dibujar el puntaje y los zombies restantes de cada uno.

Para lanzar un zombie utilizaremos la función game\_lanzar\_zombie. Esta lo que hará es, si hay un zombie disponible, marcarlo como vivo. Recibirá como parámetro el jugador con el que deberá realizar esta acción. Además debemos asignarle la posición adecuada a la estructura zombie(su x e y). Completaremos su tss usando completar\_tss\_zombie que describimos en el punto 6. Mapearemos las páginas del zombie en las posiciones adecuadas usando game\_cargar\_codigo\_zombie. Lo pintaremos en pantalla e incrementaremos la cantidad de zombies lanzados del jugador, recordar que solamente puede tener 8 en juego. Y por último pintaremos este dato. 

Como aclaración, para que exista un zombie disponible se tiene que cumplir que alguno de los zombies del jugador este marcado como muerto y además el jugador debe hacer lanzado menos de 20 zombies. 

También modificamos la función convertir\_scancode para que cuando se oprima cada una de las teclas se realiza la acción apropiada del juego. En el caso de que la pantalla del debug este activa, la única tecla que hará algo sera la 'Y' que lo desactivará.

\begin{enumerate}
  \item En el caso de apretar 'A' ó 'D' el zombie del jugador uno cambiara de tipo, alternando entre guerrero, mago ó clérigo(Cambia el valor de jugA\_ztype). 
  \item En caso de 'W' y 'S' el lanzador se moverá arriba y abajo respectivamente(Cambia el valor de jugA\_pos).
  \item En el caso de la tecla shift izquierdo lo que hará es lanzar el zombie pasándole como parámetro a la función game\_lanzar\_zombie un 0(jugador 1). 
\end{enumerate}

Estos procesos se hacen sencillamente cambiando el valor de las variables descriptas al principio de la sección, según corresponda. Para las teclas 'I', 'K', 'J', 'L' y shift derecho se hace lo mismo pero para el segundo jugador y sus respectivas variables.

\end{document}
