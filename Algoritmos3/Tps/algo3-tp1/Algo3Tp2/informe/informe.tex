\documentclass[a4paper, 10pt, spanish]{article}

\usepackage[paper=a4paper, left=1.5cm, right=1.5cm, bottom=1.5cm, top=3.5cm]{geometry}
\usepackage[spanish, es-noshorthands]{babel}
\usepackage[utf8x]{inputenc}
\usepackage[none]{hyphenat}
\usepackage[colorlinks,citecolor=black,filecolor=black,linkcolor=black,    urlcolor=black]{hyperref}

% Simbolos matemáticos
\usepackage{amsthm}
\usepackage{amsmath}
\usepackage{amsfonts}
\usepackage{amssymb}
\usepackage{algorithm}
\usepackage[noend]{algpseudocode}
\usepackage{algorithmicx}
\usepackage{listings}

% Descoración y gráficos
\usepackage{caratulaV}
\usepackage{graphicx} 
\usepackage{fancyhdr}
\usepackage{lastpage}
\usepackage{caption}
\usepackage{subcaption}
\usepackage{multirow}
\usepackage{alltt}
\usepackage{tikz}
\usepackage{varwidth,xcolor}
\usepackage{color}
\usepackage{gnuplottex}
\usepackage{verbatim}
\usepackage{framed}


% Del enunciado
\usepackage{a4wide}
\usepackage{amsmath}
\usepackage{amsfonts}
%\usepackage[ruled,vlined]{algorithm2e}

\newcommand{\kknn}{k}
\newcommand{\kpca}{\alpha}
\newcommand{\kkfold}{K}

% Acomodo fancyhdr.
\pagestyle{fancy}
\thispagestyle{fancy}
\addtolength{\headheight}{1pt}
\lhead{Algoritmos y estructuras de datos III}
\rhead{$2^{\mathrm{do}}$ cuatrimestre de 2015}
\cfoot{\thepage /\pageref*{LastPage}}
\renewcommand{\footrulewidth}{0.4pt}

\floatname{algorithm}{Pseudocodigo}
\algrenewcommand\algorithmicfunction{\textbf{Funcion}}
\algrenewcommand\algorithmicwhile{\textbf{mientras}}
\algrenewcommand\algorithmicfor{\textbf{para}}
\algrenewcommand\algorithmicforall{\textbf{para cada}}
\algrenewcommand\algorithmicdo{\textbf{hacer:}}
\algrenewcommand\algorithmicif{\textbf{si}}
\algrenewcommand\algorithmicthen{\textbf{entonces:}}
\algrenewcommand\algorithmicelse{\textbf{si no:}}
\algrenewcommand\algorithmicend{\textbf{fin}}
\algrenewcommand\algorithmicreturn{\textbf{devolver}}



\sloppy

\parskip=5pt % 10pt es el tama de fuente

% Pongo en 0 la distancia extra entre itemes.
\let\olditemize\itemize
\def\itemize{\olditemize\itemsep=0pt}



\usepackage{tikz}
%\usepackage{tikz-qtree}


\usetikzlibrary{arrows,backgrounds,calc}

\pgfdeclarelayer{background}
\pgfsetlayers{background,main}

\newcommand{\real}{\mathbb{R}}
\newcommand{\nat}{\mathbb{N}}

\newcommand{\revJ}[1]{{\color{red} #1}}

\newcommand{\convexpath}[2]{
[ 
    create hullnodes/.code={
        \global\edef\namelist{#1}
        \foreach [count=\counter] \nodename in \namelist {
            \global\edef\numberofnodes{\counter}
            \node at (\nodename) [draw=none,name=hullnode\counter] {};
        }
        \node at (hullnode\numberofnodes) [name=hullnode0,draw=none] {};
        \pgfmathtruncatemacro\lastnumber{\numberofnodes+1}
        \node at (hullnode1) [name=hullnode\lastnumber,draw=none] {};
    },
    create hullnodes
]
($(hullnode1)!#2!-90:(hullnode0)$)
\foreach [
    evaluate=\currentnode as \previousnode using \currentnode-1,
    evaluate=\currentnode as \nextnode using \currentnode+1
    ] \currentnode in {1,...,\numberofnodes} {
-- ($(hullnode\currentnode)!#2!-90:(hullnode\previousnode)$)
  let \p1 = ($(hullnode\currentnode)!#2!-90:(hullnode\previousnode) - (hullnode\currentnode)$),
    \n1 = {atan2(\x1,\y1)},
    \p2 = ($(hullnode\currentnode)!#2!90:(hullnode\nextnode) - (hullnode\currentnode)$),
    \n2 = {atan2(\x2,\y2)},
    \n{delta} = {-Mod(\n1-\n2,360)}
  in 
    {arc [start angle=\n1, delta angle=\n{delta}, radius=#2]}
}
-- cycle
}

\newcommand{\todo}[1]{
\textbf{\color{red}{\underline{Nota:} #1}}
}

\newcommand\param[3]{\ensuremath{\mathbf{\textbf{#1}}\,#2\!:} \texttt{#3}}

\let\state\State
\let\while\While
\let\endwhile\EndWhile
\let\endif\EndIf
\let\elseif\ElsIf
\let\for\For
\let\endfor\EndFor
\let\function\Function
\let\endfunction\EndFunction


\newcommand{\degree}{\ensuremath{^\circ}}

\begin{document}
%\setcounter{tocdepth}{2}
\renewcommand{\tablename}{Tabla} 


\thispagestyle{empty}
\materia{Algoritmos y estructuras de datos III}
\submateria{Segundo Cuatrimestre de 2015}
\titulo{Trabajo Práctico II}

\integrante{Federico De Rocco }{403/13}{fede.183@hotmail.com}
\integrante{Federico Nicolás Esquivel}{915/12}{alt.juss@gmail.com}
\integrante{Fernando Otero}{424/11}{fergabot@gmail.com}

\maketitle
\newpage
%\begin{titlepage}

%\maketitle

%\end{titlepage}
\setcounter{page}{1}

\newpage
\tableofcontents

\newpage


\newpage


\section{Ejercicio 1}
\subsection{Introducción}

\subsection{Desarrollo}


\subsection{Correctitud}



\subsection{Complejidad}

\begin{tikzpicture}
\node(pseudo) at (-1,0){};
\node(0) at (0,0)[shape=circle,draw]        {$0$};
\node(1) at (2,0)[shape=circle,draw]        {$1$};
\path [-]
  (0)      edge                 node [above]  {}     (1);

\end{tikzpicture}

%Pongo los tests que hice por si acaso.
\begin{table}[H]
\begin{center}
\begin{tabular}{|l|l|}
\hline
Materias & Aulas en las que se puede dictar \\
\hline \hline
0 & (1, 2) \\ \hline
1 & (1) \\ \hline
\end{tabular}
\end{center}
\end{table}

\begin{tikzpicture}
\node(pseudo) at (-1,0){};
\node(0) at (0,0)[shape=circle,draw]        {$0$};
\node(1) at (2,2)[shape=circle,draw]        {$1$};
\node(2) at (4,0)[shape=circle,draw]        {$2$};
\path [-]
  (0)      edge                 node [above]  {}     (1)
  (1)      edge                 node [above]  {}     (2)
  (2)      edge                 node [above]  {}     (0);
\end{tikzpicture}


\begin{table}[H]
\begin{center}
\begin{tabular}{|l|l|}
\hline
Materias & Aulas en las que se puede dictar \\
\hline \hline
0 & (1, 3) \\ \hline
1 & (2) \\ \hline
2 & (1, 2) \\ \hline
\end{tabular}
\end{center}
\end{table}

\begin{tikzpicture}
\node(pseudo) at (-1,0){};
\node(0) at (0,2)[shape=circle,draw]        {$0$};
\node(1) at (0,0)[shape=circle,draw]        {$1$};
\node(2) at (0,-2)[shape=circle,draw]        {$2$};
\node(3) at (2,0)[shape=circle,draw]        {$3$};
\node(4) at (4,0)[shape=circle,draw]        {$4$};
\node(5) at (6,2)[shape=circle,draw]        {$5$};
\node(6) at (6,0)[shape=circle,draw]        {$6$};
\node(7) at (6,-2)[shape=circle,draw]        {$7$};
\path [-]
  (0)      edge                 node [above]  {}     (3)
  (1)      edge                 node [above]  {}     (3)
  (2)      edge                 node [above]  {}     (3)
  (3)      edge                 node [above]  {}     (4)
  (4)      edge                 node [above]  {}     (5)
  (4)      edge                 node [above]  {}     (6)
  (4)      edge                 node [above]  {}     (7);
\end{tikzpicture}


\begin{table}[H]
\begin{center}
\begin{tabular}{|l|l|}
\hline
Materias & Aulas en las que se puede dictar \\
\hline \hline
0 & (1, 4) \\ \hline
1 & (2, 3) \\ \hline
2 & (3, 4) \\ \hline
3 & (4, 3) \\ \hline
4 & (1, 3) \\ \hline
5 & (2, 1) \\ \hline
6 & (3, 4) \\ \hline
7 & (4, 1) \\ \hline
\end{tabular}
\end{center}
\end{table}

\subsection{Experimentación}






\section{Ejercicio 2}
\subsection{Introducción}
En este problema buscamos minimizar el tiempo necesario para llegar desde el principio del pasillo en la planta baja al final del pasillo en el último piso.
Se garantiza que existe al menos un camino entre estos puntos.


\subsection{Desarrollo}
Para resolver este ejercicio en la complejidad pedida consideramos usar un algoritmo Breadth-First Search.
Sin embargo, como usar un portal tiene costo distinto a caminar en el mismo piso y los portales pueden estar a cualquier distancia entre si, necesitamos primero normalizar los costos.
Para ello consideramos en lugar de un grafo con solo portales como vértices, uno donde cada posición de cada piso esta representada.
De ésta forma, entre portales del mismo piso hay tantos vértices como posiciones intermedias, y atravesar cada una tiene costo 1, por lo que es suficiente contar la cantidad vértices.
Como utilizar un portal tiene costo 2, agregamos un nodo intermedio entre las ubicaciones que comunica cada portal.
De esta forma, la distancia entre cada posición y portal queda definida por la cantidad de vértices intermedios, con lo cual podemos aplicar BFS para obtener el resultado.


Para representar esta conversión, usamos una matriz de $N+1 \times  L+1$ posiciones (incluimos desde la posición $0$ a la $L$ de cada piso), donde cada posición almacena su piso, metro, distancia hasta ella -inicialmente 0, indicando que no fue calculada- y sus vecinos. 
Para cada una, sus vecinos son los destinos de cualquier portal que este ubicado allí y las posiciones contiguas a izquierda y derecha, si existen. A partir de los parámetros de entrada $N$, $L$ y la lista de portales, se construye esta matriz de acuerdo al siguiente procedimiento:


\begin{algorithm}[H]
\caption{Inicialización de Camino Mínimo}\label{init-ej2}
\begin{algorithmic}[3]
\Procedure{Init}{}
\State $\textit{N} \gets \text{cantidad de pisos}$
\State $\textit{L} \gets \text{largo de pasillos}$
\State $\textit{portales} \gets \text{lista de portales}$
\State $mapaPortales \gets $matriz de $N+1$ filas y $L+1$ columnas

\for {$i = 0\text{ hasta }N$}
\for{$j = 0\text{ hasta }L$}
\State $vecinos \gets \text{nueva lista vacía de}Posicion$
\State $pos \gets \text{nueva }Posicion(i,j,vecinos) $
\State $mapaPortales_{i,j} \gets pos$
\EndFor
\EndFor

\for {$i = 0\text{ hasta }N$}
\for{$j = 0\text{ hasta }L$}
\If{$j > 0$}
\State $mapaPortales_{i,j}.vecinos.agregarAtras(mapaPortales_{i,j-1})$
\EndIf
\If{$j < l$}
\State $mapaPortales_{i,j}.vecinos.agregarAtras(mapaPortales_{i,j+1})$
\EndIf
\EndFor
\EndFor


\for {cada $portal$ en $portales$}
\State $d \gets portal.desde$
\State $h \gets portal.hasta$
\State $mapaPortales_{d.piso, d.metros}.vecinos.agregarAtras(mapaPortales_{h.piso, h.metros})$
\State $mapaPortales_{h.piso, h.metros}.vecinos.agregarAtras(mapaPortales_{d.piso, d.metros})$
\EndFor
  

\EndProcedure
\end{algorithmic}
\end{algorithm}


Mientras calculamos distancias entendemos que hay un portal entre una posición y su vecino si están en diferentes pisos, o a mas de un metro de distancia en el mismo piso. 
Si hubiera un portal entre dos ubicaciones contiguas no lo consideramos, dado que es más barato caminar entre ellas.
Cuando existe un portal, encolamos el vértice intermedio, cuyo único vecino es el nodo destino del portal al que corresponde.\\
Se implementa este comportamiento por medio del siguiente algoritmo:



\begin{algorithm}[H]
\caption{Camino Mínimo}\label{alg-ej2}
\begin{algorithmic}[4]
\Procedure{CaminoMinimo}{}
\State $mapaPortales \gets$ matriz de $N$ filas y $L$ columnas construida con el procedimiento \ref{init-ej2}
\State $\textit{verticesRestantes} \gets \text{cola de }Posicion$
\State $\textit{verticesRestantes}.encolar(mapaPortales_{0,0})$

\while {$verticesRestantes$ no vacía}
\State $pos \gets verticesRestantes.desencolar()$
\for{$Posicion destino$ en $pos.vecinos$}
\If{$destino.distancia = 0$}
\If{$hayPortal(pos,destino)$}
\State $portalBuffer \gets$ nueva $Posicion(destino.piso, destino.metro)$
\State $portalBuffer.distancia = pos.distancia + 1$
\State $portalBuffer.vecinos.agregarAtras(destino)$
\State $verticesRestantes.encolar(portalBuffer)$
\Else
\State $destino.distancia = pos.distancia + 1$
\State $verticesRestantes.encolar(destino)$
\EndIf
\EndIf
\EndFor
\EndWhile

\Return $mapaPortales_{N,L}.distancia$


\EndProcedure
\end{algorithmic}
\end{algorithm}


Al terminar el algoritmo, devolvemos el costo de llegar a la última posición del último piso.




\subsection{Correctitud}

Tal como describimos en la sección anterior, representamos los costos añadiendo nodos intermedios. El costo de moverse entre dos posiciones en un piso es de un segundo por metro, es decir, igual a la cantidad de metros. 
En el algoritmo propuesto, construimos la matriz $mapaPortales$ de tamaño $N+1 \times L+1$ donde cada elemento $mapaPortales_{i,j}$ representa la posición en el piso $i$ a los $j$ metros del inicio del pasillo. 
De esta manera, todos los metros de cada pasillo están representados en la matriz. 
Adicionalmente, como se puede observar en el pseudocódigo \ref{init-ej2} (lineas 11 a 16), cada elemento tiene como vecinos a sus posiciones aledañas.

Durante la ejecución del algoritmo tomamos cada elemento de la matriz como un vértice, por lo que la distancia en metros entre dos posiciones del mismo piso resulta ser igual a la cantidad de vértices intermedios.


Análogamente, al agregar un nodo intermedio para cada portal, igualamos el costo de utilizarlos al número de vértices a recorrer para llegar a destino agregando un nodo intermedio, de acuerdo a lo expuesto en el pseudocódigo \ref{alg-ej2} (lineas 9 a 13). 

Finalmente, el algoritmo se reduce a BFS: Se encola un nodo inicial y mientras la cola no este vacía, se desencola un nodo $n$, se encolan todos sus vecinos y se le asigna a cada uno la distancia de $n$ al origen más uno, si no estaba ya calculada.
Por lo tanto, podemos asegurar que el algoritmo es correcto a partir de la correctitud de BFS, probada en la bibliografía \cite{Cormen}.


\subsection{Complejidad}

Para la inicialización de la estructura en la que se basa el algoritmo, de acuerdo a lo descrito en el pseudocódigo \ref{init-ej2}, se itera sobre las $N+1$ filas y $L+1$ columnas de la matriz para inicializarla y luego asignar los vecinos adyacentes a cada posición. Finalmente, se itera sobre la cantidad de portales para añadir las posiciones conectadas a las listas de vecinos correspondientes. La lista de vecinos se implementa sobre una lista enlazada y sus inserciones se realizan en tiempo constante. De igual manera, la construcción de instancias del tipo $Posicion$ también tiene costo $O(1)$. A partir de esto, podemos acotar el costo de inicialización por $O(NL + P)$


Dada la representación elegida, tenemos $NL + P$ vértices, uno por cada metro en cada pasillo, y uno intermedio por cada portal.
Asimismo, la cantidad de aristas por cada posición $p_{i,j}$ es $2 + Portales(p_{i,j})$, donde $i$ es el piso, $j$ el metro y $Portales(x)$ la cantidad de portales con un extremo en la posición $x$.

Como cada portal es bidireccional, se representa en ambos extremos (ver pseudo \ref{alg-ej2}, lineas 18 a 21), por lo que durante la ejecución de BFS se suma dos veces.
Luego, la cantidad de aristas del grafo es $2 (NL + P)$

Durante la ejecución del algoritmo, por cada vértice en la cola se ejecutan operaciones, que dadas las estructuras utilizadas son $O(1)$, tantas veces como vecinos tenga ese nodo.
Si el grafo es conexo, como para cada vértice se encolan sus vecinos no visitados, el ciclo se ejecuta una vez por nodo. Si no fuera conexo, se visitarán menos vértices.

Entonces, cada ejecución del ciclo principal sobre un vértice $v$ tiene un costo $O(1 + E_v)$, con $E_v$ la cantidad de aristas incidentes a $v$. Como en el peor caso (grafo conexo) se visitan todos los nodos del grafo, la sumatoria de este costo sobre todos ellos es $O(|V| + |E|)$, donde $V$ y $E$ son los conjuntos de vértices y aristas del grafo respectivamente. 

Como establecimos previamente, la cantidad de vértices del grafo que consideramos es $NL + P$, y la de aristas es $2 (NL + P)$, por lo que el costo en peor caso suma $O((NL + P) + 2(NL + P))$, que queda acotado por $O( NL + P )$.

\subsection{Experimentación}

Dado que la complejidad temporal del código de inicialización es igual al del algoritmo, decidimos obviarlo de las mediciones de tiempos y concentrar nuestro análisis en el código que resuelve el problema.

Consideramos el grafo completo como el peor caso, ya que al ser conexo se analizan todos los vertices, y cada uno tiene la máxima cantidad de vecinos. En particular, al haber un portal entre el principio del pasillo en planta baja y el final en el último piso, el resultado esperado es $2$.
Al estar dado por la máxima cantidad de portales, que depende del tamaño del grafo ya que cada posición puede tener un portal a cada una de las $NL - 1$ posiciones restantes, observamos los tiempos variando $NL$.

Para el mejor caso consideramos que al tener como única restricción la existencia de un camino entre la primer posición de la planta baja y la última de la mas alta, basta incluir un portal desde el piso $0$ al $N$. Elegimos un portal entre la posición inicial y final, con lo que el resultado esperado es el mismo al del caso anterior.
Análogamente al anterior, este caso se da en la mínima cantidad de portales, por lo que realizamos el mismo análisis en función de $NL$.

En los experimentos realizados se observaron resultados similares al fijar una variable del producto $NL$, aumentando la otra para variarlo.

\begin{figure}[H]
\input{plots/Tp2Ej2Exp1.tex}
\caption{Comparación de tiempos variando NL}
\end{figure}

Omitimos de esta figura las mediciones del mejor caso, dado que involucra $2L$ vértices, pues la componente conexa que contiene la primer posición del piso 0 y la última del piso N esta compuesta por los nodos de ambos pasillos, sus tiempos de ejecución son mucho menores a los de peor caso, aun aumentando $L$ para variar $NL$.


Como no realizamos optimizaciones sobre BFS para especializarlo al problema de la búsqueda de camino mínimo de uno a uno, el costo del algoritmo es el mismo para cualquier par de metros en los pisos elegidos, pues se analizan la misma cantidad de nodos y ejes.
Esto es porque se calcula la distancia a todos los vértices de la componente conexa del nodo original. Debido a la forma de representar el problema que elegimos, en la ausencia de portales tenemos por cada piso una componente conexa. Al agregar portales entre pisos, las unimos. 


Dado que para este algoritmo, una vez que se calcula la distancia a un nodo sabemos que su valor es final, podríamos detener la ejecución ni bien se calcule la distancia a la posición final. Para esto, agregamos al algoritmo descrito en el pseudocódigo \ref{alg-ej2} un chequeo sobre la distancia a la posición en $mapaPortales_{N,L}$ luego de calcular la distancia para cada vecino de una posición, y devolvemos ese valor si ya fue calculado.

La complejidad temporal sigue estando acotada por $O(NL+P)$, pero en ciertos casos 
mejora considerablemente el tiempo de ejecución. 

Con esta modificación, entendemos que el peor caso se dará cuando se deban visitar todos los nodos antes de llegar al vértice objetivo. Esto sucede cuando se cuenta con un grafo completo entre los pisos 0 y N-1, con el último piso conectado en su posición 0 con algún nodo del anterior.

\begin{figure}[H]
\input{plots/Tp2Ej2Exp2.tex}
\caption{Comparación de tiempos variando NL}
\end{figure}

Nuevamente omitimos de la figura el mejor caso. Con esta modificación, en ese escenario solo se calcula la distancia al vértice objetivo independientemente de la cantidad de vértices o portales, por lo que su tiempo de ejecución es casi inmediato.





\section{Ejercicio 3}

\subsection{Introducción}

Esta sección desarrollaremos el ejercicio 3 el cual consiste en programar y explicar una heurística constructiva golosa para resolver el problema de list coloring.

\subsection{Desarrollo}

La heurística que pensamos para este punto consiste en colorear un nodo usando el color, dentro de los posibles, que tenga menos apariciones en los nodos adyacentes a él. En
otras palabras nuestro algoritmo le asigna a cada materia el aula que menos materias adyacentes pueden usar. Para recorrer el grafo utilizaremos un BFS ya que consideramos que
, por la naturaleza de la heurística, es un mecanismo más lógico para avanzar en él, debido a que al asignar a una materia un aula las que estaban superpuestas en horario 
serán las siguientes y las mismas utilizaran en algún punto el resultado anterior ya que al colorear una se quitaran de las adyacentes el color con que se pinto. Por lo tanto,
la segunda en el paso BFS no podrá ser coloreada con el color que se pinto la primera aunque lo poseía en un principio. En el caso que se llegue a una materia que no pueda ser
dictada en ningún aula, porque las adyacentes ya las cubren todas, se le asignara el color entre los descartados que menos aparezca que las superpuestas en horarios. Esto último
lo hacemos para colorear todas de algún color aunque después este coloreo no sea valido.

En el caso particular donde exista una materia que tenga un aula posible que ninguna de sus superpuestas en horario posea, esta será sin dudas una aula posible para dictar la materia. Para ilustrar esto último usamos el siguiente gráfico, 
suponemos que el BFS toma como nodo inicial al número 0:

\begin{tikzpicture}
\node(pseudo) at (-1,0){};
\node(1) at (0,2)[shape=circle,draw]        {$1$};
\node(2) at (0,0)[shape=circle,draw]        {$2$};
\node(3) at (0,-2)[shape=circle,draw]        {$3$};
\node(0) at (2,0)[shape=circle,draw]        {$0$};
\node(4) at (4,2)[shape=circle,draw]        {$4$};
\node(5) at (4,0)[shape=circle,draw]        {$5$};
\node(6) at (4,-2)[shape=circle,draw]        {$6$};
\path [-]
  (1)      edge                 node [above]  {}     (0)
  (2)      edge                 node [above]  {}     (0)
  (3)      edge                 node [above]  {}     (0)
  (0)      edge                 node [above]  {}     (4)
  (0)      edge                 node [above]  {}     (5)
  (0)      edge                 node [above]  {}     (6);
\end{tikzpicture}

\begin{table}[H]
\begin{center}
\begin{tabular}{|l|l|}
\hline
Materias & Aulas en las que se puede dictar \\
\hline \hline
0 & (4, 3) \\ \hline
1 & (1, 5) \\ \hline
2 & (2, 3) \\ \hline
3 & (3) \\ \hline
4 & (1, 3) \\ \hline
5 & (2, 1) \\ \hline
6 & (3, 5) \\ \hline
\end{tabular}
\end{center}
\end{table}

Se ve que en el nodo número 0 no hay ninguna materia vecina que pueda ser dictada en el aula 4. Por lo tanto asignar a esta la misma aula será un coloreo valido para el nodo.
Obviamente existen casos en los que no se cumple esto y, sin embargo, la heurística nos puede devolver un resultado correcto. Para ejemplificar esto usaremos el grafo anterior
cambiando los colores posibles.

\begin{table}[H]
\begin{center}
\begin{tabular}{|l|l|}
\hline
Materias & Aulas en las que se puede dictar \\
\hline \hline
0 & (1, 3, 5) \\ \hline
1 & (1, 5) \\ \hline
2 & (1, 3) \\ \hline
3 & (3) \\ \hline
4 & (1, 3) \\ \hline
5 & (1, 5) \\ \hline
6 & (3, 5) \\ \hline
\end{tabular}
\end{center}
\end{table}

Podemos ver que no se cumple que halla alguna materia que tenga como aula posibles una que no pueda ser usada por las adyacentes. Sin embargo partiendo de cualquier nodo 
podemos colorear el grafo de la siguiente manera.

\begin{table}[H]
\begin{center}
\begin{tabular}{|l|l|}
\hline
Materias & Aula en que se dicta \\
\hline \hline
0 & (5) \\ \hline
1 & (1) \\ \hline
2 & (1) \\ \hline
3 & (3) \\ \hline
4 & (1) \\ \hline
5 & (1) \\ \hline
6 & (3) \\ \hline
\end{tabular}
\end{center}
\end{table}

Para mejorar un poco la heurística se decidió aplicar un pre-procesamiento de el grafo de coloreo que consiste en buscar todas las materias que solo puedan ser dictadas en una
aula y eliminar de sus adyacentes el aula como posible. Se repite esto hasta que no aparezcan más materias con esta característica. Se puede ver que este proceso no impide que 
para una instancia determinada donde se podría encontrar una solución esta no se alcance por la heurística constructiva golosa. Por otro lado hay algunos casos donde puede hacer
que en un caso particular se pueda encontrar un coloreo valido si se aplica en proceso anterior nombrado, y que antes no se encontraba.

\begin{tikzpicture}
\node(pseudo) at (-1,0){};
\node(0) at (0,2)[shape=circle,draw]        {$1$};
\node(1) at (0,0)[shape=circle,draw]        {$2$};
\node(2) at (0,-2)[shape=circle,draw]        {$3$};
\node(3) at (2,0)[shape=circle,draw]        {$0$};
\node(4) at (4,0)[shape=circle,draw]        {$4$};

\path [-]
  (0)      edge                 node [above]  {}     (3)
  (1)      edge                 node [above]  {}     (3)
  (2)      edge                 node [above]  {}     (3)
  (3)      edge                 node [above]  {}     (4);
\end{tikzpicture}


\begin{table}[H]
\begin{center}
\begin{tabular}{|l|l|}
\hline
Materias & Aulas en las que se puede dictar \\
\hline \hline
0 & (1, 2) \\ \hline
1 & (2, 3) \\ \hline
2 & (2, 4) \\ \hline
3 & (2, 5) \\ \hline
4 & (1) \\ \hline
\end{tabular}
\end{center}
\end{table}

Supongamos que comenzamos el BFS en el nodo 0. Entonces llegaremos al siguiente coloreo:

\begin{table}[H]
\begin{center}
\begin{tabular}{|l|l|}
\hline
Materias & Aula en las que se dictar \\
\hline \hline
0 & (1) \\ \hline
1 & (2) \\ \hline
2 & (2) \\ \hline
3 & (2) \\ \hline
4 & (1) \\ \hline
\end{tabular}
\end{center}
\end{table}

El cual no es valido. Sin embargo aplicando el pre-procesamiento anteriormente mencionado podemos alcanzar la siguiente solución.

\begin{table}[H]
\begin{center}
\begin{tabular}{|l|l|}
\hline
Materias & Aula en las que se dictar \\
\hline \hline
0 & (2) \\ \hline
1 & (3) \\ \hline
2 & (4) \\ \hline
3 & (5) \\ \hline
4 & (1) \\ \hline
\end{tabular}
\end{center}
\end{table}

La cual si es valida.

Obviamente también hay cirscuntancias donde la heurística va a fallar. Cambiando levemente el ejemplo anterior obtenemos lo siguiente.

\begin{tikzpicture}
\node(pseudo) at (-1,0){};
\node(0) at (0,2)[shape=circle,draw]        {$1$};
\node(1) at (0,0)[shape=circle,draw]        {$2$};
\node(2) at (0,-2)[shape=circle,draw]        {$3$};
\node(3) at (2,0)[shape=circle,draw]        {$0$};
\node(4) at (4,0)[shape=circle,draw]        {$4$};

\path [-]
  (0)      edge                 node [above]  {}     (3)
  (1)      edge                 node [above]  {}     (3)
  (2)      edge                 node [above]  {}     (3)
  (3)      edge                 node [above]  {}     (4)
  (2)      edge                 node [above]  {}     (4);
\end{tikzpicture}


\begin{table}[H]
\begin{center}
\begin{tabular}{|l|l|}
\hline
Materias & Aulas en las que se puede dictar \\
\hline \hline
0 & (1, 2) \\ \hline
1 & (2, 5) \\ \hline
2 & (2, 3) \\ \hline
3 & (1, 5) \\ \hline
4 & (1, 5) \\ \hline
\end{tabular}
\end{center}
\end{table}

Hay que notar que el pre-procesamiento no realizara ningun cambio. El resultado de aplicar nuestra heurística es el siguiente:

\begin{table}[H]
\begin{center}
\begin{tabular}{|l|l|}
\hline
Materias & Aulas en las que se puede dictar \\
\hline \hline
0 & (1) \\ \hline
1 & (2) \\ \hline
2 & (2) \\ \hline
3 & (5) \\ \hline
4 & (5) \\ \hline
\end{tabular}
\end{center}
\end{table}

Y este es un coloreo invalido porque 3 y 4 son adyacentes y los dos estan coloreados con 5.

\subsection{Complejidad}

Tomamos n como la cantidad de vértices, m como la cantidad de aristas y c como la cantidad máxima de colores disponibles.

La complejidad del pre-procesamiento es el de recorrer cada uno de los vértices del grafo en busca de nodos con un solo color posible. Esto lo haremos, a lo sumo n veces.
En cada paso al encontrar una con un solo color posibles, procederemos a eliminar ese color de los adyacentes. El costo de eliminar un color de los adyacentes es en peor caso
O(c*n), mientras que recorrer el grafo a lo sumo n veces nos cuesta O($n^2$). En otras palabras la complejidad en peor caso del pre-procesamiento es de O($c*n^3$).

En cuanto a la complejidad del BFS, esta es la complejidad de recorrer el grafo con BFS multiplicado por lo que nos cuesta el coloreo en cada paso. BFS nos cuesta O($n*m$) en 
peor caso, mientras que buscar el color que menos aparece en los adyacentes nos cuesta O($n*c^2$) sumado a lo que nos cuesta eliminar el color elegido de los adyacentes lo que 
nos cuesta O($c*n$). En otras palabras la complejidad en peor caso del BFS es de O($n*m*(n*c^2 + c*n)$) = O($m*n^2*c^2$).

Por lo tango la complejidad en peor caso de nuestra heurística constructiva golosa es de O($m*n^3*c^2$).







\subsection{Experimentación}

\section{Bibliografia}

\begin{thebibliography}{1}
	\bibitem{}Cormen, Thomas H, and Charles E Leiserson. \textit{Introduction To Algorithms, 3Rd Edition}. 
\end{thebibliography}


\newpage

\ref{LastPage}

\end{document}
