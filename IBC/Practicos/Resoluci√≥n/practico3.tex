%\begin{center}
%\large \bf \runtitulo
%\end{center}
%\vspace{1cm}
\chapter{Práctico Tema 3}

1. Las creencias del termostato son que siempre sabe cuando la temperatura, además de saber si la caldera está o no apagada. Sus deseo es que la temperatura siempre este entre 3 grados por debajo o por arriba de la temperatura dada. Su intención es que siempre que la temperatura este por debajo, encenderá la caldera para aumentarla, y la apagará cuando este por encima para disminuirla. \\


2. La cantidad de semillas en una naranja específica. \\

Se que las frutas son frutas porque tienen semillas, mi naranja tiene semillas; por lo tanto, mi naranja es una fruta. \\

En un grupo de amigos cada uno conoce los nombres, y o formas de referirse, a los otros. \\

Conocimiento interno de conocimiento externo: conocimiento como una relación entre un sujeto cognitivo y partes de la realidad. 
El conocimiento de que un perro desconocido mostrando los dientes es una señal de amenaza. \\

3. Supongamos dos KBs, una cita a todas las personas registradas que son programadores, la segunda lista a todas las personas registradas que son mujeres.

\begin{table}[]
	\begin{tabular}{lllll}
		Persona       & Especialidad &  &  &  \\
		Karen Sanchez & Programadora &  &  &  \\
		Karen Carles & Programadora  &  &  &  \\
		&              &  &  & 
	\end{tabular}
\end{table}


\begin{table}[]
	\begin{tabular}{lllll}
		Persona       & Especialidad &  &  &  \\
		Karen Sanchez & Programadora &  &  &  \\
		Karen Carles & Programadora  &  &  &  \\
		Emma Perez & Medica       &  &  &  \\
		&              &  &  & 
	\end{tabular}
\end{table}

La segunda base claramente contiene a la primera. Si tuviéramos un programador hombre, que no estuviera en la primer base, entonces la suposición de mundo cerrado no se cumpliría porque la segunda no lo puede contener.


4. Mundo de bloques: modele el mundo de bloques en lógica proposicional. Debe modelar el estado de la mesa: tenemos 3 bloques (y solo 3) A, B y C. Debemos poder expresar que los bloques están sobre la mesa, o sobre otro bloque. También queremos expresar la siguiente propiedad (para los tres bloques): un bloque se puede desapilar siempre y cuando esté sobre otro bloque y no tenga un bloque encima. ¿Cual sería la KB que represente como estado inicial el hecho de que A está sobre la mesa, B está sobre A y C está sobre la mesa? Muestre que para la consulta “¿se puede desapilar A?”, la respuesta es negativa, usando ambos algoritmos de inferencia (encadenamiento hacia atrás y hacia adelante). Asuma CWA. Recuerde que además de las reglas que defina en la teoría, puede utilizar las reglas de inferencia y equivalencias lógicas para la lógica proposicional (Artificial Intelligence A Modern Approach. Sección 7 – tercera Edición). Una lista completa de las reglas puede encontrarse en: https://es.wikipedia.org/wiki/Anexo:Reglas_de_inferencia

5. ¿Qué tipo de razonamiento cree usted que lleva a cabo un algoritmo de clasificación basado en datos, de acuerdo a los tipos vistos en clase?

6. Describa otro ejemplo concreto que muestre la necesidad de razonamiento no monótono, similar al de “Tweety”; muestre por qué la lógica clásica falla en ese caso en particular.


\bigskip