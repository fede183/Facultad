%\begin{center}
%\large \bf \runtitulo
%\end{center}
%\vspace{1cm}
\chapter{Práctico Tema 3}

1. Las creencias del termostato son que siempre sabe cuando la temperatura, además de saber si la caldera está o no apagada. Sus deseo es que la temperatura siempre este entre 3 grados por debajo o por arriba de la temperatura dada. Su intención es que siempre que la temperatura este por debajo, encenderá la caldera para aumentarla, y la apagará cuando este por encima para disminuirla. \\


2. La cantidad de semillas en una naranja específica. \\

Se que las frutas son frutas porque tienen semillas, mi naranja tiene semillas; por lo tanto, mi naranja es una fruta. \\

En un grupo de amigos cada uno conoce los nombres, y o formas de referirse, a los otros. \\

Conocimiento interno de conocimiento externo: conocimiento como una relación entre un sujeto cognitivo y partes de la realidad. 
El conocimiento de que un perro desconocido mostrando los dientes es una señal de amenaza. \\

3. Supongamos dos KBs, una cita a todas las personas registradas que son programadores, la segunda lista a todas las personas registradas que son mujeres.

\begin{table}[h]
	\begin{tabular}{l|l}
		Persona       & Especialidad \\ \hline
		Karen Sanchez & Programadora \\
		Karen Carles & Programadora  \\
	\end{tabular}
\end{table}


\begin{table}[h]
	\begin{tabular}{l|l}
		Persona       & Especialidad  \\ \hline
		Karen Sanchez & Programadora  \\
		Karen Carles & Programadora   \\
		Emma Perez & Medica           \\
	\end{tabular}
\end{table}

La segunda base claramente contiene a la primera. Si tuviéramos un programador hombre, que no estuviera en la primer base, entonces la monotonía derivada de la suposición de mundo cerrado no se cumpliría porque la segunda no lo puede contener.\\

4. Si X en mesa lo que estoy diciendo es que el bloque representado por X está sobre la mesa. Si X en Y, con $X \neq Y$, digo que el bloque X está sobre el Y. Además de esto tengo la siguientes reglas: \\

$\neg X \enBloque X$ \\

$\neg X \enBloque Y \wedge \neg X \enBloque Z \wedge Y \neq Z \wedge X \neq Y \wedge X \neq Z \iff X \enBloque mesa$ \\

$X = A \lor X = B \lor X = C$ \\

$Y = A \lor Y = B \lor Y = C$ \\

$Z = A \lor Z = B \lor Z = C$ \\

La primera dice que un bloque no puede estar encima de si mismo, y la segunda que si el bloque X no está en encima de Y o Z, siendo estos tres bloques distintos, entonces X está sobre la mesa. Por último los valores de X, Y y Z se restringen a los tres bloques. Debemos considerar también el desapilar. Para eso tenemos esta regla:\\

$X\ se\ puede\ desapilar \iff \neg X \enBloque mesa \wedge \neg Y \enBloque X \wedge \neg Z \enBloque X \wedge Y \neq Z \wedge X \neq Y \wedge X \neq Z$ \\

Veremos ahora la KB sería: $\{B \enBloque A\}$. Dado que al no decir si A o C están sobre ningún bloque se pueden inferir, asumiendo CWA, que están sobre la mesa. Ahora procederemos a responder la consulta de si se puede desapilar A. \\

Encadenamiento para adelante: \\

Por CWA, podemos deducir los siguientes hechos:\\

\begin{enumerate}
	\item $\neg A \enBloque B$
	\item $\neg A \enBloque C$
	\item $B \enBloque A$
	\item $\neg B \enBloque C$
	\item $\neg C \enBloque A$
	\item $\neg C \enBloque B$
	\item $A \neq B$
	\item $A \neq C$
	\item $B \neq C$
\end{enumerate}

Si combinamos 1, 2, 7, 8 y 9 tenemos: \\

$\neg A \enBloque B \wedge \neg A \enBloque C \wedge B \neq C \wedge A \neq B \wedge A \neq C$ \\

Según la regla de mesa, esto implica que $A \enBloque mesa$. Si a su vez, combinamos está conclusión con los hechos 3, 5, 7, 8, 9, tenemos lo siguiente:\\ 

$\neg A \enBloque mesa \wedge \neg B \enBloque A \wedge \neg C \enBloque A \wedge B \neq C \wedge A \neq B \wedge A \neq C$ \\

Y con esto podemos derivar que A se puede desapilar por la regla de desapilación. Pero esto se deduce como falso ya que por un lado $A \enBloque mesa$, por la regla anterior, y $B \enBloque A$, este es un hecho.\\

Encadenamiento para atrás: \\

Comenzando por la regla del desapilado, que es lo que queremos probar. \\ 

$\neg A \enBloque mesa \wedge \neg B \enBloque A \wedge \neg C \enBloque A \wedge B \neq C \wedge A \neq B \wedge A \neq C \iff A\ se\ puede\ desapilar$ \\

Por el lado izquierdo, tenemos $\neg A \enBloque mesa$. Usando la regla de la mesa podemos ver que: \\

$\neg A \enBloque B \wedge \neg A \enBloque C \wedge B \neq C \wedge A \neq B \wedge A \neq C \iff A \enBloque mesa$ \\

Se puede ver que a la izquierda cada uno de los literales es un hecho de la lista; 1, 2, 7, 8 y 9. Por lo tanto podemos concluir que $A \enBloque mesa$. Pero esto no es lo que queríamos, entonces tenemos que A no se puede dasapilar.\\


5. Un algoritmo de clasificación es en esencia un pattern matching. La idea es que tendremos diferentes patrones (nuestras clasificaciones) y lo que queremos es poder asociar a cual de estos responde mejor el dato (es más similar o es concuerda a uno y no se hace nada con los otros patrones). En mi opinión esto se acerca al razonamiento analógico ya que la idea no siempre es buscar coincidencia plena sino aproximada; si el dato "se parece" a este patrón, entonces el dato se clasifica de acuerdo a este. \\

6. 
\begin{itemize}
	\item Las playas están compuestas de arena.
	\item Algunas playas en el mundo están compuestas por una mezcla de arena y otros elementos, como botellas de plástico. Entre ellas se encuentra la playa de Mar del Plata.
	\item La playa donde fui de vacaciones está compuesta por pura arena.
\end{itemize}

Como se puede ver este es un ejemplo de razonamiento no-monótono ya que se está llegando a una conclusión a partir de información incompleta. Si se agrega la frase "La playa donde fui de vacaciones fue la Mar del Plata", se puede ver la conclusión debe retractarse y se debería cambiar por "La playa donde fui de vacaciones no está compuesta por pura arena". \\

Ahora si queremos representar esto con lógica clásica: \\

"Las playas están compuestas de arena" \\

$\forall X (esPlaya(X) \land \neg excepcion(X) \longrightarrow elContenidoEsArena(X))$ \\

Tenemos que agregar una regla para las excepciones: \\

$\forall X (excepcion(X) \iff \neg marDelPlata(X) \land ... )$ \\

No sé todas las excepciones y para poder llegar a la conclusión de que el contenido es arena, debería afirmar que la playa no es ninguna de las excepciones. Con esto concluyo que no puedo realizar una representación de este razonamiento con lógica clásica. \\

Dado que la información está incompleta el razonamiento presentado es no-monótono. 

\bigskip