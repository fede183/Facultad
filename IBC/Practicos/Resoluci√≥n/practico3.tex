%\begin{center}
%\large \bf \runtitulo
%\end{center}
%\vspace{1cm}
\chapter{Práctico Tema 3}

1. En el ejemplo del termostato: ahora podemos pensarlo como un agente racional: – ¿Qué creencias tendría el agente? ¿Qué deseos? ¿Qué intenciones? Describa con sus palabras en lenguaje informal. 

2. Notemos que este tipo de razonamiento no es monótono, i.e. puede suceder que KB1 $\subseteq$ KB2 y sin embargo CWA(KB1) $\nsubseteq$ CWA(KB2). Ejercicio: muestre un ejemplo concreto de este resultado. 

3. Mundo de bloques: modele el mundo de bloques en lógica proposicional. Debe modelar el estado de la mesa: tenemos 3 bloques (y solo 3) A, B y C. Debemos poder expresar que los bloques están sobre la mesa, o sobre otro bloque. También queremos expresar la siguiente propiedad (para los tres bloques): un bloque se puede desapilar siempre y cuando esté sobre otro bloque y no tenga un bloque encima. Cual sería la KB que represente como estado inicial el hecho de que A está sobre la mesa, B está sobre A y C está sobre la mesa? Muestre que para la consulta “se puede desapilar A?”, la respuesta es negativa, usando ambos algoritmos de inferencia (encadenamiento hacia atrás y hacia adelante). Asuma CWA. Recuerde que además de las reglas que defina en la teoría, puede utilizar las reglas de inferencia y equivalencias lógicas para la lógica proposicional (Artificial Intelligence A Modern Approach. Sección 7 – tercera Edición). Una lista completa de las reglas puede encontrarse en: https://es.wikipedia.org/wiki/Anexo:Reglas_de_inferencia

4. ¿Que tipo de razonamiento cree usted que lleva a cabo un algoritmo de clasificación basado en datos, de acuerdo a los tipos vistos en clase? 

5. Describa otro ejemplo concreto que muestre la necesidad de razonamiento no monótono, similar al de “Tweety”; muestre por qué la lógica clásica falla en ese caso en particular.

\bigskip