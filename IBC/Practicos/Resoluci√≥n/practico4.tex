%\begin{center}
%\large \bf \runtitulo
%\end{center}
%\vspace{1cm}
\chapter{Práctico Tema 4}

1. AbuelaMaterna $\subseteq$ Madre $\sqcup$ $\exists$hijo-de.Madre


PadreOMadreDeMasDeDos $\subseteq$ Persona $\sqcup$ <=2 ($\exists$hijo-de.Persona)

HermanoOHermana $\subseteq$ Persona $\sqcup$ $\exists$hijo-de-.PadreOMadreDeMasDeDos

TioSinHijos $\subseteq$ Hombre $\sqcup$ HermanoOHermana $\sqcup$ $\neg$ $\exists$hijo-de.Persona


AbueloOAbuelaDeMasDeDos $\subseteq$ Persona $\sqcup$ <=2 ($\exists$hijo-de.PadreOMadre)

HermanoOHermanaPadreOMadre $\subseteq$ Persona $\sqcup$ $\exists$hijo-de-.AbueloOAbuelaDeMasDeDos

SobrinoDeUnaNuera $\subseteq$ Hombre $\sqcup$ $\exists$hijo-de-.(HermanoOHermanaPadreOMadre)


PadreConAlMenos3HijosDosMujeres $\subseteq$ Persona $\sqcup$ <=3 ($\exists$hijo-de.Persona) $\sqcup$ <=2 ($\exists$hijo-de.Mujer) $\sqcup$ >=2 ($\exists$hijo-de.Mujer)



2. Dado el programa Datalog anterior, indique si: • Se puede derivar parent(evan, david)? • Se puede derivar ancestor(carla, david)? En el caso donde se pueda muestre el árbol de derivación. 

3. Dada la consulta conjunctiva: Q(X,Y) = $\exists$Z.parent (X,Y) $\bigwedge$ parent (Z,Y) ¿Qué regla debemos agregarle a P (slide 49)? ¿Cual sería el conjunto de respuesta? Verifiquelo en ABCDatalog.

4. ¿Cómo sería la regla que hay que agregar en P (slide 49) dada la consulta Booleana: Q1() = $\exists$Z.parent (X,Y) $\bigwedge$ parent (Z,Y)? 

5. Escriba un programa Datalog para resolver el problema del mundo de bloques planteado en el práctico 3. Generalice su programa de manera que se pueda hacer las siguientes consultas: para cualquier bloque X, Y, “está el bloque X sobre la mesa”?, “está X apilado sobre Y”?, “está X debajo de Y?, “esta X inmediatamente sobre Y”?, “esta X inmediatamente por debajo de Y”? Pruebe la correctitud del programa utilizando la herramienta ABCDatalog. 

6. Proponga un ejemplo de dos esquemas de bases de datos (fuente y destino), muestre las s-t-TGDs necesarias para pasar de uno a otro, similar a slide 91, explique la transformación necesaria. Defina una instancia de bases de datos en el esquema fuente y muestre cual seria la solución (la instancia destino) para el problema de intercambio de datos.

\bigskip