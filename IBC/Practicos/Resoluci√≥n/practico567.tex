\chapter{Práctico Tema 5, 6 y 7}

1.a. El fragmento es "guarded" porque la única variable es $X$ y $phdStudent(X) → student(X)$ la contiene. Por otro lado no es "linear" porque existen otros átomos aparte de este último. \\

b. $chase(D, \Sigma_{T}) = D \cup \{...\}$ \\

Por "$phdStudent(john)$".\\

$chase(D, \Sigma_{T}) = \{phdStudent(john)\} \cup \{student(john), supervisor(z1, john) \}$ \\

Por "$phdStudent(X) \Longrightarrow student(X)$". \\

$chase(D, \Sigma_{T}) = \{phdStudent(john)\} \cup \{student(john), ...\}$\\

Por "$phdStudent(X) \Longrightarrow \exists Y supervisor(Y, X)$". \\

$chase(D, \Sigma_{T}) = \{phdStudent(john)\} \cup \{student(john), supervisor(z1, john) \}$ \\

c. La respuesta de $Q() = \exists X supervisor(X, john)$ es $True$ ya que por el átomo "$phdStudent(X) \Longrightarrow \exists Y supervisor(Y, X)$" y por "$phdStudent(john)$". \\

Por otro lado la respuesta a $Q(X) = supervisor(X, john)$ es \{\}. Ya que la base de datos D no nos provee más información que el hecho de que 
esta "$phdStudent(X)$". \\


3. a
a. $F=\{A, R\}$ donde: \\
$A=\{A, B, C, D, E, F, G\}$ \\
$R=\{(A, B), (A, F), (B, D), (D, C), (C, E), (E, F), (G, C)\}$ \\

b. $\{B, F\}$ es un conjunto que es conflict free ya que ninguno de los dos miembros del mismo ataca al otro. Pero no es admisible porque tanto B como F son atacados por A y ninguno se puede defender de este.\\

c. $\{A, D, E\}$ ninguno se ataca con lo cual es conflict free, E es atacado por C, pero es defendido por D, D es atacado por B pero es defendido por A. A no es atacado por nadie. Con lo cual es admissible. \\

$\{G, E\}$ ninguno se ataca con lo que es conflict free, E es atacado por C pero defendido por G, nadie ataca a G. Con lo cual es admissible.

\bigskip