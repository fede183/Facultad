\chapter{Práctico Tema 5, 6 y 7}

1.a. El fragmento es "guarded" porque la única variable es $X$ y $phdStudent(X) \Longrightarrow student(X)$ la contiene. Por otro lado no es "linear" porque existen otros átomos aparte de este último. \\

b. $chase(D, \Sigma_{T}) = D \cup \{...\}$ \\

Por "$phdStudent(john)$".\\

$chase(D, \Sigma_{T}) = \{phdStudent(john)\} \cup \{ ... \}$ \\

Por "$phdStudent(X) \Longrightarrow student(X)$". \\

$chase(D, \Sigma_{T}) = \{phdStudent(john)\} \cup \{student(john), ...\}$\\

Por "$phdStudent(X) \Longrightarrow \exists Y supervisor(Y, X)$". \\

$chase(D, \Sigma_{T}) = \{phdStudent(john)\} \cup \{student(john), supervisor(z1, john) \}$ \\

c. La respuesta de $Q() = \exists X supervisor(X, john)$ es $True$ ya que por el átomo "$phdStudent(X) \Longrightarrow \exists Y supervisor(Y, X)$" y por "$phdStudent(john)$". \\

Por otro lado la respuesta a $Q(X) = supervisor(X, john)$ es \{\}. Ya que la base de datos D no nos provee más información que el hecho de que 
esta "$phdStudent(X)$". \\

2.a. $K = \{p, p \Longrightarrow r , r \land s \Longrightarrow \neg p \}$ tengo que calcular el partial meet contration utilizando la formulación $K-_{\gamma}r = \cap\gamma(K \bot r)$. \\

Primero es necesario obtener $K \bot r$, el cual es el conjunto de subconjuntos maximales de A que fallan al implicar r. Tenemos dos posibilidades eliminar $p$ o $p \Longrightarrow  r$, con lo cual tenemos estos dos reminders: \\

$\{p \Longrightarrow r , r \land s \Longrightarrow \neg p \}$ \\
	
$\{p, r \land s \Longrightarrow \neg p \}$ \\

Usando full meet contraction tenemos que debemos seleccionar el resultado de la contracción buscando lo que los reminders tienen en común. \\

$\{p \Longrightarrow r , r \land s \Longrightarrow \neg p \} \cap \{p, r \land s  \Longrightarrow \neg p \} = \{r \land s \Longrightarrow \neg p\}$ \\

$\{r \land s  \Longrightarrow \neg p\}$ es el resultado de la contracción. \\

b. Tengo que calcular la contracción kernel revisión utilizando la formulación $K-_{\sigma}r = K \backslash \sigma(K \| r)$. \\

Primero es necesario obtener $K \| r$, el cual es el conjunto de subconjuntos minimales de A que implican a r. En este caso solamente existe un r-kernel $\{p, p \Longrightarrow r\}$, se puede ver que si quito $p$ entonces no se puede implicar r y si quito $p \Longrightarrow  r$ entonces r ni siquiera aparece. \\

Queda aplicar la función de incisión y sustraer los resultados de K. Se pide dos posibles funciones de incisión; la primera que voy a tomar va a elegir el primero, en este caso $p$, y la segunda elige el otro. Entonces me quedan: \\

$K-_{\sigma_{1}}r = K \backslash \sigma_{1}(K \| r)$ \\
$K-_{\sigma_{1}}r = K \backslash \sigma_{1}(\{p, p \Longrightarrow r\})$ \\
$K-_{\sigma_{1}}r = K \backslash \{p\}$ \\
$K-_{\sigma_{2}}r = \{p \Longrightarrow r , r \land s \Longrightarrow \neg p \}$ \\

$K-_{\sigma_{2}}r = K \backslash \sigma_{2}(K \| r)$ \\
$K-_{\sigma_{2}}r = K \backslash \sigma_{2}(\{p, p \Longrightarrow r\})$ \\
$K-_{\sigma_{2}}r = K \backslash \{p \Longrightarrow r\}$ \\
$K-_{\sigma_{2}}r = \{p, r \land s \Longrightarrow \neg p \}$ \\

c. Por la identidad de Levi, $K*s = (K - \neg s) + s$. Se puede notar que tanto $s$ como $\neg s$ están indeterminadas en K. Por vacuity de revisión tenemos que $K*s = K+s$. Entonces: \\

$K*s = \{p, p \Longrightarrow r , r \land s \Longrightarrow \neg p \} \cup {s} $\\

Por lo tanto, $K*s = \{p, p \Longrightarrow r , r \land s \Longrightarrow \neg p, s \}$ \\

3. a. $F=\{A, R\}$ donde: \\
$A=\{A, B, C, D, E, F, G\}$ \\
$R=\{(A, B), (A, F), (B, D), (D, C), (C, E), (E, F), (G, C)\}$ \\

b. $\{B, F\}$ es un conjunto que es conflict free ya que ninguno de los dos miembros del mismo ataca al otro. Pero no es admisible porque tanto B como F son atacados por A y ninguno se puede defender de este.\\

c. $\{A, D, E\}$ ninguno se ataca con lo cual es conflict free, E es atacado por C, pero es defendido por D, D es atacado por B pero es defendido por A. A no es atacado por nadie. Con lo cual es admisible. \\

$\{G, E\}$ ninguno se ataca con lo que es conflict free, E es atacado por C pero defendido por G, nadie ataca a G. Con lo cual es admisible.

d. 

\bigskip