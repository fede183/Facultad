\chapter{Práctico Tema 5, 6 y 7}

3. 
a. $F=\{A, R\}$ donde: \\
$A=\{A, B, C, D, E, F, G\}$ \\
$R=\{(A, B), (A, F), (B, D), (D, C), (C, E), (E, F), (G, C)\}$ \\

b. $\{B, F\}$ es un conjunto que es conflict free ya que ninguno de los dos miembros del mismo ataca al otro. Pero no es admisible porque tanto B como F son atacados por A y ninguno se puede defender de este.\\

c. $\{A, D, E\}$ ninguno se ataca con lo cual es conflict free, E es atacado por C, pero es defendido por D, D es atacado por B pero es defendido por A. A no es atacado por nadie. Con lo cual es admissible. \\

$\{G, E\}$ ninguno se ataca con lo que es conflict free, E es atacado por C pero defendido por G, nadie ataca a G. Con lo cual es admissible.

\bigskip