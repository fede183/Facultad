%\begin{center}
%\large \bf \runtitulo
%\end{center}
%\vspace{1cm}
\chapter{Práctico Tema 2: Agentes}

1. Implementar Algoritmo genérico de búsqueda (ver Slide 63)

2. Modifique mínimamente el algoritmo del punto anterior para implementar los métodos de búsquedas “Primero en profundidad” y “Primero a lo ancho”.

3. ¿Es el método de búsqueda “Iterative deeping” completo? ¿Por qué?

4. Defina en sus propios términos los siguientes términos: autonomía, agente reflejo, agente basado en metas, agente basado en utilidades.

5. Escriba un programa en pseudo-codigo para los agentes basados en metas y basados en utilidad.

6. Consideremos un termostato simple que enciende una caldera cuando la temperatura esta al menos tres grados por debajo de la temperatura seteada, y la apaga cuando la temperatura esta al menos 3 grados por encima de la temperatura seteada. El termostato, ¿es una instancia de un agente reactive? ¿un agente basado en metas? ¿o un agente basado en utilidad?\textbf{}

\bigskip