%\begin{center}
%\large \bf \runtitulo
%\end{center}
%\vspace{1cm}
\chapter{Práctico Tema 2: Agentes}

\lstset{style=csharpstyle, escapeinside={<@}{@>}}

1.\\

\begin{center}
	\begin{lstlisting}
		funcion TreeSearch(Raiz, ArbolBusqueda, Estrategia) {
			Frontera = []
			Frontera.add(Raiz)
			SelectElement = Raiz
			PotencialesCaminos = [[Raiz]]
			
			while (!esMeta(SelectElement)) {
				
				Frontera = Estrategia.nuevaFrontera(ArbolBusqueda, Frontera, SelectElement)
				
				if (Frontera.lenght == 0) {
					return failure
				}
				
				SelectElement = Estrategia.select(ArbolBusqueda, Frontera)
				
				AgregarAPotencialesCaminos(ArbolBusqueda, PotencialesCaminos, SelectElement)
			}
		
			AgregarAPotencialesCaminos(ArbolBusqueda, PotencialesCaminos, SelectElement)
		
			Camino = PotencialesCaminos.last()
		
			return Camino
			
		}
	
		function AgregarAPotencialesCaminos(ArbolBusqueda, PotencialesCaminos, Elemento) {
			
			foreach PotencialCamino en PotencialesCaminos {
				UltimoElementoDelCamino = PotencialCamino.lastElement()
				
				if (Elemento en ArbolBusqueda.vecinos(UltimoElementoDelCamino)) {
					UltimoElementoDelCamino.add(Elemento)
				}
			}
				
			
		}
	\end{lstlisting}
\end{center}

2.\\

\begin{center}
	\begin{lstlisting}
		funcion PrimeroProfundidad(Raiz, ArbolBusqueda) {
			Frontera = []
			Frontera.add(Raiz)
			SelectElement = Raiz
			PotencialesCaminos = [[Raiz]]
			Visitados = []
			
			
			while (!esMeta(SelectElement)) {
				
				Vecinos = ArbolBusqueda.vecinos(SelectElement)
				
				Frontera.remove(SelectElement)
				
				if (Vecinos != []) {
					Frontera = Vecinos + Frontera			
				}
				
				if (Frontera.lenght == 0) {
					return failure
				}
				
				SelectElement = Frontera.first()
				
				AgregarAPotencialesCaminos(ArbolBusqueda, PotencialesCaminos, SelectElement)
			}
			
			AgregarAPotencialesCaminos(ArbolBusqueda, PotencialesCaminos, SelectElement)
			
			Camino = PotencialesCaminos.last()
			
			return Camino
			
		}


		funcion PrimeroAncho(Raiz, ArbolBusqueda) {
			Frontera = []
			Frontera.add(Raiz)
			SelectElement = Raiz
			PotencialesCaminos = [[Raiz]]
			
			while (!esMeta(SelectElement)) {
				
				Vecinos = ArbolBusqueda.vecinos(SelectElement)
				
				Frontera.remove(SelectElement)
				
				Frontera = Frontera	+ Vecinos
				
				if (Frontera.lenght == 0) {
					return failure
				}
				
				SelectElement = Frontera.first()
				
				AgregarAPotencialesCaminos(ArbolBusqueda, PotencialesCaminos, SelectElement)
			}
			
			AgregarAPotencialesCaminos(ArbolBusqueda, PotencialesCaminos, SelectElement)
			
			Camino = PotencialesCaminos.last()
			
			return Camino
			
		}
	
	\end{lstlisting}
\end{center}


3. El algoritmo de búsqueda Iterative deeping es completo. La razón es que al correr búsqueda en profundidad repetidamente con niveles de profundidades incrementales, se tiene efectivamente un algoritmo del cual el orden conmutativo de los nodos que visita es el de búsqueda en lo ancho. Este último es un algoritmo de búsqueda completo, con lo cual, Iterative deeping también lo és.\\

4.\\

Autonomía: La capacidad de elegir si actuar o no, y en que forma hacerlo.\\

Agente reflejo: Es un tipo de agente que observa y si ve algo determinado realiza una acción. Caso contrario, no hace nada o hace otra cosa. Esto en base a reglas de condición-acción. No posee un estado interno. \\

Agente basado en metas: Posee un estado interno que le permite determinar las consecuencias de realizar una acción. A su vez posee metas a alcanzar. De todas las acciones posibles, descarta las que no tienen que ver con las metas que posee, sin importar cual de las que la lleva a la meta es la mejor. \\

Agente basado en utilidades: En lugar de metas se tiene una utilidad, en estos casos puede haber más de una acción que lleva a la meta. La utilidad sirve para que se tomen las “mejores” decisiones.\\


5. Escriba un programa en pseudo-codigo para los agentes basados en metas y basados en utilidad.

6. Consideremos un termostato simple que enciende una caldera cuando la temperatura esta al menos tres grados por debajo de la temperatura seteada, y la apaga cuando la temperatura esta al menos 3 grados por encima de la temperatura seteada. El termostato, ¿es una instancia de un agente reactive? ¿un agente basado en metas? ¿o un agente basado en utilidad?\textbf{}

\bigskip