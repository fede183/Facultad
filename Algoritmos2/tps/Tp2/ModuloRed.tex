\documentclass[a4paper,10pt]{article}
\usepackage[paper=a4paper, hmargin=1.5cm, bottom=1.5cm, top=3.5cm]{geometry}
\usepackage[latin1]{inputenc}
\usepackage[T1]{fontenc}
\usepackage[spanish]{babel}
\usepackage{xspace}
\usepackage{xargs}
\usepackage{ifthen}
\usepackage{algorithmicx}
\usepackage{algpseudocode}
\usepackage{algorithm}
\usepackage{aed2-tad,aed2-symb,aed2-itef}
\usepackage[T1]{fontenc}
\usepackage{selinput}
\SelectInputMappings{%
  aacute={á},
  ntilde={ñ},
  Euro={€}
}
\usepackage{babel}

\newcommand{\moduloNombre}[1]{\textbf{#1}}

\let\NombreFuncion=\textsc
\let\TipoVariable=\texttt
\let\ModificadorArgumento=\textbf
\newcommand{\res}{$res$\xspace}
\newcommand{\tab}{\hspace*{7mm}}

\newcommandx{\TipoFuncion}[3]{%
  \NombreFuncion{#1}(#2) \ifx#3\empty\else $\to$ \res\,: \TipoVariable{#3}\fi%
}
\newcommand{\In}[2]{\ModificadorArgumento{in} \ensuremath{#1}\,: \TipoVariable{#2}\xspace}
\newcommand{\Out}[2]{\ModificadorArgumento{out} \ensuremath{#1}\,: \TipoVariable{#2}\xspace}
\newcommand{\Inout}[2]{\ModificadorArgumento{in/out} \ensuremath{#1}\,: \TipoVariable{#2}\xspace}
\newcommand{\Aplicar}[2]{\NombreFuncion{#1}(#2)}

\newlength{\IntFuncionLengthA}
\newlength{\IntFuncionLengthB}
\newlength{\IntFuncionLengthC}
%InterfazFuncion(nombre, argumentos, valor retorno, precondicion, postcondicion, complejidad, descripcion, aliasing)
\newcommandx{\InterfazFuncion}[9][4=true,6,7,8,9]{%
  \hangindent=\parindent
  \TipoFuncion{#1}{#2}{#3}\\%
  \textbf{Pre} $\equiv$ \{#4\}\\%
  \textbf{Post} $\equiv$ \{#5\}%
  \ifx#6\empty\else\\\textbf{Complejidad:} #6\fi%
  \ifx#7\empty\else\\\textbf{Descripción:} #7\fi%
  \ifx#8\empty\else\\\textbf{Aliasing:} #8\fi%
  \ifx#9\empty\else\\\textbf{Requiere:} #9\fi%
}

\newenvironment{Interfaz}{%
  \parskip=2ex%
  \noindent\textbf{\Large Interfaz}%
  \par%
}{}

\newenvironment{Representacion}{%
  \vspace*{2ex}%
  \noindent\textbf{\Large Representación}%
  \vspace*{2ex}%
}{}

\newenvironment{Algoritmos}{%
  \vspace*{2ex}%
  \noindent\textbf{\Large Algoritmos}%
  \vspace*{2ex}%
}{}


\newcommand{\Titulo}[1]{
  \vspace*{1ex}\par\noindent\textbf{\large #1}\par
}

\newenvironmentx{Estructura}[2][2={estr}]{%
  \par\vspace*{2ex}%
  \TipoVariable{#1} \textbf{se representa con} \TipoVariable{#2}%
  \par\vspace*{1ex}%
}{%
  \par\vspace*{2ex}%
}%

\newboolean{EstructuraHayItems}
\newlength{\lenTupla}
\newenvironmentx{Tupla}[1][1={estr}]{%
    \settowidth{\lenTupla}{\hspace*{3mm}donde \TipoVariable{#1} es \TipoVariable{tupla}$($}%
    \addtolength{\lenTupla}{\parindent}%
    \hspace*{3mm}donde \TipoVariable{#1} es \TipoVariable{tupla}$($%
    \begin{minipage}[t]{\linewidth-\lenTupla}%
    \setboolean{EstructuraHayItems}{false}%
}{%
    $)$%
    \end{minipage}
}

\newcommandx{\tupItem}[3][1={\ }]{%
    %\hspace*{3mm}%
    \ifthenelse{\boolean{EstructuraHayItems}}{%
        ,#1%
    }{}%
    \emph{#2}: \TipoVariable{#3}%
    \setboolean{EstructuraHayItems}{true}%
}

\newcommandx{\RepFc}[3][1={estr},2={e}]{%
  \tadOperacion{Rep}{#1}{bool}{}%
  \tadAxioma{Rep($#2$)}{#3}%
}%

\newcommandx{\Rep}[3][1={estr},2={e}]{%
  \tadOperacion{Rep}{#1}{bool}{}%
  \tadAxioma{Rep($#2$)}{true \ssi #3}%
}%

\newcommandx{\Abs}[5][1={estr},3={e}]{%
  \tadOperacion{Abs}{#1/#3}{#2}{Rep($#3$)}%
  \settominwidth{\hangindent}{Abs($#3$) \igobs #4: #2 $\mid$ }%
  \addtolength{\hangindent}{\parindent}%
  Abs($#3$) \igobs #4: #2 $\mid$ #5%
}%

\newcommandx{\AbsFc}[4][1={estr},3={e}]{%
  \tadOperacion{Abs}{#1/#3}{#2}{Rep($#3$)}%
  \tadAxioma{Abs($#3$)}{#4}%
}%


\newcommand{\DRef}{\ensuremath{\rightarrow}}

\begin{document}

\textbf{M\'odulos b\'asicos}
hostname es string\\
interfaz es nat\\
prioridad es nat\\
id es nat\\
paquete es tupla<id:Id, prioridad : prioridad, origen : compu, destino : compu>\\
compu es tupla<ip : ip, interfaces : conj(interfaz)>\\

\textbf{Aclaraciones:}



\section{Módulo Red}




\begin{Interfaz}
  
	
	\textbf{se explica con}: {\tadNombre{Red}}.

	\textbf{géneros}:{\TipoVariable{red}}.

	\Titulo{Operaciones b\'asicas de Red}
	
	%iniciarRed:
	\InterfazFuncion{IniciarRed}{}{Red}%
	{$res \igobs iniciarRed(r)$}%
	[O(1)]%
	[Crea una red vac\'ia]%
	
	\InterfazFuncion{AgregarComputadora}{\Inout {r}{Red} , \In {c}{compu}}{}
	[$r \igobs r_{0}$]
	{$r \igobs agregarComputadora(r_{0}, c)$}
	[O(L)]
	[Agrega una computadora a la red]
	
	\InterfazFuncion{Conectar}{\Inout {r}{Red} , \In {c1}{compu} , \In {i1}{interfaz}, \In {c2}{compu}, \In {i2}{interfaz}}{}
	[$r \igobs r_{0} \wedge c1 \in computadoras(r_{0}) \wedge c2 \in computadoras(r_{0}) \wedge ip(c1) \neq ip(c2) \wedge \neg conectadas(r_{0}, c1, c2) \wedge \neg usaInterfaz(r_{0}, c1, i1) \wedge \neg usaInterfaz(r_{0}, c2, i2)$]
	{$r \igobs conectar(r_{0}, c1, i1, c2, i2) \wedge conectadas(r_{0}, c1, c2) \wedge usaInterfaz(r_{0}, c1, i1) \wedge usaInterfaz(r_{0}, c2, i2)$}
	[O((L x n + Cardinal(dU.interfaces))x $n^{3}$) donde dU es la computadora de la red con m\'as interfaces.]
	[Modifica la red r conectando las computadoras c1 y c2 a trav\'es de las interfaces i1 e i2 respectivamente.]
	
	\InterfazFuncion{Computadoras}{\In {r}{Red}}{conj(compu)}
	{$res \igobs computadoras(r) $} %POST
	[O(1)]
	[Devuelve un conjunto con todas las computadoras de la red]
	
	\InterfazFuncion{Conectadas?}{\In {r}{Red}, \In {c1}{compu}, \In {c2}{compu}}{bool}
	[$c1 \in computadoras(r) \wedge c2 \in computadoras(r)$] %PRE
	{$res \igobs conectadas(r, c1, c2)$} %POST
	[O($\Sigma_{a' \in Significado(r.Conexiones, c1.hostname)} equal(c2.hostname,a'))$)]
	[Devuelve true si c1 y c2 est\'an conectadas en la red r.]
	
	\InterfazFuncion{InterfazUsada}{\In {r} {Red}, \In {c1}{compu}, \In {c2}{compu}}{interfaz}
	[$conectadas?(r, c1, c2)$] %PRE
	{$res \igobs interfazUsada(r, c1, c2)$} %POST
	[O(L)]
	[Devuelve la interfaz de c1 que se conecta con c2 en la red r.]
	
	\InterfazFuncion{Vecinos}{\In {r} {Red}, \In {c}{compu}}{conj(compu)}
	[$c \in computadoras(r)$] %PRE
	{$res \igobs vecinos(r, c)$} %POST
	[O(L)]
	[Devuelve un conjunto de computadoras con las cuales est\'a conectada la computadora c.]
	
	\InterfazFuncion{UsaInterfaz?}{\In {r} {Red}, \In {c}{compu}, \In {i}{interfaz}}{bool}
	[$c \in computadoras(r)$] %PRE
	{$res \igobs usaInterfaz?(r, c, i)$} %POST
	[O(n x L)]
	[Devuelve true si la interfaz i de la computadora c est\'a siendo utilizada para conectarse con alguna computadora de la red r.]
	
	\InterfazFuncion{CaminosMinimos}{\In {r} {Red}, \In {c1}{compu}, \In {c2}{compu}} {conj(secu(compu)} %Esto hay que corregirlo
	[$c1 \in computadoras(r) \wedge c2 \in computadoras(r)$] %PRE
	{$res \igobs caminosMinimos(r, c1, c2)$} %POST
	[O(L)]
	[Devuelve todos los caminos m\'as cortos para llegar de c1 a c2.]
	
	\InterfazFuncion{HayCamino?}{\In {r} {Red}, \In {c1}{compu}, \In {c2}{compu}}{bool}
	[$c1 \in computadoras(r) \wedge c2 \in computadoras(r)$] %PRE
	{$res \igobs hayCamino?(r, c1, c2)$} %POST
	[O(L)]
	[Devuelve true si hay alg\'un camino que conduzca de c1 a c2.]
	
	
	\textbf{Funciones auxiliares:}

  ConstruirCaminosMinimos(in r : Red, in c1 : compu , in c2 : compu) $\rightarrow$ res : conj(lista(compu)) \\
  -Dado unas computadoras c1 y c2 de una red r, se debe contruir un conj que contenga los caminos minimos para llegar desde la computadora c1 a c2. c1 y c2 deben tener distinto hostname.

  AuxCaminos(in r : Red, in c1 : compu , in c2 : compu, in recorrido : lista(compu), in candidatos : conj(compu)) $\rightarrow$ res : conj(lista(compu)) \\
  -Dado dos computadoras c1 y c2 de una red r, una lista de compu que estan conectadas cada una con la siguiente formando un camino que comienza con c1 y el conjunto de computadoras conectadas con el \'ultimo valor de recorrido en la red r.





\end{Interfaz} 

\begin{Representacion}
 
  \Titulo{Representación de Red}

  \begin{Estructura}{red}[rd]
    \begin{Tupla}[rd]
      \tupItem{Computadoras}{conj(compu)}
      \tupItem{Conexiones}{tried(hostname, conj(compu))} 
      \tupItem{InterfacesQueConectan}{tried(hostname,tried(hostname, interfaz))}%
      \tupItem{caminosMinimos}{tried(hostname,tried(hostname, conj(lista(compu))))} 
    \end{Tupla}
    \end{Estructura}

   

  \Rep[rd][r]{($\forall$ c : compu)(c $\in$ r.Computadoras $\Leftrightarrow$ (Definido?(r.Conexiones, c.hostname) $\wedge$ Definido?(r.InterfacesQueConectan, c.hostname) $\wedge$ Definido?(r.caminosMinimos, c.hostname) $\yluego$ \\
  ($\forall$ ca : compu)(ca $\in$ Significado(r.Conexiones, c.hostname) $\Leftrightarrow$ ca $\in$ r.Computadoras $\wedge$ (Definido?(Significado(r.InterfacesQueConectan, c.hostname), ca.hostname) $\yluego$ Significado(Significado(r.InterfacesQueConectan, c.hostname), ca.hostname) $\in$ c.interfaces) $\wedge$ (Definido?(Significado(r.InterfacesQueConectan, ca.hostname), c.hostname) $\yluego$ Significado(Significado(r.InterfacesQueConectan, ca.hostname), c.hostname) $\in$ ca.interfaces))))) $\wedge$ \\
  ($\forall$ c1, c2 : compu)(c1 $\in$ r.Computadoras $\wedge$ c2 $\in$ r.Computadoras $\Leftrightarrow$ Definido?(Significado(r.caminosMinimos, c1.hostname), c2.hostname) $\wedge$ Definido?(Significado(r.caminosMinimos, c2.hostname), c1.hostname) $\yluego$ Significado(Significado(r.caminosMinimos,c1.hostname), c2.hostname)$=$construirCaminosMinimos(r.Conexiones, c1, c2) )}

  ~
  
  \tadOperacion{construirCaminosMinimos}{diccionario(hostname,conj(compu)),compu,compu)}{conj(secu(compu))}{}
  \tadAxioma{construirCaminosMinimos(cr, c1, c2)}{auxMinimos(caminos(cr, c1, c2))}
  
  ~
  
  \tadOperacion{caminos}{diccionario(hostname,conj(compu)),compu,compu)}{conj(secu(compu))}{}
  \tadAxioma{caminos(cr, c1, c2)}{auxCaminos(cr, c1, c2, c1 $\bullet$ <>, Significado(cr, c1.hostname))}
  
  ~
  
  \tadOperacion{auxCaminos}{diccionario(hostname,conj(compu)),compu,compu,secu(compu),conj(compu)}{conj(secu(compu))}{}
  \tadAxioma{auxCaminos(r, c1, c2, recorrido, candidatos)}{\IF $\phi$?(candidatos) THEN $ag(<>, \phi)$ ELSE {\IF ult(recorrido) = c2 THEN ag(recorrido, $\phi$) ELSE {\IF $\neg$esta?(dameUno(candidatos), recorrido) THEN auxCaminos(r, c1 , c2 , recorrido $\circ$ dameUno(candidatos), Significado(r, dameUno(candidatos).hostname)) $\cup$ auxCaminos(r, c1 , c2 , recorrido, sinUno(candidatos)) ELSE auxCaminos(r, c1 , c2 , recorrido, sinUno(candidatos)) FI} FI} FI}
 
  ~
  
  \tadOperacion{auxMinimos}{conj(secu(compu))}{bool}{}
  \tadAxioma{auxMinimos(cc)}{\IF $\phi$?(cc) THEN $\phi$ ELSE {\IF $\#$(cc) = 1 THEN ag(dameUno(c), $\phi$) ELSE {\IF long(dameUno(cc)) < long(dameUno(auxMinimos(sinUno(cc)))) THEN ag(dameUno(cc), $\phi$) ELSE {\IF long(dameUno(cc)) = long(dameUno(auxMinimos(sinUno(cc)))) THEN ag(dameUno(c), auxMinimos(sinUno(cc))) ELSE auxMinimos(sinUno(cc)) FI} FI} FI}FI}
  
  ~
 
  \Abs[rd]{red}[r]{e}{Computadoras(e) $=$ r.Computadora $\yluego$ ($\forall$ c1, c2 : compu)(conectadas?(e, c1, c2) = (c2 $\in$ Significado(r.Conexiones, c1.hostname) $\wedge$ c1 $\in$ Significado(r.Conexiones, c2.hostname)) $\yluego$ interfazUsada(e, c1, c2) $=$ Significado(Significado(r.InterfacesQueConectan, c1.hostname), c2.hostname))}
O(n x L + Cardinal(c.interfaces))
  
\end{Representacion}	


\begin{Algoritmos}

\begin{algorithm}[H]{\textbf{iIniciarRed}() $\to$ $res$ : $red$} 
	\begin{algorithmic}
	
			\State res $\gets$ <Vac\'io(),Vac\'io(),Vac\'io(),Vac\'io()>  \Comment O(1) 
			\medskip
			\Statex \underline{Complejidad:} O(1)
			\Statex \underline{Justificación:} 
    	\end{algorithmic}
\end{algorithm}

\begin{algorithm}[H]{\textbf{iAgregarCompu}(\Inout {r}{Red}, \In {c}{compu})} 
	\begin{algorithmic}
			\State Agregar(r.Computadoras, c) 		   \Comment $O(\sum_{a' \in r.Computadoras} equal(c,a'))$%costo de agregar c al conjunto lineal r.computadoras
			\State Definir(r.Conexiones, c.hostname, Vac\'io())	\Comment O(long(c.hostname) + copy(c.hostname))	= O(long(c.hostname)) = O(L)
			\State Definir(r.InterfacesQueConectan, c.hostname, Vac\'io())       \Comment O(long(c.hostname) + copy(c.hostname))	= O(long(c.hostname)) = O(L)
			\State Definir(r.caminosMinimos, c.hostname, Vac\'io())       \Comment O(long(c.hostname) + copy(c.hostname))	= O(long(c.hostname)) = O(L)
			\State iteradorCompu : itConj(compu) $\gets$ crearIT(r.Computadoras) 		\Comment O(1)
			
			\While{HaySiguiente(iteradorCompu)} 				\Comment O((n-1) x L) = O(n x L)
			  \If{Siguiente(iteradorCompu).hostname $\neq$ c.hostname} 		\Comment O(3 x L + 4) = O(L)
			    \State vacia1 : lista(compu) $\gets$ Vac\'ia()   			\Comment O(1)
			    \State vacia2 : lista(compu) $\gets$ Vac\'ia()   			\Comment O(1)
			    \State Agregar(vacia1, Vac\'ia()) 		\Comment O(1) ya que no hay elementos en vacia1
			    \State Agregar(vacia1, Vac\'ia()) 		\Comment O(1) ya que no hay elementos en vacia2
			    \State Definir(Significado(r.caminosMinimos, c.hostname) , Siguiente(iteradorCompu).hostname, vacia1) 		\Comment O(2 x L) = O(L)
			    \State Definir(Significado(r.caminosMinimos, Siguiente(iteradorCompu).hostname) , c.hostname, vacia2) 		\Comment O(2 x L) = O(L)
			  \EndIf
			  \State Avanzar(iteradorCompu)   \Comment O(1)
			\EndWhile
			
			
			\medskip
			\Statex \underline{Complejidad:} O(n x L + Cardinal(c.interfaces))
			\Statex \underline{Justificación:} O(n x L + 3 x L + $\sum_{a' \in r.Computadoras} equal(c,a')$ +1) = O(n x L + $\sum_{a' \in r.Computadoras} equal(c,a')$). El costo de ver si dos computadoras son iguales es O(Cardinal(c.interfaces) + L), por las interfaces y el hostname. Nos queda O(n x L + Cardinal(c.interfaces) + L) = O(n x L + Cardinal(c.interfaces))  
    	\end{algorithmic}
\end{algorithm}

\begin{algorithm}[H]{\textbf{iConectar}(\Inout {r}{Red}, \In {c1}{compu}, \In {c2}{compu}, \In {i1}{interfaz}, \In {i2}{interfaz})} 
	\begin{algorithmic}
			\State AgregarRapido(Significado(r.Conexiones, c1.hostname), c2) 		\Comment O(copy(c2)) = O(Cardinal(c2.interfaces) + long(c2.hostname)) = O(Cardinal(c2.interfaces) + L)
			\State AgregarRapido(Significado(r.Conexiones, c2.hostname), c1)		\Comment O(copy(c1)) = O(Cardinal(c1.interfaces) + long(c1.hostname)) = O(Cardinal(c1.interfaces) + L)
			\If{$\neg$Definido?(r.InterfacesQueConectan, c1.hostname)} 	\Comment O(logn(c1.hostname)) = O(L)
			  \State Definir(r.InterfacesQueConectan, c1.hostname, Vac\'io()) \Comment O(logn(c1.hostname)) = O(L)
			\EndIf
			
			\If{$\neg$Definido?(r.InterfacesQueConectan, c2.hostname)} 		\Comment O(logn(c2.hostname)) = O(L)
			  \State Definir(r.InterfacesQueConectan, c2.hostname, Vac\'io()) 	\Comment O(logn(c2.hostname)) = O(L)
			\EndIf
			
			\State Definir(Significado(r.InterfacesQueConectan,c1.hostname), c2.hostname, i2) 	\Comment O(logn(c2.hostname) + logn(c1.hostname)) = O(L + L) = O(L)
			\State Definir(Significado(r.InterfacesQueConectan,c2.hostname), c1.hostname, i1) 	\Comment O(logn(c2.hostname) + logn(c1.hostname)) = O(L + L) = O(L)
			%ALGO PARA Los caminos minimos!
			
			
			\State iteradorCompu1 : itConj(compu) $\gets$ crearIT(r.Computadoras) 		\Comment O(1)
			\While{HayCamino(iteradorCompu1)} \Comment =O((L x n + Cardinal(dU.interfaces))x $n^{3}$ + n) = O((L x n + Cardinal(dU.interfaces))x $n^{3}$)
			  \State iteradorCompu2 : itConj(compu) $\gets$ crearIT(r.Computadoras) 		\Comment O(1)
			  \While{HayCamino(iteradorCompu2)} 	\Comment O(2 x (L x n + Cardinal(dU.interfaces))x $n^{2}$ + n) = O((L x n + Cardinal(dU.interfaces))x $n^{2}$)
			    \If{Siguiente(iteradorCompu1).hostname $\neq$ Siguiente(iteradorCompu2).hostname} \Comment O(L + (L x n + Cardinal(dU.interfaces))x n) = O((L x n + Cardinal(dU.interfaces))x n)
			      \State Definir(Significado(r.caminosMinimos, Siguiente(iteradorCompu1).hostname), Siguiente(iteradorCompu2).hostname, ConstruirCaminosMinimos(r, Siguiente(iteradorCompu1), Siguiente(iteradorCompu2))) \Comment O(2 x L + (L x n + Cardinal(dU.interfaces))x n)=O((L x n + Cardinal(dU.interfaces))x n) donde dU es la computadora entre los candidatos con mayor cantidad de interfaces conectadas.
			      \State Definir(Significado(r.caminosMinimos, Siguiente(iteradorCompu2).hostname), Siguiente(iteradorCompu1).hostname, ConstruirCaminosMinimos(r, Siguiente(iteradorCompu2), Siguiente(iteradorCompu1))) \Comment O(2 x L + (L x n + Cardinal(dU.interfaces))x n)=O((L x n + Cardinal(dU.interfaces))x n) donde dU es la computadora entre los candidatos con mayor cantidad de interfaces conectadas.
			    \EndIf
			  
			    \State Avanzar(iteradorCompu2) 	\Comment O(1)
			  \EndWhile
			  \State Avanzar(iteradorCompu1) 	\Comment O(1)
			\EndWhile
			
			
			\medskip
			\Statex \underline{Complejidad:} O((L x n + Cardinal(dU.interfaces))x $n^{3}$) donde dU es la computadora de la red con m\'as interfaces.
			\Statex \underline{Justificación:} O(Cardinal(c1.interfaces) + Cardinal(c2.interfaces) + 8 x L + 1 + (L x n + Cardinal(dU.interfaces))x $n^{3}$) = O((L x n + Cardinal(dU.interfaces))x $n^{3}$) . Podemos usar AgregarRapido ya que la precondici\'on de conectar nos dice que las computadoras c1 y c2 est\'an en la red y no estan conectadas entre si. Como Cardinal(dU.interfaces) es la cantidad de interfaces de la compu con m\'as interfaces esta le gana a los cardinales de c1.interfaces y c2.interfaces
    	\end{algorithmic}
\end{algorithm}



\begin{algorithm}[H]{\textbf{iComputadoras}(\In {r}{Red}) $\to$ $res$ : $conj(compu)$} 
	\begin{algorithmic}
			\State res $\gets$ r.Computadoras  		   \Comment O(1)
			\medskip
			\Statex \underline{Complejidad:} O(1)
			\Statex \underline{Justificación:} 
    	\end{algorithmic}
\end{algorithm}

\begin{algorithm}[H]{\textbf{iConectadas?}(\In {r}{Red}, \In {c1}{compu}, \In {c2}{compu}) $\to$ $res$ : $bool$} 
	\begin{algorithmic}
			\State res $\gets$ Pertenece?(Significado(r.Conexiones, c1.hostname), c2)   \Comment O($\Sigma_{a' \in Significado(r.Conexiones, c1.hostname)} equal(c2.hostname,a'))$ + long(c1.hostname))
			\medskip
			\Statex \underline{Complejidad:} O($\Sigma_{a' \in Significado(r.Conexiones, c1.hostname)} equal(c2.hostname,a'))$)
			\Statex \underline{Justificación:} O($\Sigma_{a' \in Significado(r.Conexiones, c1.hostname)} equal(c2.hostname,a'))$ + long(c1.hostname)) = O($\Sigma_{a' \in Significado(r.Conexiones, c1.hostname)} equal(c2.hostname,a'))$ + L) = O($\Sigma_{a' \in Significado(r.Conexiones, c1.hostname)} equal(c2.hostname,a'))$). Como comparar las computadoras es hacerlo con sus hostname e interfaces entonces puedo decir que L pierde contra la comparaci\'on de todas las computadoras que puede tener el conjunto. 
    	\end{algorithmic}
\end{algorithm}

\begin{algorithm}[H]{\textbf{iInterfazUsada}(\In {r}{Red}, \In {c1}{compu}, \In {c2}{compu}) $\to$ $res$ : $interfaz$} 
	\begin{algorithmic}
			\State res $\gets$ Significado(Significado(r.InterfacesQueConectan, c1.hostname), c2.hostname) 	\Comment O(long(c1.hostname) + long(c2.hostname))
			

			\medskip
			\Statex \underline{Complejidad:} O(L)
			\Statex \underline{Justificación:} O(long(c1.hostname) + long(c2.hostname)) = O(L + L) = O(L)
    	\end{algorithmic}
\end{algorithm}

\begin{algorithm}[H]{\textbf{iVecinos}(\In {r}{Red}, \In {c}{compu}, \In {i}{interfaz}) $\to$ $res$ : $conj(compu)$} 
	\begin{algorithmic}
			\State res $\gets$ Significado(r.Conexiones, c.hostname) 		   \Comment O(long(c.hostname))
			 
			\medskip
			\Statex \underline{Complejidad:} O(L)
			\Statex \underline{Justificación:} O(long(c.hostname)) = O(L)
    	\end{algorithmic}
\end{algorithm}

\begin{algorithm}[H]{\textbf{iUsaInterfaz?}(\In {r}{Red}, \In {c}{compu}, \In {i}{interfaz}) $\to$ $res$ : $bool$} 
	\begin{algorithmic}
			\State iteradorConexion : itConj(hostname)$\gets$ crearIT(Claves(Significado(r.InterfacesQueConectan, c1.hostname))) 	\Comment O(long(c1.hostname)) = O(L)
			\State encontre : bool $\gets$ false 		\Comment O(1)
			\While{HaySiguiente(iteradorConexion) $\wedge$ $\neg$encontre}  \Comment O(n x L)
			  \If{Significado(Significado(r.InterfacesQueConectan, c1.hostname), Siguiente(iteradorConexion)) $\neq$ i} 	\Comment O(1 + long(c1.hostname) + long(Siguiente(iteradorConexion)) = O(1 + L + L) = O(L)
			    \State encontre $\gets$ true 		\Comment O(1)
			  \EndIf
			  \State Avanzar(iteradorConexion) 					\Comment O(1)
			\EndWhile
			
			 \State res $\gets$ HaySiguiente(iteradorConexion)  			\Comment O(1)			 
			 
			\medskip
			\Statex \underline{Complejidad:} O(n x L)
			\Statex \underline{Justificación:} O(n x L + 1 + L) = O(n x L)
    	\end{algorithmic}
\end{algorithm}

\begin{algorithm}[H]{\textbf{iCaminosMinimos}(\In {r}{Red}, \In {c1}{compu}, \In {c2}{compu}) $\to$ $res$ : $bool$} 
	\begin{algorithmic}
			\State res $\gets$ Significado(Significado(r.caminosMinimos, c1.hostname), c2.hostname) 		\Comment O(long(c1.hostname) + long(c2.hostname))
			 
			\medskip
			\Statex \underline{Complejidad:} O(L)
			\Statex \underline{Justificación:} O(long(c1.hostname) + long(c2.hostname)) = O(L + L) = O(L)
    	\end{algorithmic}
\end{algorithm}

\begin{algorithm}[H]{\textbf{iHayCamino?}(\In {r}{Red}, \In {c1}{compu}, \In {c2}{compu}) $\to$ $res$ : $bool$} 
	\begin{algorithmic}

			\State res $\gets$ $\neg$EsVac\'ia?(Siguiente(crearIT(Significado(Significado(r.caminosMinimos, c1.hostname), c2.hostname), Vac\'ia()))) \Comment O(long(c1.hostname) + long(c2.hostname) + 2)
			\medskip
			\Statex \underline{Complejidad:} O(L)
			\Statex \underline{Justificación:} O(long(c1.hostname) + long(c2.hostname) + 2) = O(L + L) = O(L)
    	\end{algorithmic}
\end{algorithm}

\begin{algorithm}[H]{\textbf{iConstruirCaminosMinimos}(\In {r}{Red}, \In {c1}{compu} , \In {c2}{compu}) $\to$ $res$ : $conj(lista(compu))$} 
	\begin{algorithmic}
	
	\State recorrido $\gets$ AgregarAdelante(Vac\'ia(), c1)  		\Comment O(1)
	\State candidatos $\gets$ Significado(r.Conexiones, c1.hostname)   \Comment O(long(c1.hostname)) = O(L)
	\State camino $\gets$ AuxCaminos(r, c1, c2, recorrido, candidatos) 	\Comment O((L x n + Cardinal(dU.interfaces))x n) donde dU es la computadora entre los candidatos con mayor cantidad de interfaces conectadas.
	\State iterador : itConj(lista(compu)) $\gets$ crearIT(camino) 	\Comment O(1)
	\State minimoActual : lista(compu) $\gets$ Siguiente(iterador) 	\Comment O(1)
	\State Avanzar(iterador) 					\Comment O(1)
	\While{HaySiguiente(iterador)} 					\Comment O(3 x n)
	  \If{Longitud(minimoActual) > Longitud(Siguiente(iterador))} 	\Comment O(2)
	    \State minimoActual $\gets$ Siguiente(iterador) 		\Comment O(1)
	  \EndIf
	  \State Avanzar(iterador) 					\Comment O(1)
	\EndWhile 
	\State iterador2 : itConj(lista(compu)) $\gets$ crearIT(camino) 	\Comment O(1)
	\While{HaySiguiente(iterador2)} 				\Comment O(3 x n)
	  \If{Longitud(minimoActual) < Longitud(Siguiente(iterador2))}  	\Comment O(2)
	    \State EliminarSiguiente(iterador2) 			 \Comment O(1)
	  \Else
	    \State Avanzar(iterador) 					\Comment O(1)
	  \EndIf

	\EndWhile
	\State res $\gets$ camino  	\Comment O(1)
			\medskip
			\Statex \underline{Complejidad:} O((L x n + Cardinal(dU.interfaces))x n) donde dU es la computadora entre los candidatos con mayor cantidad de interfaces conectadas.
			\Statex \underline{Justificación:} O(6 x n + 6 + L + (L x n + Cardinal(dU.interfaces))x n) = O((L x n + Cardinal(dU.interfaces))x n) donde dU es la computadora entre los candidatos con mayor cantidad de interfaces conectadas.
    	\end{algorithmic}
\end{algorithm}
\begin{algorithm}[H]{\textbf{iAuxCaminos}(\In {r}{Red}, \In {c1}{compu} , \In {c2}{compu}, \In {recorrido}{lista(compu)}, \In {candidatos}{conj(compu)}) $\to$ $res$ : $conj(lista(compu))$} 
	\begin{algorithmic}
			\If {EsVac\'io?(candidatos)} 	\Comment O(2)
			  \State res $\gets$ AgregarRapido(Vac\'io(), Vac\'ia())  \Comment O(copy(Vac\'ia())) = O(1)
			\ElsIf {Ultimo(recorrido).hostname = c2.hostname} 	        \Comment O(n x (Cardinal(recorrido[j].interfaces)+L)) donde j es el indice de la compu con mayor cantidad de interfaces conectadas. 
			    \State res $\gets$ AgregarRapido(Vac\'io(), recorrido) 	\Comment O(copy(recorrido))=O(Copiar(recorrido))=O($\sum_{i=1}^{long(recorrido)}copy(recorrido[i])$)=O($\sum_{i=1}^{long(recorrido)}$ \\$(Cardinal(recorrido[i].interfaces) + long(recorrido[i].hostname))$)$=$O($\sum_{i=1}^{long(recorrido)}$ \\$(Cardinal(recorrido[i].interfaces) + L))$)$=$O(n x (Cardinal(recorrido[j].interfaces)+L))donde j es el indice de la compu con mayor cantidad de interfaces conectadas.
			  \Else  						\Comment O(3 + n x L + $Cardinal(dUno.interfaces)+L$ + 2 x (AuxCaminos(r, c1, c2, recorrido, candidatos) - 1) + 1 + n) = O(n x L + $Cardinal(dUno.interfaces)+L$ + (AuxCaminos(r, c1, c2, recorrido, candidatos) - 1))
			      \State iterador : itLista(compu) $\gets$ crearIT(recorrido)   	\Comment O(1)
			      \State dameUno : compu $\gets$ Siguiente(crearIT(candidatos))     \Comment O(1)
			      \State loEncontre : bool $\gets$ false 				\Comment O(1)
			      \While{HaySiguiente(iterador) $\wedge$ $\neg$loEncontre}   \Comment O(L x n)
				\If{dameUno.hostname $=$ Siguiente(iterador).hostname}   \Comment O(MAX(long(Siguiente(iterador).hostname), long(dameUno.hostname)) + 1) = O(L)
				  \State loEncontre $\gets$ true 			\Comment O(1)
				\EndIf
				\State Avanzar(iterador) 				\Comment O(1)
			      \EndWhile
			      
			      \If {$\neg$loEncontre} 			\Comment O($Cardinal(dUno.interfaces)+L$ + 2 x (AuxCaminos(r, c1, c2, recorrido, candidatos) - 1) + 1 + n) donde dU es la computadora entre los candidatos con mayor cantidad de interfaces conectadas.
				  \State camino1 : conj(lista(compu)) $\gets$ AuxCaminos(r, c1 , c2 , AgregarAtras(recorrido, dameUno), Significado(r, dameUno.hostname)) \Comment O(long(dameUno.hostname) + (AuxCaminos(r, c1, c2, recorrido, candidatos) - 1))
				  \State camino2 : conj(lista(compu)) $\gets$ AuxCaminos(r, c1 , c2 , recorrido, Eliminar(candidatos, dameUno)) 			\Comment O($\sum_{a' \in candidatos} equal(dameUno,a')$ + (AuxCaminos(r, c1, c2, recorrido, candidatos) - 1)) = O($Cardinal(dUno.interfaces)+L$ + (AuxCaminos(r, c1, c2, recorrido, candidatos) - 1)) donde dU es la computadora entre los candidatos con mayor cantidad de interfaces conectadas.
				  \State iteradorCamino : itConj(lista(compu)) $\gets$ crearIT(camino2) 	\Comment O(1)
				  \While{HayCamino(iteradorCamino)} 		\Comment O((1 + 1)x n)=O(2 x n) = O(n)
				    \State AgregarRapido(camino1, Siguiente(iteradorCamino))  \Comment O(copy(Siguiente(iteradorCamino))) $\leq$ O(1), pues O(siguiente(iteradorCamino))=O(1).
				    \State Avanzar(iteradorCamino) \Comment O(1)
				  \EndWhile
			      \Else
				\State res $\gets$ AuxCaminos(r, c1 , c2 , recorrido, Eliminar(candidatos, dameUno)) 		\Comment O((AuxCaminos(r, c1, c2, recorrido, candidatos) - 1))
			      \EndIf
			  \EndIf      
			\medskip
			\Statex \underline{Complejidad:} O((L x n + 1+ Cardinal(dU.interfaces)+ L)x n) = O((L x n + Cardinal(dU.interfaces))x n) donde dU es la computadora entre los candidatos con mayor cantidad de interfaces conectadas.%Creo  que es así porque tenemos a lo sumo n llamados recursivos y lo peor que puede pasar es que no encontremos el candidato hasta la última iteración
			\Statex \underline{Justificación:} Como cada vez que llamamos recursivamente a la funcion AuxCaminos estamos continuando con la contrucci\'on del mismo a travez de recorrido, podemos decir que la funci\'on demora a lo sumo n en cada recurci\'on. Esto se debe a que el camino no tiene repetidos.
    	\end{algorithmic}
\end{algorithm}


\end{Algoritmos}
\end{document}

