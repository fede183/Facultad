\documentclass[a4paper,10pt]{article}
\usepackage[paper=a4paper, hmargin=1.5cm, bottom=1.5cm, top=3.5cm]{geometry}
\usepackage[latin1]{inputenc}
\usepackage[T1]{fontenc}
\usepackage[spanish]{babel}
\usepackage{xspace}
\usepackage{xargs}
\usepackage{ifthen}
\usepackage{algorithmicx}
\usepackage{algpseudocode}
\usepackage{algorithm}
\usepackage{aed2-tad,aed2-symb,aed2-itef}


\newcommand{\moduloNombre}[1]{\textbf{#1}}

\let\NombreFuncion=\textsc
\let\TipoVariable=\texttt
\let\ModificadorArgumento=\textbf
\newcommand{\res}{$res$\xspace}
\newcommand{\tab}{\hspace*{7mm}}

\newcommandx{\TipoFuncion}[3]{%
  \NombreFuncion{#1}(#2) \ifx#3\empty\else $\to$ \res\,: \TipoVariable{#3}\fi%
}
\newcommand{\In}[2]{\ModificadorArgumento{in} \ensuremath{#1}\,: \TipoVariable{#2}\xspace}
\newcommand{\Out}[2]{\ModificadorArgumento{out} \ensuremath{#1}\,: \TipoVariable{#2}\xspace}
\newcommand{\Inout}[2]{\ModificadorArgumento{in/out} \ensuremath{#1}\,: \TipoVariable{#2}\xspace}
\newcommand{\Aplicar}[2]{\NombreFuncion{#1}(#2)}

\newlength{\IntFuncionLengthA}
\newlength{\IntFuncionLengthB}
\newlength{\IntFuncionLengthC}
%InterfazFuncion(nombre, argumentos, valor retorno, precondicion, postcondicion, complejidad, descripcion, aliasing)
\newcommandx{\InterfazFuncion}[9][4=true,6,7,8,9]{%
  \hangindent=\parindent
  \TipoFuncion{#1}{#2}{#3}\\%
  \textbf{Pre} $\equiv$ \{#4\}\\%
  \textbf{Post} $\equiv$ \{#5\}%
  \ifx#6\empty\else\\\textbf{Complejidad:} #6\fi%
  \ifx#7\empty\else\\\textbf{Descripci�n:} #7\fi%
  \ifx#8\empty\else\\\textbf{Aliasing:} #8\fi%
  \ifx#9\empty\else\\\textbf{Requiere:} #9\fi%
}

\newenvironment{Interfaz}{%
  \parskip=2ex%
  \noindent\textbf{\Large Interfaz}%
  \par%
}{}

\newenvironment{Representacion}{%
  \vspace*{2ex}%
  \noindent\textbf{\Large Representaci�n}%
  \vspace*{2ex}%
}{}

\newenvironment{Algoritmos}{%
  \vspace*{2ex}%
  \noindent\textbf{\Large Algoritmos}%
  \vspace*{2ex}%
}{}


\newcommand{\Titulo}[1]{
  \vspace*{1ex}\par\noindent\textbf{\large #1}\par
}

\newenvironmentx{Estructura}[2][2={estr}]{%
  \par\vspace*{2ex}%
  \TipoVariable{#1} \textbf{se representa con} \TipoVariable{#2}%
  \par\vspace*{1ex}%
}{%
  \par\vspace*{2ex}%
}%

\newboolean{EstructuraHayItems}
\newlength{\lenTupla}
\newenvironmentx{Tupla}[1][1={estr}]{%
    \settowidth{\lenTupla}{\hspace*{3mm}donde \TipoVariable{#1} es \TipoVariable{tupla}$($}%
    \addtolength{\lenTupla}{\parindent}%
    \hspace*{3mm}donde \TipoVariable{#1} es \TipoVariable{tupla}$($%
    \begin{minipage}[t]{\linewidth-\lenTupla}%
    \setboolean{EstructuraHayItems}{false}%
}{%
    $)$%
    \end{minipage}
}

\newcommandx{\tupItem}[3][1={\ }]{%
    %\hspace*{3mm}%
    \ifthenelse{\boolean{EstructuraHayItems}}{%
        ,#1%
    }{}%
    \emph{#2}: \TipoVariable{#3}%
    \setboolean{EstructuraHayItems}{true}%
}

\newcommandx{\RepFc}[3][1={estr},2={e}]{%
  \tadOperacion{Rep}{#1}{bool}{}%
  \tadAxioma{Rep($#2$)}{#3}%
}%

\newcommandx{\Rep}[3][1={estr},2={e}]{%
  \tadOperacion{Rep}{#1}{bool}{}%
  \tadAxioma{Rep($#2$)}{true \ssi #3}%
}%

\newcommandx{\Abs}[5][1={estr},3={e}]{%
  \tadOperacion{Abs}{#1/#3}{#2}{Rep($#3$)}%
  \settominwidth{\hangindent}{Abs($#3$) \igobs #4: #2 $\mid$ }%
  \addtolength{\hangindent}{\parindent}%
  Abs($#3$) \igobs #4: #2 $\mid$ #5%
}%

\newcommandx{\AbsFc}[4][1={estr},3={e}]{%
  \tadOperacion{Abs}{#1/#3}{#2}{Rep($#3$)}%
  \tadAxioma{Abs($#3$)}{#4}%
}%


\newcommand{\DRef}{\ensuremath{\rightarrow}}

\begin{document}

\section{M�dulo DcNet}


\begin{Interfaz}

  \textbf{se explica con}: {\tadNombre{DCNet}}.

  \textbf{g�neros}:{ \TipoVariable{dcnet}}.

  \Titulo{Operaciones b�sicas de DCNet}
%iniciarDCNet:
  \InterfazFuncion{iniciarDCNet}{\In {r}{red}}{dcnet}%
  {res $\igobs$ iniciarDCNet(r)}%
  [O(n x L)]
  [genera una dcnet con la red r.]
  
%  crearPaquete:
 \InterfazFuncion{crearPaquete}{\Inout {d} {dcnet} , \In {p} {paquete} }{} % {nombreFunci�n}{Signatura}{Res}
  [$(\neg ( (\exists p':paquete) (paqueteEnTransito(s,p') \wedge id(p') = id(p))
\wedge origen(p) \in computadoras(red(s))$
\yluego $destino(p) \in computadoras(red(s))$
\yluego $hayCamino?(red(s), origen(p), destino(p) )) \wedge d\igobs d_{0}$] % "Condici�n del tad" es: (\neg ( (\exists p':paquete) (paqueteEnTransito(s,p') \wedge id(p') = id(p))
%\wedge origen(p) \in computadoras(red(s))$
%\yluego $destino(p) \in computadoras(red(s))$ est� copiado textual del TAD, si hay que cambiarlo avisenme
%\yluego $hayCamino?(red(s), origen(p), destino(p) ))
  {$d \igobs crearPaquete(d_{0}, p)$} %POST
  [O($L + log(k)$)] %COMPLEJIDAD
  [A�ade el paquete p a la dcnet d. Poniendolo en la cola de los paquetes de la computadora inicial.] %DESCRIPCI�N
  [] %ALIASING

%avanzarSegundo:
  \InterfazFuncion{avanzarSegundo}{\Inout {d} {dcnet}}{} % {nombreFunci�n}{Signatura}{Res}
  [$d \igobs d_{0}$] %PRE
  {$d \igobs avanzarSegundo(d_{0})$} %POST
  [O(n x (L + log(n) + log(k)))] %COMPLEJIDAD
  [genera la dcnet correspondiente a pasar un segundo en d.] %DESCRIPCI�N
  [] %ALIASING
  
%red:
  \InterfazFuncion{red}{\In {d}{dcnet}}{red} % {nombreFunci�n}{Signatura}{Res}
  %PRE es true
  {$res \igobs red(d)$} %POST
  [O(1)] %COMPLEJIDAD
  [Devuelve la red asosiada a la dcnet.] %DESCRIPCI�N
  [] %ALIASING
  
%enEspera:
  \InterfazFuncion{enEspera}{\In {d}{dcnet} , \In  {c}{compu}}{colaPriori} % {nombreFunci�n}{Signatura}{Res}
  [$c \in computadoras(red(d))$] %PRE
  {$res \igobs enEspera(d)$} %POST
  [O(L)] %COMPLEJIDAD
  [Devuelve los paquetes de la compu c en d.] %DESCRIPCI�N
  [res es modificables si y s\'olo si d es modificables.] %ALIASING
  
%caminoRecorrido:
  \InterfazFuncion{caminoRecorrido}{\In {d}{dcnet} , \In {p}{paquete}}{secu(compu)} % {nombreFunci�n}{Signatura}{Res}
  [$(\exists c:compu)(c \in computadoras(red(d)) \yluego $est\' a?$(p, enEspera(d, c)) )$] %PRE
  {$res \igobs caminoRecorrido(d, p)$} %POST
  [O(n + log(k))] %COMPLEJIDAD
  [Devuelve el camino recorrido por el paquete p en la dcnet. (Este debe ser el m�s corto)] %DESCRIPCI�N
  [] %ALIASING
  
%cantidadEnviados:
  \InterfazFuncion{cantidadEnviados}{\In {d}{dcnet}, \In {c}{compu}}{nat} % {nombreFunci�n}{Signatura}{Res}
  [$c \in computadoras(red(d))$] %PRE
  {$res \igobs cantidadEnviados(d, c)$} %POST
  [O(L)] %COMPLEJIDAD
  [Devuelve la cantidad de paquetes que envio la compu c.] %DESCRIPCI�N
  [res es modificables si y s\'olo si d es modificables.] %ALIASING
  
%paqueteEnTransito?:
  \InterfazFuncion{paqueteEnTransito?}{\In {d}{dcnet}, \In {p}{paquete}}{bool} % {nombreFunci�n}{Signatura}{Res}
%PRE es true
  {$res \igobs paqueteEnTransito?(d, p)$} %POST
  [O(n x (L + log(k)))] %COMPLEJIDAD
  [Devuelve true si existe alguna computadora que tenga ese paquete.] %DESCRIPCI�N
  [] %ALIASING
  
%laQueMasEnvio:
  \InterfazFuncion{laQueMasEnvio}{\In {d}{dcnet}}{compu} % {nombreFunci�n}{Signatura}{Res}
%PRE es true
  {$res \igobs LaQueMasEnvio(d)$} %POST
  [O(1)] %COMPLEJIDAD
  [Devuelve la computadora que envio m�s paquetes.] %DESCRIPCI�N
  [] %ALIASING


%  \InterfazFuncion{}{}{} % {nombreFunci�n}{Signatura}{Res}
%  [] %PRE
%  {} %POST
%  [] %COMPLEJIDAD
%  [] %DESCRIPCI�N
%  [] %ALIASING

 
\end{Interfaz}

\begin{Representacion}
 
  \Titulo{Representaci�n de DCNet}

  \begin{Estructura}{dcnet}[dc]
    \begin{Tupla}[dc]
      \tupItem{LaQueMasEnvio}{puntero(compu)}%
      \tupItem{enEspera}{tried(hostname, colaPaquetes)}
       \tupItem{\#Enviados}{tried(hostname, nat)}%
       \tupItem{paquetesEnTransitoNoOrigen}{Avld(id, itLista(compu))} 
        \tupItem{Red}{Red}%
    \end{Tupla}
    \end{Estructura}

   

  \Rep[dc][d]{($\forall$ c : compu)(c $\in$ Computadoras(d.Red) $\Leftrightarrow$ Definido?(d.enEspera, c.hostname) $\wedge$ Definido?(d.$\#$Enviados, c.hostname)) $\wedge$ *d.LaQueMasEnvio $\in$ Computadoras(d.Red) $\yluego$ ($\neg$(Existe ca : compu)(ca $\in$ Computadoras(d.Red) $\yluego$ Significado(d.$\#$Enviados,ca.hostname) > Significado(d.$\#$Enviados, *d.LaQueMasEnvio.hostname))) $\wedge$ ($\forall$ h : Computadoras(d.Red))($\forall$ p : paquete)((p $\in$ pasarColaAConjunto(Significado(d.enEspera, h.hostname)) $\wedge$ h.hostname $\neq$ p.origen.hostname) $\Leftrightarrow$ (Definido?AVL(d.paquetesEnTransitoNoOrigen,p.id) $\yluego$ h $=$ Siguiente(SignificadoAVL(d.paquetesEnTransitoNoOrigen, p.id))))}
  ~
  \tadOperacion{pasarColaAConjunto}{colaPrior(paquete)}{conj(paquete)}{}
  \tadAxioma{pasarColaAConjunto(c)}{\IF vac\'ia?(c) THEN $\phi$ ELSE ag(Pr\'oximo(c) , pasarColaAConjunto(Desencolar(c))) FI}
  ~

  \Abs[dc]{dcnet}[d]{e}{d.Red $=$ red(e) $\wedge$ ($\forall$ c : compu)((c $\in$ Computadoras(red(e)) $\impluego$ (cantidadEnviados(e, c) $=$ Significado(d.$\#$Enviados, c.hostname) $\wedge$ enEspera(e, c) $=$ Significado(d.enEspera, c.hostname))) $\wedge$ ($\forall$ p : paquete)(p $\in$ pasarColaAConjunto(Significado(d.enEspera, c.hostname)) $\impluego$ \\($\neg$Definido?AVL(d.paquetesEnTransitoNoOrigen, p.id) $\Leftrightarrow$ caminoRecorrido(e,p) $=$ <>  $\wedge$ \\ Definido?AVL(d.paquetesEnTransitoNoOrigen, p.id) $\impluego$  caminoRecorrido(e,p) $=$ p.origen $\bullet$ \\ ContruirCamino(SignificadoAVL(d.paquetesEnTransitoNoOrigen, p.id)))))}
  
    ~
    
  \tadOperacion{ConstruirCamino}{itLista(compu))}{secu(compu)}{}
  \tadAxioma{ConstruirCamino(c)}{\IF HayAnterior(c) THEN ConstruirCamino(Retroceder(c)) $\circ$ Anterior(c) ELSE <> FI}
  
\end{Representacion}% PARA ITERADORES SI USARAMOS	

\begin{Algoritmos}

%iIniciarDCNet:
\begin{algorithm}[H]{\textbf{iIniciarDCNet}(\In{r}{red}) $\to$ $res$ : $dc$} 
	\begin{algorithmic}
			\State enviados : tried(hostname, nat) $\gets$  Vac\'io() 		   	\Comment O(1)    
			\State iteradorCompu : itConj(compu) $\gets$ crearIT(r.Computadoras)	\Comment O(1)
			\State enespera : tried(hostname, colaPaquetes) $\gets$ Vac\'io() 			\Comment O(1)
			\State *laQueMasEnvio : puntero(compu) $\gets$ NULL 			\Comment O(1)
			\If{HaySiguiente(iteradorCompu)} 		\Comment O(2)
			   \State laQueMasEnvio $\gets$ Siguiente(iteradorCompu) 	\Comment O(1)
			\EndIf
			
			\While{HaySiguiente(iteradorCompu)} \Comment O(n x ((2 x L) + 1))  = O(n x L)
				\State Definir(enviados, Siguiente(iteradorCompu).hostname, 0) 		\Comment O(long(Siguiente(iteradorCompu).hostname)) = O(L)
				\State Definir(enespera, Siguiente(iteradorCompu).hostname, Vac\'io()) 	\Comment O(long(Siguiente(iteradorCompu).hostname)) = O(L)
				\State Avanzar(iteradorCompu) 						 \Comment O(1)
			\EndWhile
			 
			 \State res $\gets$ <laQueMasEnvio, enespera, enviados, Vac\'io(), r> 		\Comment O(1)

			\medskip
			\Statex \underline{Complejidad:} O(n x L)
			\Statex \underline{Justificaci�n:} O(n x L + 8) = O(n x L)
    	\end{algorithmic}
\end{algorithm}

%icrearPaquete:
\begin{algorithm}[H]{\textbf{iCrearPaquete}(\Inout{d}{dc}, \In {p}{paquete})} 
	\begin{algorithmic}
			\State Encolar(p, Significado(d.enEspera, p.origen.hostname))\Comment O(long(p.origen.hostname) + log(k))   

			\medskip
			\Statex \underline{Complejidad:} O(L + log(k))
			\Statex \underline{Justificaci�n:} O(long(p.origen.hostname) + log(k)) = O(L + log(k))
    	\end{algorithmic}
\end{algorithm}

%iavanzarSegundo:





\begin{algorithm}[H]{\textbf{iavanzarSegundo}(\Inout{d}{dc})} 
	\begin{algorithmic}
			\State ListaAEnviar : lista(tupla<hostname, paquete>) $\gets$ Vac\'io() 					\Comment O(1)
			\State iteradorCompu : itConj(compu) $\gets$ crearIT(Computadoras(d.red)) 					\Comment O(1)
			\While {HaySiguiente(iteradorCompu)}				\Comment O(n x (L + log(n) + log(k)))
			
			  \If{$\neg$Vac\'ia?(Significado(d.enEspera, iteradorCompu.hostname))} 			\Comment O(2 x L + log(n) + log(k)) = O(L + log(n) + log(k))
				\State Significado(d.\#Enviados, Siguiente(iteradorCompu).hostname) + 1    	\Comment O(L)
				\If{Significado(d.\#Enviados, Siguiente(iteradorCompu).hostname) > Significado(d.$\#$Enviados, *d.LaQueMasEnvio.hostname)}  	\Comment O(2 x L) = O(L)
				  \State d.LaQueMasEnvio $\gets$ $\delta$Siguiente(iteradorCompu) 			\Comment O(1)
				\EndIf
				\State aEnviar : paquete $\gets$ Pr\'oximo(Significado(d.enEspera, Siguiente(iteradorCompu).hostname))   \Comment O(L + log(k))
				\State Desencolar(Significado(d.enEspera, Siguiente(iteradorCompu).hostname)) 		\Comment O(L + log(k))
				
				\If{Definida?AVL(d.paquetesEnTransitoNoOrigen, p.id)} 			\Comment O(2x(log(n) + log(k))) = O(log(n) + log(k))
				    \State cam : itLista(compu) $\gets$ SignificadoAVL(d.paquetesEnTransitoNoOrigen, p.id) 	\Comment O(log(n)+ log(k))
				    \If{HaySiguiente(cam)}  					\Comment O(3)
				      \State AgregarAtras(ListaAEnviar, <Siguiente(cam).hostname, aEnviar>) 		\Comment O(copy(<Siguiente(cam).hostname, aEnviar>)) = O(1)
				      \State Avanzar(cam) 					\Comment O(1)
				    \Else 								\Comment O(log(n) + log(k))
				      \State BorrarAVL(d.paquetesEnTransitoNoOrigen, p.id) 		\Comment O(log(n) + log(k))
				    \EndIf
				\Else 										\Comment  O(L + 2 x log(k) + 2 x log(n)) = O(L + log(n) + log(k))
				  \State camino : itLista(compu) $\gets$ crearIT(Siguiente(crearIT(caminosMinimos(d.red, p.origen, p.destino))))  \Comment O(L)
				  \State Avanzar(camino) 			\Comment O(1)
				  \If{HaySiguiente(camino)} 						\Comment O(2 + log(n) + 2 x log(k)) = O(log(n) + log(k))
				      \State AgregarAtras(ListaAEnviar, <Siguiente(camino).hostname, aEnviar>) 	\Comment O(copy(<Siguiente(camino).hostname, aEnviar>)) = O(1)
				      \State DefinirAVL(d.paquetesEnTransitoNoOrigen, p.id, camino) 	\Comment O(log(n) + log(k))
				    \EndIf 
				\EndIf
				
			  \EndIf
			  \State Avanzar(iteradorCompu) 				\Comment O(1)
			\EndWhile

			\State iteradorEnviar : itLista(tupla<hostname, paquete>) $\gets$ crearIT(ListaAEnviar) 		\Comment O(1)
			
			\While{HaySiguiente(iteradorEnviar)} 				\Comment O(n x (L + log(k)))
			  \State Encolar(Significado(d.enEspera, PI1(Siguiente(iteradorEnviar)) , PI2(Siguiente(iteradorEnviar)))) 	\Comment O(L + log(k))
			  \State Avanzar(iteradorEnviar) 										\Comment O(1)
			\EndWhile
			
			\medskip
			\Statex \underline{Complejidad:} O(n x (L + log(n) + log(k)))
			\Statex \underline{Justificaci�n:} O(n x (L + log(k)) + n x (L + log(n) + log(k)) + 4) = O(n x (L + log(n) + log(k))). Para facilitar la escritura de las complejidades y embellecer el seudoc\'odigo. Se decidi\'o ahorrarnos un paso en el calculo de complejidades y lo que vale O(long(hostname)) o similar se lo escribe directamente como O(L).
    	\end{algorithmic}
\end{algorithm}

%ired: 
\begin{algorithm}[H]{\textbf{iRed}(\In{d}{dc}) $\to$ $res$ : $red$} 
	\begin{algorithmic}
			\State res $\gets d.Red$ \Comment O(1)
			\medskip
			\Statex \underline{Complejidad:} O(1)
    	\end{algorithmic}
\end{algorithm}

%ienEspera:
\begin{algorithm}[H]{\textbf{ienEspera}(\In{d}{dc}, \In {c}{compu}) $\to$ $res$ : $colaPaquetes$} 
	\begin{algorithmic}
			\State res $\gets$ Significado(c.hostname, d.enEspera) \Comment O(long(c.hostname)) 
			\medskip
			\Statex \underline{Complejidad:} O(L)
			\Statex \underline{Justificaci�n:} O(long(c.hostname)) = O(L)
    	\end{algorithmic}
\end{algorithm}

%icaminoRecorrido(in d:dc, in p:paquete) -> res:  //
\begin{algorithm}[H]{\textbf{icaminoRecorrido}(\In{d}{dc}, \In {p}{paquete}) $\to$ $res$ : $lista(compu)$} 
	\begin{algorithmic}
			\State res $\gets$ Vac\'ia()						\Comment O(1)
			\If{$\neg$Definido?AVL(d.paquetesEnTransitoNoOrigen, p.id)} 			\Comment O(2 x n + 2 x log(n) + 2 x log(k)) = O(n + log(n) + log(k)) = O(n + log(k))
			    \State cam : itLista(compu) $\gets$ SignificadoAVL(d.paquetesEnTransitoNoOrigen, p.id)	\Comment O(log(n) + log(k))
			    \State i : nat $\gets$ 0 									\Comment O(1)
			    
			    \While{HayAnterior(cam)} 					\Comment O(4 x n) = O(n)	    
			      \State AgregarAdelante(res, Anterior(cam)) 	\Comment O(copy(Anterior(cam))) = O(1)
			      \State Retroceder(cam) 				\Comment O(1)
			      \State i $\gets$ i + 1 				\Comment O(1)
			    \EndWhile
			    
			     \While{i $\neq$ 0} 					\Comment O(3 x n) = O(n)
			      \State Avanzar(cam) 				\Comment O(1)
			      \State i $\gets$ i - 1 				\Comment O(1)
			    \EndWhile
			    
			\EndIf
			\medskip
			\Statex \underline{Complejidad:} O(n + log(k))
			\Statex \underline{Justificaci�n:} Qvq O(n + log(k)) $\subseteq$ O(n x log(n + k)) = O(MAX(n,log(k))) $\subseteq$ O(n x log(MAX(n,k))). Si n<=k entonces O(MAX(n, log(k))) $\subseteq$ O(n x log(k)). Si n<=log(k) entonces O(log(k)) $\subseteq$ O(n x log(k)) esto es correcto, para el caso n>log(k) O(n) $\subseteq$ O(n x log(k)) esto es correcto. Para el caso n>k tenemos que O(n) $\subseteq$ O(n x log(n)) y esto tambi\'en es correcto. Con esto vemos que en todos los casos es correcto.
    	\end{algorithmic}
\end{algorithm}

%icantidadEnviados:
\begin{algorithm}[H]{\textbf{icantidadEnviados}(\In{d}{dc}, \In {c}{compu}) $\to$ $res$ : $nat$} 
	\begin{algorithmic}
			\State res $\gets$ Significado(c.hostname, d.\# Enviados)     \Comment O(long(c.hostname))
			\medskip
			\Statex \underline{Complejidad:} O(L)
			\Statex \underline{Justificaci�n:} O(long(c.hostname)) = O(L)
    	\end{algorithmic}
\end{algorithm}

%iPaqueteEnTransito?:
\begin{algorithm}[H]{\textbf{iPaqueteEnTransito?}(\In{d}{dc}, \In {p}{paquete}) $\to$ $res$ : $bool$} 
	\begin{algorithmic}
			\State res $\gets$ false	\Comment O(1)
			\State iteradorCompu : itConj(compu) $\gets$ crearIT(Computadoras(d.red))  \Comment O(1)
			\While {HaySiguiente(iteradorCompu) $\wedge$ $\neg$ res} 		\Comment O(n x (L + log(k) + 1)) $=$ O(n x (L + log(k)))
				\State res $\gets$ Pertenece(p, Significado(d.enEspera, hostname(Siguiente(c)))))		\Comment O(long(hostname(Siguiente(c))) + log(k)) $=$ O(L + log(k))	
				\State Avanzar(iteradorCompu) 					\Comment O(1)  
			\EndWhile
			\medskip
			\Statex \underline{Complejidad:} O(n x (L + log(k)))
			\Statex \underline{Justificaci�n:} O(n x (L + log(k)) + 2) = O(n x (L + log(k)))
    	\end{algorithmic}
\end{algorithm}

%iLaQueMasEnvio:

\begin{algorithm}[H]{\textbf{iLaQueMasEnvio}(\In{d}{dc}) $\to$ $res$ : $compu$} 
	\begin{algorithmic}
			\State res $\gets$ *d.LaQueMasEnvio \Comment O(1)
			\medskip
			\Statex \underline{Complejidad:} O(1) 
    	\end{algorithmic}
\end{algorithm}




\end{Algoritmos}

\end{document}
