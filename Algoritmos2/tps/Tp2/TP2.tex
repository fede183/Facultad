\documentclass[a4paper,10pt]{article}
\usepackage[paper=a4paper, hmargin=1.5cm, bottom=1.5cm, top=3.5cm]{geometry}
\usepackage[latin1]{inputenc}
\usepackage[T1]{fontenc}
\usepackage[spanish]{babel}
%\usepackage{xspace}
%\usepackage{xargs}
%\usepackage{ifthen}
%\usepackage{algorithmicx}
%\usepackage{algpseudocode}
\usepackage{algorithm}
\usepackage{aed2-symb,aed2-itef,aed2-tad,caratula}
% Otros paquetes del ejemplo de modulos basicos
\usepackage{fancyhdr,lastpage,xspace,xargs,ifthen}
\usepackage{algorithmicx, algpseudocode, algorithm}
% Por si queremos links de colores en el índice
\usepackage[colorlinks=true, linkcolor=blue]{hyperref}
%%%%%%%%%%%%%%%%%%%%%%%%%%%%%%%%%%%%%%%%%%%%%%%%%%%%%%%%%
%NO SÉ SI ESTO HACE FALTA

\newcommand{\moduloNombre}[1]{\textbf{#1}}

\let\NombreFuncion=\textsc
\let\TipoVariable=\texttt
\let\ModificadorArgumento=\textbf
\newcommand{\res}{$res$\xspace}
\newcommand{\tab}{\hspace*{7mm}}

\newcommandx{\TipoFuncion}[3]{%
  \NombreFuncion{#1}(#2) \ifx#3\empty\else $\to$ \res\,: \TipoVariable{#3}\fi%
}
\newcommand{\In}[2]{\ModificadorArgumento{in} \ensuremath{#1}\,: \TipoVariable{#2}\xspace}
\newcommand{\Out}[2]{\ModificadorArgumento{out} \ensuremath{#1}\,: \TipoVariable{#2}\xspace}
\newcommand{\Inout}[2]{\ModificadorArgumento{in/out} \ensuremath{#1}\,: \TipoVariable{#2}\xspace}
\newcommand{\Aplicar}[2]{\NombreFuncion{#1}(#2)}

\newlength{\IntFuncionLengthA}
\newlength{\IntFuncionLengthB}
\newlength{\IntFuncionLengthC}
%InterfazFuncion(nombre, argumentos, valor retorno, precondicion, postcondicion, complejidad, descripcion, aliasing)
\newcommandx{\InterfazFuncion}[9][4=true,6,7,8,9]{%
  \hangindent=\parindent
  \TipoFuncion{#1}{#2}{#3}\\%
  \textbf{Pre} $\equiv$ \{#4\}\\%
  \textbf{Post} $\equiv$ \{#5\}%
  \ifx#6\empty\else\\\textbf{Complejidad:} #6\fi%
  \ifx#7\empty\else\\\textbf{Descripción:} #7\fi%
  \ifx#8\empty\else\\\textbf{Aliasing:} #8\fi%
  \ifx#9\empty\else\\\textbf{Requiere:} #9\fi%
}

\newenvironment{Interfaz}{%
  \parskip=2ex%
  \noindent\textbf{\Large Interfaz}%
  \par%
}{}

\newenvironment{Representacion}{%
  \vspace*{2ex}%
  \noindent\textbf{\Large Representación}%
  \vspace*{2ex}%
}{}

\newenvironment{Algoritmos}{%
  \vspace*{2ex}%
  \noindent\textbf{\Large Algoritmos}%
  \vspace*{2ex}%
}{}


\newcommand{\Titulo}[1]{
  \vspace*{1ex}\par\noindent\textbf{\large #1}\par
}

\newenvironmentx{Estructura}[2][2={estr}]{%
  \par\vspace*{2ex}%
  \TipoVariable{#1} \textbf{se representa con} \TipoVariable{#2}%
  \par\vspace*{1ex}%
}{%
  \par\vspace*{2ex}%
}%

\newboolean{EstructuraHayItems}
\newlength{\lenTupla}
\newenvironmentx{Tupla}[1][1={estr}]{%
    \settowidth{\lenTupla}{\hspace*{3mm}donde \TipoVariable{#1} es \TipoVariable{tupla}$($}%
    \addtolength{\lenTupla}{\parindent}%
    \hspace*{3mm}donde \TipoVariable{#1} es \TipoVariable{tupla}$($%
    \begin{minipage}[t]{\linewidth-\lenTupla}%
    \setboolean{EstructuraHayItems}{false}%
}{%
    $)$%
    \end{minipage}
}

\newcommandx{\tupItem}[3][1={\ }]{%
    %\hspace*{3mm}%
    \ifthenelse{\boolean{EstructuraHayItems}}{%
        ,#1%
    }{}%
    \emph{#2}: \TipoVariable{#3}%
    \setboolean{EstructuraHayItems}{true}%
}

\newcommandx{\RepFc}[3][1={estr},2={e}]{%
  \tadOperacion{Rep}{#1}{bool}{}%
  \tadAxioma{Rep($#2$)}{#3}%
}%

\newcommandx{\Rep}[3][1={estr},2={e}]{%
  \tadOperacion{Rep}{#1}{bool}{}%
  \tadAxioma{Rep($#2$)}{true \ssi #3}%
}%

\newcommandx{\Abs}[5][1={estr},3={e}]{%
  \tadOperacion{Abs}{#1/#3}{#2}{Rep($#3$)}%
  \settominwidth{\hangindent}{Abs($#3$) \igobs #4: #2 $\mid$ }%
  \addtolength{\hangindent}{\parindent}%
  Abs($#3$) \igobs #4: #2 $\mid$ #5%
}%

\newcommandx{\AbsFc}[4][1={estr},3={e}]{%
  \tadOperacion{Abs}{#1/#3}{#2}{Rep($#3$)}%
  \tadAxioma{Abs($#3$)}{#4}%
}%


\newcommand{\DRef}{\ensuremath{\rightarrow}}

%%%%%%%%%%%%%%%%%%%%%%%%%%%%%%%%%%%%%%%%%%%%%%%%%%%%%%%%%%%%%%%%%%%%%%%%%%%%%%%%%%%%%%%%%%%%%%%%%%%%

\begin{document}

% Estos comandos deben ir antes del \maketitle
\materia{Algoritmos y Estructuras de Datos II} % obligatorio
\submateria{Primer Cuatrimestre de 2015} % opcional
\titulo{Trabajo Practico 2} % obligatorio
\subtitulo{DCNet} % opcional
\grupo{Grupo 15} % opcional 

\integrante{De Rocco Federico Lucas}{403/13}{fede.183@hotmail.com} % obligatorio 
\integrante{Pardo Florencia}{626/08}{florenciapardo@yahoo.com.ar} % obligatorio 


\maketitle

% compilar 2 veces para actualizar las referencias
\tableofcontents

\pagebreak

\documentclass[a4paper,10pt]{article}
\usepackage[paper=a4paper, hmargin=1.5cm, bottom=1.5cm, top=3.5cm]{geometry}
\usepackage[latin1]{inputenc}
\usepackage[T1]{fontenc}
\usepackage[spanish]{babel}
\usepackage{xspace}
\usepackage{xargs}
\usepackage{ifthen}
\usepackage{algorithmicx}
\usepackage{algpseudocode}
\usepackage{algorithm}
\usepackage{aed2-tad,aed2-symb,aed2-itef}
\usepackage[T1]{fontenc}
\usepackage{selinput}
\SelectInputMappings{%
  aacute={á},
  ntilde={ñ},
  Euro={€}
}
\usepackage{babel}

\newcommand{\moduloNombre}[1]{\textbf{#1}}

\let\NombreFuncion=\textsc
\let\TipoVariable=\texttt
\let\ModificadorArgumento=\textbf
\newcommand{\res}{$res$\xspace}
\newcommand{\tab}{\hspace*{7mm}}

\newcommandx{\TipoFuncion}[3]{%
  \NombreFuncion{#1}(#2) \ifx#3\empty\else $\to$ \res\,: \TipoVariable{#3}\fi%
}
\newcommand{\In}[2]{\ModificadorArgumento{in} \ensuremath{#1}\,: \TipoVariable{#2}\xspace}
\newcommand{\Out}[2]{\ModificadorArgumento{out} \ensuremath{#1}\,: \TipoVariable{#2}\xspace}
\newcommand{\Inout}[2]{\ModificadorArgumento{in/out} \ensuremath{#1}\,: \TipoVariable{#2}\xspace}
\newcommand{\Aplicar}[2]{\NombreFuncion{#1}(#2)}

\newlength{\IntFuncionLengthA}
\newlength{\IntFuncionLengthB}
\newlength{\IntFuncionLengthC}
%InterfazFuncion(nombre, argumentos, valor retorno, precondicion, postcondicion, complejidad, descripcion, aliasing)
\newcommandx{\InterfazFuncion}[9][4=true,6,7,8,9]{%
  \hangindent=\parindent
  \TipoFuncion{#1}{#2}{#3}\\%
  \textbf{Pre} $\equiv$ \{#4\}\\%
  \textbf{Post} $\equiv$ \{#5\}%
  \ifx#6\empty\else\\\textbf{Complejidad:} #6\fi%
  \ifx#7\empty\else\\\textbf{Descripción:} #7\fi%
  \ifx#8\empty\else\\\textbf{Aliasing:} #8\fi%
  \ifx#9\empty\else\\\textbf{Requiere:} #9\fi%
}

\newenvironment{Interfaz}{%
  \parskip=2ex%
  \noindent\textbf{\Large Interfaz}%
  \par%
}{}

\newenvironment{Representacion}{%
  \vspace*{2ex}%
  \noindent\textbf{\Large Representación}%
  \vspace*{2ex}%
}{}

\newenvironment{Algoritmos}{%
  \vspace*{2ex}%
  \noindent\textbf{\Large Algoritmos}%
  \vspace*{2ex}%
}{}


\newcommand{\Titulo}[1]{
  \vspace*{1ex}\par\noindent\textbf{\large #1}\par
}

\newenvironmentx{Estructura}[2][2={estr}]{%
  \par\vspace*{2ex}%
  \TipoVariable{#1} \textbf{se representa con} \TipoVariable{#2}%
  \par\vspace*{1ex}%
}{%
  \par\vspace*{2ex}%
}%

\newboolean{EstructuraHayItems}
\newlength{\lenTupla}
\newenvironmentx{Tupla}[1][1={estr}]{%
    \settowidth{\lenTupla}{\hspace*{3mm}donde \TipoVariable{#1} es \TipoVariable{tupla}$($}%
    \addtolength{\lenTupla}{\parindent}%
    \hspace*{3mm}donde \TipoVariable{#1} es \TipoVariable{tupla}$($%
    \begin{minipage}[t]{\linewidth-\lenTupla}%
    \setboolean{EstructuraHayItems}{false}%
}{%
    $)$%
    \end{minipage}
}

\newcommandx{\tupItem}[3][1={\ }]{%
    %\hspace*{3mm}%
    \ifthenelse{\boolean{EstructuraHayItems}}{%
        ,#1%
    }{}%
    \emph{#2}: \TipoVariable{#3}%
    \setboolean{EstructuraHayItems}{true}%
}

\newcommandx{\RepFc}[3][1={estr},2={e}]{%
  \tadOperacion{Rep}{#1}{bool}{}%
  \tadAxioma{Rep($#2$)}{#3}%
}%

\newcommandx{\Rep}[3][1={estr},2={e}]{%
  \tadOperacion{Rep}{#1}{bool}{}%
  \tadAxioma{Rep($#2$)}{true \ssi #3}%
}%

\newcommandx{\Abs}[5][1={estr},3={e}]{%
  \tadOperacion{Abs}{#1/#3}{#2}{Rep($#3$)}%
  \settominwidth{\hangindent}{Abs($#3$) \igobs #4: #2 $\mid$ }%
  \addtolength{\hangindent}{\parindent}%
  Abs($#3$) \igobs #4: #2 $\mid$ #5%
}%

\newcommandx{\AbsFc}[4][1={estr},3={e}]{%
  \tadOperacion{Abs}{#1/#3}{#2}{Rep($#3$)}%
  \tadAxioma{Abs($#3$)}{#4}%
}%


\newcommand{\DRef}{\ensuremath{\rightarrow}}

\begin{document}

\textbf{M\'odulos b\'asicos}
hostname es string\\
interfaz es nat\\
prioridad es nat\\
id es nat\\
paquete es tupla<id:Id, prioridad : prioridad, origen : compu, destino : compu>\\
compu es tupla<ip : ip, interfaces : conj(interfaz)>\\

\textbf{Aclaraciones:}



\section{Módulo Red}




\begin{Interfaz}
  
	
	\textbf{se explica con}: {\tadNombre{Red}}.

	\textbf{géneros}:{\TipoVariable{red}}.

	\Titulo{Operaciones b\'asicas de Red}
	
	%iniciarRed:
	\InterfazFuncion{IniciarRed}{}{Red}%
	{$res \igobs iniciarRed(r)$}%
	[O(1)]%
	[Crea una red vac\'ia]%
	
	\InterfazFuncion{AgregarComputadora}{\Inout {r}{Red} , \In {c}{compu}}{}
	[$r \igobs r_{0}$]
	{$r \igobs agregarComputadora(r_{0}, c)$}
	[O(L)]
	[Agrega una computadora a la red]
	
	\InterfazFuncion{Conectar}{\Inout {r}{Red} , \In {c1}{compu} , \In {i1}{interfaz}, \In {c2}{compu}, \In {i2}{interfaz}}{}
	[$r \igobs r_{0} \wedge c1 \in computadoras(r_{0}) \wedge c2 \in computadoras(r_{0}) \wedge ip(c1) \neq ip(c2) \wedge \neg conectadas(r_{0}, c1, c2) \wedge \neg usaInterfaz(r_{0}, c1, i1) \wedge \neg usaInterfaz(r_{0}, c2, i2)$]
	{$r \igobs conectar(r_{0}, c1, i1, c2, i2) \wedge conectadas(r_{0}, c1, c2) \wedge usaInterfaz(r_{0}, c1, i1) \wedge usaInterfaz(r_{0}, c2, i2)$}
	[O((L x n + Cardinal(dU.interfaces))x $n^{3}$) donde dU es la computadora de la red con m\'as interfaces.]
	[Modifica la red r conectando las computadoras c1 y c2 a trav\'es de las interfaces i1 e i2 respectivamente.]
	
	\InterfazFuncion{Computadoras}{\In {r}{Red}}{conj(compu)}
	{$res \igobs computadoras(r) $} %POST
	[O(1)]
	[Devuelve un conjunto con todas las computadoras de la red]
	
	\InterfazFuncion{Conectadas?}{\In {r}{Red}, \In {c1}{compu}, \In {c2}{compu}}{bool}
	[$c1 \in computadoras(r) \wedge c2 \in computadoras(r)$] %PRE
	{$res \igobs conectadas(r, c1, c2)$} %POST
	[O($\Sigma_{a' \in Significado(r.Conexiones, c1.hostname)} equal(c2.hostname,a'))$)]
	[Devuelve true si c1 y c2 est\'an conectadas en la red r.]
	
	\InterfazFuncion{InterfazUsada}{\In {r} {Red}, \In {c1}{compu}, \In {c2}{compu}}{interfaz}
	[$conectadas?(r, c1, c2)$] %PRE
	{$res \igobs interfazUsada(r, c1, c2)$} %POST
	[O(L)]
	[Devuelve la interfaz de c1 que se conecta con c2 en la red r.]
	
	\InterfazFuncion{Vecinos}{\In {r} {Red}, \In {c}{compu}}{conj(compu)}
	[$c \in computadoras(r)$] %PRE
	{$res \igobs vecinos(r, c)$} %POST
	[O(L)]
	[Devuelve un conjunto de computadoras con las cuales est\'a conectada la computadora c.]
	
	\InterfazFuncion{UsaInterfaz?}{\In {r} {Red}, \In {c}{compu}, \In {i}{interfaz}}{bool}
	[$c \in computadoras(r)$] %PRE
	{$res \igobs usaInterfaz?(r, c, i)$} %POST
	[O(n x L)]
	[Devuelve true si la interfaz i de la computadora c est\'a siendo utilizada para conectarse con alguna computadora de la red r.]
	
	\InterfazFuncion{CaminosMinimos}{\In {r} {Red}, \In {c1}{compu}, \In {c2}{compu}} {conj(secu(compu)} %Esto hay que corregirlo
	[$c1 \in computadoras(r) \wedge c2 \in computadoras(r)$] %PRE
	{$res \igobs caminosMinimos(r, c1, c2)$} %POST
	[O(L)]
	[Devuelve todos los caminos m\'as cortos para llegar de c1 a c2.]
	
	\InterfazFuncion{HayCamino?}{\In {r} {Red}, \In {c1}{compu}, \In {c2}{compu}}{bool}
	[$c1 \in computadoras(r) \wedge c2 \in computadoras(r)$] %PRE
	{$res \igobs hayCamino?(r, c1, c2)$} %POST
	[O(L)]
	[Devuelve true si hay alg\'un camino que conduzca de c1 a c2.]
	
	
	\textbf{Funciones auxiliares:}

  ConstruirCaminosMinimos(in r : Red, in c1 : compu , in c2 : compu) $\rightarrow$ res : conj(lista(compu)) \\
  -Dado unas computadoras c1 y c2 de una red r, se debe contruir un conj que contenga los caminos minimos para llegar desde la computadora c1 a c2. c1 y c2 deben tener distinto hostname.

  AuxCaminos(in r : Red, in c1 : compu , in c2 : compu, in recorrido : lista(compu), in candidatos : conj(compu)) $\rightarrow$ res : conj(lista(compu)) \\
  -Dado dos computadoras c1 y c2 de una red r, una lista de compu que estan conectadas cada una con la siguiente formando un camino que comienza con c1 y el conjunto de computadoras conectadas con el \'ultimo valor de recorrido en la red r.





\end{Interfaz} 

\begin{Representacion}
 
  \Titulo{Representación de Red}

  \begin{Estructura}{red}[rd]
    \begin{Tupla}[rd]
      \tupItem{Computadoras}{conj(compu)}
      \tupItem{Conexiones}{tried(hostname, conj(compu))} 
      \tupItem{InterfacesQueConectan}{tried(hostname,tried(hostname, interfaz))}%
      \tupItem{caminosMinimos}{tried(hostname,tried(hostname, conj(lista(compu))))} 
    \end{Tupla}
    \end{Estructura}

   

  \Rep[rd][r]{($\forall$ c : compu)(c $\in$ r.Computadoras $\Leftrightarrow$ (Definido?(r.Conexiones, c.hostname) $\wedge$ Definido?(r.InterfacesQueConectan, c.hostname) $\wedge$ Definido?(r.caminosMinimos, c.hostname) $\yluego$ \\
  ($\forall$ ca : compu)(ca $\in$ Significado(r.Conexiones, c.hostname) $\Leftrightarrow$ ca $\in$ r.Computadoras $\wedge$ (Definido?(Significado(r.InterfacesQueConectan, c.hostname), ca.hostname) $\yluego$ Significado(Significado(r.InterfacesQueConectan, c.hostname), ca.hostname) $\in$ c.interfaces) $\wedge$ (Definido?(Significado(r.InterfacesQueConectan, ca.hostname), c.hostname) $\yluego$ Significado(Significado(r.InterfacesQueConectan, ca.hostname), c.hostname) $\in$ ca.interfaces))))) $\wedge$ \\
  ($\forall$ c1, c2 : compu)(c1 $\in$ r.Computadoras $\wedge$ c2 $\in$ r.Computadoras $\Leftrightarrow$ Definido?(Significado(r.caminosMinimos, c1.hostname), c2.hostname) $\wedge$ Definido?(Significado(r.caminosMinimos, c2.hostname), c1.hostname) $\yluego$ Significado(Significado(r.caminosMinimos,c1.hostname), c2.hostname)$=$construirCaminosMinimos(r.Conexiones, c1, c2) )}

  ~
  
  \tadOperacion{construirCaminosMinimos}{diccionario(hostname,conj(compu)),compu,compu)}{conj(secu(compu))}{}
  \tadAxioma{construirCaminosMinimos(cr, c1, c2)}{auxMinimos(caminos(cr, c1, c2))}
  
  ~
  
  \tadOperacion{caminos}{diccionario(hostname,conj(compu)),compu,compu)}{conj(secu(compu))}{}
  \tadAxioma{caminos(cr, c1, c2)}{auxCaminos(cr, c1, c2, c1 $\bullet$ <>, Significado(cr, c1.hostname))}
  
  ~
  
  \tadOperacion{auxCaminos}{diccionario(hostname,conj(compu)),compu,compu,secu(compu),conj(compu)}{conj(secu(compu))}{}
  \tadAxioma{auxCaminos(r, c1, c2, recorrido, candidatos)}{\IF $\phi$?(candidatos) THEN $ag(<>, \phi)$ ELSE {\IF ult(recorrido) = c2 THEN ag(recorrido, $\phi$) ELSE {\IF $\neg$esta?(dameUno(candidatos), recorrido) THEN auxCaminos(r, c1 , c2 , recorrido $\circ$ dameUno(candidatos), Significado(r, dameUno(candidatos).hostname)) $\cup$ auxCaminos(r, c1 , c2 , recorrido, sinUno(candidatos)) ELSE auxCaminos(r, c1 , c2 , recorrido, sinUno(candidatos)) FI} FI} FI}
 
  ~
  
  \tadOperacion{auxMinimos}{conj(secu(compu))}{bool}{}
  \tadAxioma{auxMinimos(cc)}{\IF $\phi$?(cc) THEN $\phi$ ELSE {\IF $\#$(cc) = 1 THEN ag(dameUno(c), $\phi$) ELSE {\IF long(dameUno(cc)) < long(dameUno(auxMinimos(sinUno(cc)))) THEN ag(dameUno(cc), $\phi$) ELSE {\IF long(dameUno(cc)) = long(dameUno(auxMinimos(sinUno(cc)))) THEN ag(dameUno(c), auxMinimos(sinUno(cc))) ELSE auxMinimos(sinUno(cc)) FI} FI} FI}FI}
  
  ~
 
  \Abs[rd]{red}[r]{e}{Computadoras(e) $=$ r.Computadora $\yluego$ ($\forall$ c1, c2 : compu)(conectadas?(e, c1, c2) = (c2 $\in$ Significado(r.Conexiones, c1.hostname) $\wedge$ c1 $\in$ Significado(r.Conexiones, c2.hostname)) $\yluego$ interfazUsada(e, c1, c2) $=$ Significado(Significado(r.InterfacesQueConectan, c1.hostname), c2.hostname))}
O(n x L + Cardinal(c.interfaces))
  
\end{Representacion}	


\begin{Algoritmos}

\begin{algorithm}[H]{\textbf{iIniciarRed}() $\to$ $res$ : $red$} 
	\begin{algorithmic}
	
			\State res $\gets$ <Vac\'io(),Vac\'io(),Vac\'io(),Vac\'io()>  \Comment O(1) 
			\medskip
			\Statex \underline{Complejidad:} O(1)
			\Statex \underline{Justificación:} 
    	\end{algorithmic}
\end{algorithm}

\begin{algorithm}[H]{\textbf{iAgregarCompu}(\Inout {r}{Red}, \In {c}{compu})} 
	\begin{algorithmic}
			\State Agregar(r.Computadoras, c) 		   \Comment $O(\sum_{a' \in r.Computadoras} equal(c,a'))$%costo de agregar c al conjunto lineal r.computadoras
			\State Definir(r.Conexiones, c.hostname, Vac\'io())	\Comment O(long(c.hostname) + copy(c.hostname))	= O(long(c.hostname)) = O(L)
			\State Definir(r.InterfacesQueConectan, c.hostname, Vac\'io())       \Comment O(long(c.hostname) + copy(c.hostname))	= O(long(c.hostname)) = O(L)
			\State Definir(r.caminosMinimos, c.hostname, Vac\'io())       \Comment O(long(c.hostname) + copy(c.hostname))	= O(long(c.hostname)) = O(L)
			\State iteradorCompu : itConj(compu) $\gets$ crearIT(r.Computadoras) 		\Comment O(1)
			
			\While{HaySiguiente(iteradorCompu)} 				\Comment O((n-1) x L) = O(n x L)
			  \If{Siguiente(iteradorCompu).hostname $\neq$ c.hostname} 		\Comment O(3 x L + 4) = O(L)
			    \State vacia1 : lista(compu) $\gets$ Vac\'ia()   			\Comment O(1)
			    \State vacia2 : lista(compu) $\gets$ Vac\'ia()   			\Comment O(1)
			    \State Agregar(vacia1, Vac\'ia()) 		\Comment O(1) ya que no hay elementos en vacia1
			    \State Agregar(vacia1, Vac\'ia()) 		\Comment O(1) ya que no hay elementos en vacia2
			    \State Definir(Significado(r.caminosMinimos, c.hostname) , Siguiente(iteradorCompu).hostname, vacia1) 		\Comment O(2 x L) = O(L)
			    \State Definir(Significado(r.caminosMinimos, Siguiente(iteradorCompu).hostname) , c.hostname, vacia2) 		\Comment O(2 x L) = O(L)
			  \EndIf
			  \State Avanzar(iteradorCompu)   \Comment O(1)
			\EndWhile
			
			
			\medskip
			\Statex \underline{Complejidad:} O(n x L + Cardinal(c.interfaces))
			\Statex \underline{Justificación:} O(n x L + 3 x L + $\sum_{a' \in r.Computadoras} equal(c,a')$ +1) = O(n x L + $\sum_{a' \in r.Computadoras} equal(c,a')$). El costo de ver si dos computadoras son iguales es O(Cardinal(c.interfaces) + L), por las interfaces y el hostname. Nos queda O(n x L + Cardinal(c.interfaces) + L) = O(n x L + Cardinal(c.interfaces))  
    	\end{algorithmic}
\end{algorithm}

\begin{algorithm}[H]{\textbf{iConectar}(\Inout {r}{Red}, \In {c1}{compu}, \In {c2}{compu}, \In {i1}{interfaz}, \In {i2}{interfaz})} 
	\begin{algorithmic}
			\State AgregarRapido(Significado(r.Conexiones, c1.hostname), c2) 		\Comment O(copy(c2)) = O(Cardinal(c2.interfaces) + long(c2.hostname)) = O(Cardinal(c2.interfaces) + L)
			\State AgregarRapido(Significado(r.Conexiones, c2.hostname), c1)		\Comment O(copy(c1)) = O(Cardinal(c1.interfaces) + long(c1.hostname)) = O(Cardinal(c1.interfaces) + L)
			\If{$\neg$Definido?(r.InterfacesQueConectan, c1.hostname)} 	\Comment O(logn(c1.hostname)) = O(L)
			  \State Definir(r.InterfacesQueConectan, c1.hostname, Vac\'io()) \Comment O(logn(c1.hostname)) = O(L)
			\EndIf
			
			\If{$\neg$Definido?(r.InterfacesQueConectan, c2.hostname)} 		\Comment O(logn(c2.hostname)) = O(L)
			  \State Definir(r.InterfacesQueConectan, c2.hostname, Vac\'io()) 	\Comment O(logn(c2.hostname)) = O(L)
			\EndIf
			
			\State Definir(Significado(r.InterfacesQueConectan,c1.hostname), c2.hostname, i2) 	\Comment O(logn(c2.hostname) + logn(c1.hostname)) = O(L + L) = O(L)
			\State Definir(Significado(r.InterfacesQueConectan,c2.hostname), c1.hostname, i1) 	\Comment O(logn(c2.hostname) + logn(c1.hostname)) = O(L + L) = O(L)
			%ALGO PARA Los caminos minimos!
			
			
			\State iteradorCompu1 : itConj(compu) $\gets$ crearIT(r.Computadoras) 		\Comment O(1)
			\While{HayCamino(iteradorCompu1)} \Comment =O((L x n + Cardinal(dU.interfaces))x $n^{3}$ + n) = O((L x n + Cardinal(dU.interfaces))x $n^{3}$)
			  \State iteradorCompu2 : itConj(compu) $\gets$ crearIT(r.Computadoras) 		\Comment O(1)
			  \While{HayCamino(iteradorCompu2)} 	\Comment O(2 x (L x n + Cardinal(dU.interfaces))x $n^{2}$ + n) = O((L x n + Cardinal(dU.interfaces))x $n^{2}$)
			    \If{Siguiente(iteradorCompu1).hostname $\neq$ Siguiente(iteradorCompu2).hostname} \Comment O(L + (L x n + Cardinal(dU.interfaces))x n) = O((L x n + Cardinal(dU.interfaces))x n)
			      \State Definir(Significado(r.caminosMinimos, Siguiente(iteradorCompu1).hostname), Siguiente(iteradorCompu2).hostname, ConstruirCaminosMinimos(r, Siguiente(iteradorCompu1), Siguiente(iteradorCompu2))) \Comment O(2 x L + (L x n + Cardinal(dU.interfaces))x n)=O((L x n + Cardinal(dU.interfaces))x n) donde dU es la computadora entre los candidatos con mayor cantidad de interfaces conectadas.
			      \State Definir(Significado(r.caminosMinimos, Siguiente(iteradorCompu2).hostname), Siguiente(iteradorCompu1).hostname, ConstruirCaminosMinimos(r, Siguiente(iteradorCompu2), Siguiente(iteradorCompu1))) \Comment O(2 x L + (L x n + Cardinal(dU.interfaces))x n)=O((L x n + Cardinal(dU.interfaces))x n) donde dU es la computadora entre los candidatos con mayor cantidad de interfaces conectadas.
			    \EndIf
			  
			    \State Avanzar(iteradorCompu2) 	\Comment O(1)
			  \EndWhile
			  \State Avanzar(iteradorCompu1) 	\Comment O(1)
			\EndWhile
			
			
			\medskip
			\Statex \underline{Complejidad:} O((L x n + Cardinal(dU.interfaces))x $n^{3}$) donde dU es la computadora de la red con m\'as interfaces.
			\Statex \underline{Justificación:} O(Cardinal(c1.interfaces) + Cardinal(c2.interfaces) + 8 x L + 1 + (L x n + Cardinal(dU.interfaces))x $n^{3}$) = O((L x n + Cardinal(dU.interfaces))x $n^{3}$) . Podemos usar AgregarRapido ya que la precondici\'on de conectar nos dice que las computadoras c1 y c2 est\'an en la red y no estan conectadas entre si. Como Cardinal(dU.interfaces) es la cantidad de interfaces de la compu con m\'as interfaces esta le gana a los cardinales de c1.interfaces y c2.interfaces
    	\end{algorithmic}
\end{algorithm}



\begin{algorithm}[H]{\textbf{iComputadoras}(\In {r}{Red}) $\to$ $res$ : $conj(compu)$} 
	\begin{algorithmic}
			\State res $\gets$ r.Computadoras  		   \Comment O(1)
			\medskip
			\Statex \underline{Complejidad:} O(1)
			\Statex \underline{Justificación:} 
    	\end{algorithmic}
\end{algorithm}

\begin{algorithm}[H]{\textbf{iConectadas?}(\In {r}{Red}, \In {c1}{compu}, \In {c2}{compu}) $\to$ $res$ : $bool$} 
	\begin{algorithmic}
			\State res $\gets$ Pertenece?(Significado(r.Conexiones, c1.hostname), c2)   \Comment O($\Sigma_{a' \in Significado(r.Conexiones, c1.hostname)} equal(c2.hostname,a'))$ + long(c1.hostname))
			\medskip
			\Statex \underline{Complejidad:} O($\Sigma_{a' \in Significado(r.Conexiones, c1.hostname)} equal(c2.hostname,a'))$)
			\Statex \underline{Justificación:} O($\Sigma_{a' \in Significado(r.Conexiones, c1.hostname)} equal(c2.hostname,a'))$ + long(c1.hostname)) = O($\Sigma_{a' \in Significado(r.Conexiones, c1.hostname)} equal(c2.hostname,a'))$ + L) = O($\Sigma_{a' \in Significado(r.Conexiones, c1.hostname)} equal(c2.hostname,a'))$). Como comparar las computadoras es hacerlo con sus hostname e interfaces entonces puedo decir que L pierde contra la comparaci\'on de todas las computadoras que puede tener el conjunto. 
    	\end{algorithmic}
\end{algorithm}

\begin{algorithm}[H]{\textbf{iInterfazUsada}(\In {r}{Red}, \In {c1}{compu}, \In {c2}{compu}) $\to$ $res$ : $interfaz$} 
	\begin{algorithmic}
			\State res $\gets$ Significado(Significado(r.InterfacesQueConectan, c1.hostname), c2.hostname) 	\Comment O(long(c1.hostname) + long(c2.hostname))
			

			\medskip
			\Statex \underline{Complejidad:} O(L)
			\Statex \underline{Justificación:} O(long(c1.hostname) + long(c2.hostname)) = O(L + L) = O(L)
    	\end{algorithmic}
\end{algorithm}

\begin{algorithm}[H]{\textbf{iVecinos}(\In {r}{Red}, \In {c}{compu}, \In {i}{interfaz}) $\to$ $res$ : $conj(compu)$} 
	\begin{algorithmic}
			\State res $\gets$ Significado(r.Conexiones, c.hostname) 		   \Comment O(long(c.hostname))
			 
			\medskip
			\Statex \underline{Complejidad:} O(L)
			\Statex \underline{Justificación:} O(long(c.hostname)) = O(L)
    	\end{algorithmic}
\end{algorithm}

\begin{algorithm}[H]{\textbf{iUsaInterfaz?}(\In {r}{Red}, \In {c}{compu}, \In {i}{interfaz}) $\to$ $res$ : $bool$} 
	\begin{algorithmic}
			\State iteradorConexion : itConj(hostname)$\gets$ crearIT(Claves(Significado(r.InterfacesQueConectan, c1.hostname))) 	\Comment O(long(c1.hostname)) = O(L)
			\State encontre : bool $\gets$ false 		\Comment O(1)
			\While{HaySiguiente(iteradorConexion) $\wedge$ $\neg$encontre}  \Comment O(n x L)
			  \If{Significado(Significado(r.InterfacesQueConectan, c1.hostname), Siguiente(iteradorConexion)) $\neq$ i} 	\Comment O(1 + long(c1.hostname) + long(Siguiente(iteradorConexion)) = O(1 + L + L) = O(L)
			    \State encontre $\gets$ true 		\Comment O(1)
			  \EndIf
			  \State Avanzar(iteradorConexion) 					\Comment O(1)
			\EndWhile
			
			 \State res $\gets$ HaySiguiente(iteradorConexion)  			\Comment O(1)			 
			 
			\medskip
			\Statex \underline{Complejidad:} O(n x L)
			\Statex \underline{Justificación:} O(n x L + 1 + L) = O(n x L)
    	\end{algorithmic}
\end{algorithm}

\begin{algorithm}[H]{\textbf{iCaminosMinimos}(\In {r}{Red}, \In {c1}{compu}, \In {c2}{compu}) $\to$ $res$ : $bool$} 
	\begin{algorithmic}
			\State res $\gets$ Significado(Significado(r.caminosMinimos, c1.hostname), c2.hostname) 		\Comment O(long(c1.hostname) + long(c2.hostname))
			 
			\medskip
			\Statex \underline{Complejidad:} O(L)
			\Statex \underline{Justificación:} O(long(c1.hostname) + long(c2.hostname)) = O(L + L) = O(L)
    	\end{algorithmic}
\end{algorithm}

\begin{algorithm}[H]{\textbf{iHayCamino?}(\In {r}{Red}, \In {c1}{compu}, \In {c2}{compu}) $\to$ $res$ : $bool$} 
	\begin{algorithmic}

			\State res $\gets$ $\neg$EsVac\'ia?(Siguiente(crearIT(Significado(Significado(r.caminosMinimos, c1.hostname), c2.hostname), Vac\'ia()))) \Comment O(long(c1.hostname) + long(c2.hostname) + 2)
			\medskip
			\Statex \underline{Complejidad:} O(L)
			\Statex \underline{Justificación:} O(long(c1.hostname) + long(c2.hostname) + 2) = O(L + L) = O(L)
    	\end{algorithmic}
\end{algorithm}

\begin{algorithm}[H]{\textbf{iConstruirCaminosMinimos}(\In {r}{Red}, \In {c1}{compu} , \In {c2}{compu}) $\to$ $res$ : $conj(lista(compu))$} 
	\begin{algorithmic}
	
	\State recorrido $\gets$ AgregarAdelante(Vac\'ia(), c1)  		\Comment O(1)
	\State candidatos $\gets$ Significado(r.Conexiones, c1.hostname)   \Comment O(long(c1.hostname)) = O(L)
	\State camino $\gets$ AuxCaminos(r, c1, c2, recorrido, candidatos) 	\Comment O((L x n + Cardinal(dU.interfaces))x n) donde dU es la computadora entre los candidatos con mayor cantidad de interfaces conectadas.
	\State iterador : itConj(lista(compu)) $\gets$ crearIT(camino) 	\Comment O(1)
	\State minimoActual : lista(compu) $\gets$ Siguiente(iterador) 	\Comment O(1)
	\State Avanzar(iterador) 					\Comment O(1)
	\While{HaySiguiente(iterador)} 					\Comment O(3 x n)
	  \If{Longitud(minimoActual) > Longitud(Siguiente(iterador))} 	\Comment O(2)
	    \State minimoActual $\gets$ Siguiente(iterador) 		\Comment O(1)
	  \EndIf
	  \State Avanzar(iterador) 					\Comment O(1)
	\EndWhile 
	\State iterador2 : itConj(lista(compu)) $\gets$ crearIT(camino) 	\Comment O(1)
	\While{HaySiguiente(iterador2)} 				\Comment O(3 x n)
	  \If{Longitud(minimoActual) < Longitud(Siguiente(iterador2))}  	\Comment O(2)
	    \State EliminarSiguiente(iterador2) 			 \Comment O(1)
	  \Else
	    \State Avanzar(iterador) 					\Comment O(1)
	  \EndIf

	\EndWhile
	\State res $\gets$ camino  	\Comment O(1)
			\medskip
			\Statex \underline{Complejidad:} O((L x n + Cardinal(dU.interfaces))x n) donde dU es la computadora entre los candidatos con mayor cantidad de interfaces conectadas.
			\Statex \underline{Justificación:} O(6 x n + 6 + L + (L x n + Cardinal(dU.interfaces))x n) = O((L x n + Cardinal(dU.interfaces))x n) donde dU es la computadora entre los candidatos con mayor cantidad de interfaces conectadas.
    	\end{algorithmic}
\end{algorithm}
\begin{algorithm}[H]{\textbf{iAuxCaminos}(\In {r}{Red}, \In {c1}{compu} , \In {c2}{compu}, \In {recorrido}{lista(compu)}, \In {candidatos}{conj(compu)}) $\to$ $res$ : $conj(lista(compu))$} 
	\begin{algorithmic}
			\If {EsVac\'io?(candidatos)} 	\Comment O(2)
			  \State res $\gets$ AgregarRapido(Vac\'io(), Vac\'ia())  \Comment O(copy(Vac\'ia())) = O(1)
			\ElsIf {Ultimo(recorrido).hostname = c2.hostname} 	        \Comment O(n x (Cardinal(recorrido[j].interfaces)+L)) donde j es el indice de la compu con mayor cantidad de interfaces conectadas. 
			    \State res $\gets$ AgregarRapido(Vac\'io(), recorrido) 	\Comment O(copy(recorrido))=O(Copiar(recorrido))=O($\sum_{i=1}^{long(recorrido)}copy(recorrido[i])$)=O($\sum_{i=1}^{long(recorrido)}$ \\$(Cardinal(recorrido[i].interfaces) + long(recorrido[i].hostname))$)$=$O($\sum_{i=1}^{long(recorrido)}$ \\$(Cardinal(recorrido[i].interfaces) + L))$)$=$O(n x (Cardinal(recorrido[j].interfaces)+L))donde j es el indice de la compu con mayor cantidad de interfaces conectadas.
			  \Else  						\Comment O(3 + n x L + $Cardinal(dUno.interfaces)+L$ + 2 x (AuxCaminos(r, c1, c2, recorrido, candidatos) - 1) + 1 + n) = O(n x L + $Cardinal(dUno.interfaces)+L$ + (AuxCaminos(r, c1, c2, recorrido, candidatos) - 1))
			      \State iterador : itLista(compu) $\gets$ crearIT(recorrido)   	\Comment O(1)
			      \State dameUno : compu $\gets$ Siguiente(crearIT(candidatos))     \Comment O(1)
			      \State loEncontre : bool $\gets$ false 				\Comment O(1)
			      \While{HaySiguiente(iterador) $\wedge$ $\neg$loEncontre}   \Comment O(L x n)
				\If{dameUno.hostname $=$ Siguiente(iterador).hostname}   \Comment O(MAX(long(Siguiente(iterador).hostname), long(dameUno.hostname)) + 1) = O(L)
				  \State loEncontre $\gets$ true 			\Comment O(1)
				\EndIf
				\State Avanzar(iterador) 				\Comment O(1)
			      \EndWhile
			      
			      \If {$\neg$loEncontre} 			\Comment O($Cardinal(dUno.interfaces)+L$ + 2 x (AuxCaminos(r, c1, c2, recorrido, candidatos) - 1) + 1 + n) donde dU es la computadora entre los candidatos con mayor cantidad de interfaces conectadas.
				  \State camino1 : conj(lista(compu)) $\gets$ AuxCaminos(r, c1 , c2 , AgregarAtras(recorrido, dameUno), Significado(r, dameUno.hostname)) \Comment O(long(dameUno.hostname) + (AuxCaminos(r, c1, c2, recorrido, candidatos) - 1))
				  \State camino2 : conj(lista(compu)) $\gets$ AuxCaminos(r, c1 , c2 , recorrido, Eliminar(candidatos, dameUno)) 			\Comment O($\sum_{a' \in candidatos} equal(dameUno,a')$ + (AuxCaminos(r, c1, c2, recorrido, candidatos) - 1)) = O($Cardinal(dUno.interfaces)+L$ + (AuxCaminos(r, c1, c2, recorrido, candidatos) - 1)) donde dU es la computadora entre los candidatos con mayor cantidad de interfaces conectadas.
				  \State iteradorCamino : itConj(lista(compu)) $\gets$ crearIT(camino2) 	\Comment O(1)
				  \While{HayCamino(iteradorCamino)} 		\Comment O((1 + 1)x n)=O(2 x n) = O(n)
				    \State AgregarRapido(camino1, Siguiente(iteradorCamino))  \Comment O(copy(Siguiente(iteradorCamino))) $\leq$ O(1), pues O(siguiente(iteradorCamino))=O(1).
				    \State Avanzar(iteradorCamino) \Comment O(1)
				  \EndWhile
			      \Else
				\State res $\gets$ AuxCaminos(r, c1 , c2 , recorrido, Eliminar(candidatos, dameUno)) 		\Comment O((AuxCaminos(r, c1, c2, recorrido, candidatos) - 1))
			      \EndIf
			  \EndIf      
			\medskip
			\Statex \underline{Complejidad:} O((L x n + 1+ Cardinal(dU.interfaces)+ L)x n) = O((L x n + Cardinal(dU.interfaces))x n) donde dU es la computadora entre los candidatos con mayor cantidad de interfaces conectadas.%Creo  que es así porque tenemos a lo sumo n llamados recursivos y lo peor que puede pasar es que no encontremos el candidato hasta la última iteración
			\Statex \underline{Justificación:} Como cada vez que llamamos recursivamente a la funcion AuxCaminos estamos continuando con la contrucci\'on del mismo a travez de recorrido, podemos decir que la funci\'on demora a lo sumo n en cada recurci\'on. Esto se debe a que el camino no tiene repetidos.
    	\end{algorithmic}
\end{algorithm}


\end{Algoritmos}
\end{document}


\section{M�dulo DcNet}


\begin{Interfaz}

  \textbf{se explica con}: {\tadNombre{DCNet}}.

  \textbf{g�neros}:{ \TipoVariable{dcnet}}.

  \Titulo{Operaciones b�sicas de DCNet}
%iniciarDCNet:
  \InterfazFuncion{iniciarDCNet}{\In {r}{red}}{dcnet}%
  {res $\igobs$ iniciarDCNet(r)}%
  [O(n x L)]
  [genera una dcnet con la red r.]
  
%  crearPaquete:
 \InterfazFuncion{crearPaquete}{\Inout {d} {dcnet} , \In {p} {paquete} }{} % {nombreFunci�n}{Signatura}{Res}
  [$(\neg ( (\exists p':paquete) (paqueteEnTransito(s,p') \wedge id(p') = id(p))
\wedge origen(p) \in computadoras(red(s))$
\yluego $destino(p) \in computadoras(red(s))$
\yluego $hayCamino?(red(s), origen(p), destino(p) )) \wedge d\igobs d_{0}$] % "Condici�n del tad" es: (\neg ( (\exists p':paquete) (paqueteEnTransito(s,p') \wedge id(p') = id(p))
%\wedge origen(p) \in computadoras(red(s))$
%\yluego $destino(p) \in computadoras(red(s))$ est� copiado textual del TAD, si hay que cambiarlo avisenme
%\yluego $hayCamino?(red(s), origen(p), destino(p) ))
  {$d \igobs crearPaquete(d_{0}, p)$} %POST
  [O($L + log(k)$)] %COMPLEJIDAD
  [A�ade el paquete p a la dcnet d. Poniendolo en la cola de los paquetes de la computadora inicial.] %DESCRIPCI�N
  [] %ALIASING

%avanzarSegundo:
  \InterfazFuncion{avanzarSegundo}{\Inout {d} {dcnet}}{} % {nombreFunci�n}{Signatura}{Res}
  [$d \igobs d_{0}$] %PRE
  {$d \igobs avanzarSegundo(d_{0})$} %POST
  [O(n x (L + log(n) + log(k)))] %COMPLEJIDAD
  [genera la dcnet correspondiente a pasar un segundo en d.] %DESCRIPCI�N
  [] %ALIASING
  
%red:
  \InterfazFuncion{red}{\In {d}{dcnet}}{red} % {nombreFunci�n}{Signatura}{Res}
  %PRE es true
  {$res \igobs red(d)$} %POST
  [O(1)] %COMPLEJIDAD
  [Devuelve la red asosiada a la dcnet.] %DESCRIPCI�N
  [] %ALIASING
  
%enEspera:
  \InterfazFuncion{enEspera}{\In {d}{dcnet} , \In  {c}{compu}}{colaPriori} % {nombreFunci�n}{Signatura}{Res}
  [$c \in computadoras(red(d))$] %PRE
  {$res \igobs enEspera(d)$} %POST
  [O(L)] %COMPLEJIDAD
  [Devuelve los paquetes de la compu c en d.] %DESCRIPCI�N
  [res es modificables si y s\'olo si d es modificables.] %ALIASING
  
%caminoRecorrido:
  \InterfazFuncion{caminoRecorrido}{\In {d}{dcnet} , \In {p}{paquete}}{secu(compu)} % {nombreFunci�n}{Signatura}{Res}
  [$(\exists c:compu)(c \in computadoras(red(d)) \yluego $est\' a?$(p, enEspera(d, c)) )$] %PRE
  {$res \igobs caminoRecorrido(d, p)$} %POST
  [O(n + log(k))] %COMPLEJIDAD
  [Devuelve el camino recorrido por el paquete p en la dcnet. (Este debe ser el m�s corto)] %DESCRIPCI�N
  [] %ALIASING
  
%cantidadEnviados:
  \InterfazFuncion{cantidadEnviados}{\In {d}{dcnet}, \In {c}{compu}}{nat} % {nombreFunci�n}{Signatura}{Res}
  [$c \in computadoras(red(d))$] %PRE
  {$res \igobs cantidadEnviados(d, c)$} %POST
  [O(L)] %COMPLEJIDAD
  [Devuelve la cantidad de paquetes que envio la compu c.] %DESCRIPCI�N
  [res es modificables si y s\'olo si d es modificables.] %ALIASING
  
%paqueteEnTransito?:
  \InterfazFuncion{paqueteEnTransito?}{\In {d}{dcnet}, \In {p}{paquete}}{bool} % {nombreFunci�n}{Signatura}{Res}
%PRE es true
  {$res \igobs paqueteEnTransito?(d, p)$} %POST
  [O(n x (L + log(k)))] %COMPLEJIDAD
  [Devuelve true si existe alguna computadora que tenga ese paquete.] %DESCRIPCI�N
  [] %ALIASING
  
%laQueMasEnvio:
  \InterfazFuncion{laQueMasEnvio}{\In {d}{dcnet}}{compu} % {nombreFunci�n}{Signatura}{Res}
%PRE es true
  {$res \igobs LaQueMasEnvio(d)$} %POST
  [O(1)] %COMPLEJIDAD
  [Devuelve la computadora que envio m�s paquetes.] %DESCRIPCI�N
  [] %ALIASING


%  \InterfazFuncion{}{}{} % {nombreFunci�n}{Signatura}{Res}
%  [] %PRE
%  {} %POST
%  [] %COMPLEJIDAD
%  [] %DESCRIPCI�N
%  [] %ALIASING

 
\end{Interfaz}

\begin{Representacion}
 
  \Titulo{Representaci�n de DCNet}

  \begin{Estructura}{dcnet}[dc]
    \begin{Tupla}[dc]
      \tupItem{LaQueMasEnvio}{puntero(compu)}%
      \tupItem{enEspera}{tried(hostname, colaPaquetes)}
       \tupItem{\#Enviados}{tried(hostname, nat)}%
       \tupItem{paquetesEnTransitoNoOrigen}{Avld(id, itLista(compu))} 
        \tupItem{Red}{Red}%
    \end{Tupla}
    \end{Estructura}

   

  \Rep[dc][d]{($\forall$ c : compu)(c $\in$ Computadoras(d.Red) $\Leftrightarrow$ Definido?(d.enEspera, c.hostname) $\wedge$ Definido?(d.$\#$Enviados, c.hostname)) $\wedge$ *d.LaQueMasEnvio $\in$ Computadoras(d.Red) $\yluego$ ($\neg$(Existe ca : compu)(ca $\in$ Computadoras(d.Red) $\yluego$ Significado(d.$\#$Enviados,ca.hostname) > Significado(d.$\#$Enviados, *d.LaQueMasEnvio.hostname))) $\wedge$ ($\forall$ h : Computadoras(d.Red))($\forall$ p : paquete)((p $\in$ pasarColaAConjunto(Significado(d.enEspera, h.hostname)) $\wedge$ h.hostname $\neq$ p.origen.hostname) $\Leftrightarrow$ (Definido?AVL(d.paquetesEnTransitoNoOrigen,p.id) $\yluego$ h $=$ Siguiente(SignificadoAVL(d.paquetesEnTransitoNoOrigen, p.id))))}
  ~
  \tadOperacion{pasarColaAConjunto}{colaPrior(paquete)}{conj(paquete)}{}
  \tadAxioma{pasarColaAConjunto(c)}{\IF vac\'ia?(c) THEN $\phi$ ELSE ag(Pr\'oximo(c) , pasarColaAConjunto(Desencolar(c))) FI}
  ~

  \Abs[dc]{dcnet}[d]{e}{d.Red $=$ red(e) $\wedge$ ($\forall$ c : compu)((c $\in$ Computadoras(red(e)) $\impluego$ (cantidadEnviados(e, c) $=$ Significado(d.$\#$Enviados, c.hostname) $\wedge$ enEspera(e, c) $=$ Significado(d.enEspera, c.hostname))) $\wedge$ ($\forall$ p : paquete)(p $\in$ pasarColaAConjunto(Significado(d.enEspera, c.hostname)) $\impluego$ \\($\neg$Definido?AVL(d.paquetesEnTransitoNoOrigen, p.id) $\Leftrightarrow$ caminoRecorrido(e,p) $=$ <>  $\wedge$ \\ Definido?AVL(d.paquetesEnTransitoNoOrigen, p.id) $\impluego$  caminoRecorrido(e,p) $=$ p.origen $\bullet$ \\ ContruirCamino(SignificadoAVL(d.paquetesEnTransitoNoOrigen, p.id)))))}
  
    ~
    
  \tadOperacion{ConstruirCamino}{itLista(compu))}{secu(compu)}{}
  \tadAxioma{ConstruirCamino(c)}{\IF HayAnterior(c) THEN ConstruirCamino(Retroceder(c)) $\circ$ Anterior(c) ELSE <> FI}
  
\end{Representacion}% PARA ITERADORES SI USARAMOS	

\begin{Algoritmos}

%iIniciarDCNet:
\begin{algorithm}[H]{\textbf{iIniciarDCNet}(\In{r}{red}) $\to$ $res$ : $dc$} 
	\begin{algorithmic}
			\State enviados : tried(hostname, nat) $\gets$  Vac\'io() 		   	\Comment O(1)    
			\State iteradorCompu : itConj(compu) $\gets$ crearIT(r.Computadoras)	\Comment O(1)
			\State enespera : tried(hostname, colaPaquetes) $\gets$ Vac\'io() 			\Comment O(1)
			\State *laQueMasEnvio : puntero(compu) $\gets$ NULL 			\Comment O(1)
			\If{HaySiguiente(iteradorCompu)} 		\Comment O(2)
			   \State laQueMasEnvio $\gets$ Siguiente(iteradorCompu) 	\Comment O(1)
			\EndIf
			
			\While{HaySiguiente(iteradorCompu)} \		Comment O(n x ((2 x L) + 1))  = O(n x L)
				\State Definir(enviados, Siguiente(iteradorCompu).hostname, 0) 		\Comment O(long(Siguiente(iteradorCompu).hostname)) = O(L)
				\State Definir(enespera, Siguiente(iteradorCompu).hostname, Vac\'io()) 	\Comment O(long(Siguiente(iteradorCompu).hostname)) = O(L)
				\State Avanzar(iteradorCompu) 						 \Comment O(1)
			\EndWhile
			 
			 \State res $\gets$ <laQueMasEnvio, enespera, enviados, Vac\'io(), r> 		\Comment O(1)

			\medskip
			\Statex \underline{Complejidad:} O(n x L)
			\Statex \underline{Justificaci�n:} O(n x L + 8) = O(n x L)
    	\end{algorithmic}
\end{algorithm}

%icrearPaquete:
\begin{algorithm}[H]{\textbf{iCrearPaquete}(\Inout{d}{dc}, \In {p}{paquete})} 
	\begin{algorithmic}
			\State Encolar(p, Significado(d.enEspera, p.origen.hostname))\Comment O(long(p.origen.hostname) + log(k))   

			\medskip
			\Statex \underline{Complejidad:} O(L + log(k))
			\Statex \underline{Justificaci�n:} O(long(p.origen.hostname) + log(k)) = O(L + log(k))
    	\end{algorithmic}
\end{algorithm}

%iavanzarSegundo:





\begin{algorithm}[H]{\textbf{iavanzarSegundo}(\Inout{d}{dc})} 
	\begin{algorithmic}
			\State ListaAEnviar : lista(tupla<hostname, paquete>) $\gets$ Vac\'io() 					\Comment O(1)
			\State iteradorCompu : itConj(compu) $\gets$ crearIT(Computadoras(d.red)) 					\Comment O(1)
			\While {HaySiguiente(iteradorCompu)}				\Comment O(n x (L + log(n) + log(k)))
			
			  \If{$\neg$Vac\'ia?(Significado(d.enEspera, iteradorCompu.hostname))} 			\Comment O(2 x L + log(n) + log(k)) = O(L + log(n) + log(k))
				\State Significado(d.\#Enviados, Siguiente(iteradorCompu).hostname) + 1    	\Comment O(L)
				\If{Significado(d.\#Enviados, Siguiente(iteradorCompu).hostname) > Significado(d.$\#$Enviados, *d.LaQueMasEnvio.hostname)}  	\Comment O(2 x L) = O(L)
				  \State d.LaQueMasEnvio $\gets$ $\delta$Siguiente(iteradorCompu) 			\Comment O(1)
				\EndIf
				\State aEnviar : paquete $\gets$ Pr\'oximo(Significado(d.enEspera, Siguiente(iteradorCompu).hostname))   \Comment O(L + log(k))
				\State Desencolar(Significado(d.enEspera, Siguiente(iteradorCompu).hostname)) 		\Comment O(L + log(k))
				
				\If{Definida?AVL(d.paquetesEnTransitoNoOrigen, p.id)} 			\Comment O(2x(log(n) + log(k))) = O(log(n) + log(k))
				    \State cam : itLista(compu) $\gets$ SignificadoAVL(d.paquetesEnTransitoNoOrigen, p.id) 	\Comment O(log(n)+ log(k))
				    \If{HaySiguiente(cam)}  					\Comment O(3)
				      \State AgregarAtras(ListaAEnviar, <Siguiente(cam).hostname, aEnviar>) 		\Comment O(copy(<Siguiente(cam).hostname, aEnviar>)) = O(1)
				      \State Avanzar(cam) 					\Comment O(1)
				    \Else 								\Comment O(log(n) + log(k))
				      \State BorrarAVL(d.paquetesEnTransitoNoOrigen, p.id) 		\Comment O(log(n) + log(k))
				    \EndIf
				\Else 										\Comment  O(L + 2 x log(k) + 2 x log(n)) = O(L + log(n) + log(k))
				  \State camino : itLista(compu) $\gets$ crearIT(Siguiente(crearIT(caminosMinimos(d.red, p.origen, p.destino))))  \Comment O(L)
				  \State Avanzar(camino) 			\Comment O(1)
				  \If{HaySiguiente(camino)} 						\Comment O(2 + log(n) + 2 x log(k)) = O(log(n) + log(k))
				      \State AgregarAtras(ListaAEnviar, <Siguiente(camino).hostname, aEnviar>) 	\Comment O(copy(<Siguiente(camino).hostname, aEnviar>)) = O(1)
				      \State DefinirAVL(d.paquetesEnTransitoNoOrigen, p.id, camino) 	\Comment O(log(n) + log(k))
				    \EndIf 
				\EndIf
				
			  \EndIf
			  \State Avanzar(iteradorCompu) 				\Comment O(1)
			\EndWhile

			\State iteradorEnviar : itLista(tupla<hostname, paquete>) $\gets$ crearIT(ListaAEnviar) 		\Comment O(1)
			
			\While{HaySiguiente(iteradorEnviar)} 				\Comment O(n x (L + log(k)))
			  \State Encolar(Significado(d.enEspera, PI1(Siguiente(iteradorEnviar)) , PI2(Siguiente(iteradorEnviar)))) 	\Comment O(L + log(k))
			  \State Avanzar(iteradorEnviar) 										\Comment O(1)
			\EndWhile
			
			\medskip
			\Statex \underline{Complejidad:} O(n x (L + log(n) + log(k)))
			\Statex \underline{Justificaci�n:} O(n x (L + log(k)) + n x (L + log(n) + log(k)) + 4) = O(n x (L + log(n) + log(k))). Para facilitar la escritura de las complejidades y embellecer el seudoc\'odigo. Se decidi\'o ahorrarnos un paso en el calculo de complejidades y lo que vale O(long(hostname)) o similar se lo escribe directamente como O(L).
    	\end{algorithmic}
\end{algorithm}

%ired: 
\begin{algorithm}[H]{\textbf{iRed}(\In{d}{dc}) $\to$ $res$ : $red$} 
	\begin{algorithmic}
			\State res $\gets d.Red$ \Comment O(1)
			\medskip
			\Statex \underline{Complejidad:} O(1)
    	\end{algorithmic}
\end{algorithm}

%ienEspera:
\begin{algorithm}[H]{\textbf{ienEspera}(\In{d}{dc}, \In {c}{compu}) $\to$ $res$ : $colaPaquetes$} 
	\begin{algorithmic}
			\State res $\gets$ Significado(c.hostname, d.enEspera) \Comment O(long(c.hostname)) 
			\medskip
			\Statex \underline{Complejidad:} O(L)
			\Statex \underline{Justificaci�n:} O(long(c.hostname)) = O(L)
    	\end{algorithmic}
\end{algorithm}

%icaminoRecorrido(in d:dc, in p:paquete) -> res:  //
\begin{algorithm}[H]{\textbf{icaminoRecorrido}(\In{d}{dc}, \In {p}{paquete}) $\to$ $res$ : $lista(compu)$} 
	\begin{algorithmic}
			\State res $\gets$ Vac\'ia()						\Comment O(1)
			\If{$\neg$Definido?AVL(d.paquetesEnTransitoNoOrigen, p.id)} 			\Comment O(2 x n + 2 x log(n) + 2 x log(k)) = O(n + log(n) + log(k)) = O(n + log(k))
			    \State cam : itLista(compu) $\gets$ SignificadoAVL(d.paquetesEnTransitoNoOrigen, p.id)	\Comment O(log(n) + log(k))
			    \State i : nat $\gets$ 0 									\Comment O(1)
			    
			    \While{HayAnterior(cam)} 					\Comment O(4 x n) = O(n)	    
			      \State AgregarAdelante(res, Anterior(cam)) 	\Comment O(copy(Anterior(cam))) = O(1)
			      \State Retroceder(cam) 				\Comment O(1)
			      \State i $\gets$ i + 1 				\Comment O(1)
			    \EndWhile
			    
			     \While{i $\neq$ 0} 					\Comment O(3 x n) = O(n)
			      \State Avanzar(cam) 				\Comment O(1)
			      \State i $\gets$ i - 1 				\Comment O(1)
			    \EndWhile
			    
			\EndIf
			\medskip
			\Statex \underline{Complejidad:} O(n + log(k))
			\Statex \underline{Justificaci�n:} Qvq O(n + log(k)) $\subseteq$ O(n x log(n + k)) = O(MAX(n,log(k))) $\subseteq$ O(n x log(MAX(n,k))). Si n<=k entonces O(MAX(n, log(k))) $\subseteq$ O(n x log(k)). Si n<=log(k) entonces O(log(k)) $\subseteq$ O(n x log(k)) esto es correcto, para el caso n>log(k) O(n) $\subseteq$ O(n x log(k)) esto es correcto. Para el caso n>k tenemos que O(n) $\subseteq$ O(n x log(n)) y esto tambi\'en es correcto. Con esto vemos que en todos los casos es correcto.
    	\end{algorithmic}
\end{algorithm}

%icantidadEnviados:
\begin{algorithm}[H]{\textbf{icantidadEnviados}(\In{d}{dc}, \In {c}{compu}) $\to$ $res$ : $nat$} 
	\begin{algorithmic}
			\State res $\gets$ Significado(c.hostname, d.\# Enviados)     \Comment O(long(c.hostname))
			\medskip
			\Statex \underline{Complejidad:} O(L)
			\Statex \underline{Justificaci�n:} O(long(c.hostname)) = O(L)
    	\end{algorithmic}
\end{algorithm}

%iPaqueteEnTransito?:
\begin{algorithm}[H]{\textbf{iPaqueteEnTransito?}(\In{d}{dc}, \In {p}{paquete}) $\to$ $res$ : $bool$} 
	\begin{algorithmic}
			\State res $\gets$ false	\Comment O(1)
			\State iteradorCompu : itConj(compu) $\gets$ crearIT(Computadoras(d.red))  \Comment O(1)
			\While {HaySiguiente(iteradorCompu) $\wedge$ $\neg$ res} 		\Comment O(n x (L + log(k) + 1)) $=$ O(n x (L + log(k)))
				\State res $\gets$ Pertenece(p, Significado(d.enEspera, hostname(Siguiente(c)))))		\Comment O(long(hostname(Siguiente(c))) + log(k)) $=$ O(L + log(k))	
				\State Avanzar(iteradorCompu) 					\Comment O(1)  
			\EndWhile
			\medskip
			\Statex \underline{Complejidad:} O(n x (L + log(k)))
			\Statex \underline{Justificaci�n:} O(n x (L + log(k)) + 2) = O(n x (L + log(k)))
    	\end{algorithmic}
\end{algorithm}

%iLaQueMasEnvio:

\begin{algorithm}[H]{\textbf{iLaQueMasEnvio}(\In{d}{dc}) $\to$ $res$ : $compu$} 
	\begin{algorithmic}
			\State res $\gets$ *d.LaQueMasEnvio \Comment O(1)
			\medskip
			\Statex \underline{Complejidad:} O(1) 
    	\end{algorithmic}
\end{algorithm}




\end{Algoritmos}



\section{M�dulo Tried(k, $\sigma$)}


\begin{Interfaz}
  
  \textbf{par�metros formales}\hangindent=2\parindent\\
  \parbox{1.7cm}{\textbf{g�neros}} k, $\sigma$\\
  \parbox[t]{1.7cm}{\textbf{funci�n}}\parbox[t]{\textwidth-2\parindent-1.7cm}{%
    \InterfazFuncion{$\bullet = \bullet$}{\In{k1}{k}, \In{k2}{k}}{bool}
    {$res \igobs (k1 = k2)$}
    [$\Theta(equal(k1, k2))$]
    [funci�n de comparaci\'on de k's]
    
    \InterfazFuncion{Copiar}{\In{a}{k}}{k}
    {$res \igobs a$}
    [$\Theta(copy(a))$]
    [funci�n de copia de k's]
    
    \InterfazFuncion{Copiar}{\In{s}{$\sigma$}}{$\sigma$}
    {$res \igobs s$}
    [$\Theta(copy(s))$]
    [funci�n de copia de $\sigma$'s]
    
    \InterfazFuncion{Long}{\In{a}{k}}{nat}
    {$res \igobs long(a)$}
    [$\Theta(1)$]
    [funci�n que devuelve el tama�o de k's]
  }

  \textbf{se explica con}: \tadNombre{Diccionario$(k, \sigma)$}.

  \textbf{g�neros}: \TipoVariable{tried$(k, \sigma)$}.

  \Titulo{Operaciones b�sicas de tried}

  \InterfazFuncion{Vac\'io}{}{tried(k,$\sigma)$}%
  {$res \igobs$ vacio}%
  [$\Theta(1)$]
  [genera un tried vac\'io.]

  \InterfazFuncion{Definir}{\Inout{d}{tried(k,$\sigma)$}, \In{k}{k}, \In{s}{$\sigma$} }{}
  [$d \igobs d_0$]
  {$d \igobs definir(d_0, k, s)$}
  [O(long(k) + copy(k))]
  [define la clave k con el significado s en el tried d.]
  
  \InterfazFuncion{Definido?}{\In{d}{tried(k,$\sigma)$}, \In{k}{k}}{bool}
  {$res \igobs def?(d, k)$}
  [O(long(k) + copy(k))]
  [devuelve true si y s\'olo k est\'a definido en el tried.]

  \InterfazFuncion{Significado}{\In{d}{tried(k,$\sigma)$}, \In{k}{k}}{$\sigma$}
  [$def?(d, k)$]
  {$alias(res \igobs obtener(d, k))$}
  [O(long(k) + copy(k))]
  [devuelve el significado de la clave k en el tried d.]
  [res es modificable si y s\'olo si d es modificable.]

  \InterfazFuncion{Borrar}{\Inout{d}{tried(k,$\sigma$)}, \In{k}{k}}{}
  [$d \igobs d_0 \wedge def?(d, k)$]
  {$d \igobs borrar(d_0, k)$}
  [$O(\Sigma_{a' \in d.claves} equal(k,a')) + copy(k) + long(k))$]
  [borra la clave k del tried d.]

  \InterfazFuncion{Claves}{\In{d}{tried(k,$\sigma$)}}{conj(k)}
  {$res \igobs claves(d)$}
  [$\Theta(1)$]
  [devuelve las claves del tried.]
  [res es modificable si y s\'olo si d es modificable.]

  \InterfazFuncion{$\#$Claves}{\In{d}{tried(k,$\sigma$)}}{nat}
  {$res \igobs \#(claves(d))$}
  [$\Theta(1)$]
  [devuelve la cantidad de claves del tried.]
 
  
\end{Interfaz}

\begin{Representacion}
  
  \Titulo{Representaci�n de la tried}

  \begin{Estructura}{tried(k,$\sigma$)}[trie]
    \begin{Tupla}[trie]
      \tupItem{diccionario}{lista(nodoTrie)}%
      \tupItem{claves}{conj(k)}%
    \end{Tupla}

    \begin{Tupla}[nodoTrie]
      \tupItem{letra}{k}%
      \tupItem{final}{bool}%
      \tupItem{siguientes}{lista(nodoTrie)}%
      \tupItem{significado}{puntero($\sigma$)}%
    \end{Tupla}
  \end{Estructura}

  \Rep[trie][t]{($\forall i$: nat)((i > 0  $\wedge$ i < Longitud(t.diccionario)) $\impluego$ long(t.diccionario[i].letra) $=$ 1   $\wedge$ $\neg$LetrasReperidas?(t.diccionario) $\wedge$ (t.final $\Rightarrow$ t.significado $\neq$ NULL)}\mbox{}
  

  \tadOperacion{LetrasReperidas?}{lista(nodoTrie)}{bool}{}
  \tadAxioma{LetrasReperidas?(l)}{\IF EsVac\'ia?(l) THEN true ELSE LetrasReperidas?(Fin(l)) $\wedge$ LetrasReperidas?Aux(Fin(l), Primero(l).letra) FI}
  ~      
 \tadOperacion{LetrasReperidas?Aux}{lista(nodoTrie), k}{bool}{}
  \tadAxioma{LetrasReperidas?Aux($l$, $k$)}{\IF(EsVac�a?(l)) THEN true ELSE {\IF(Primero(l).letra $=$ k) THEN false ELSE LetrasReperidas?Aux(Fin(l), k) FI} FI}
 
   ~   
  \AbsFc[trie]{tried}[t]{e}{($\forall$ c : k)(def?(e, c) $\Leftrightarrow$ Pertenece?(t.claves, c) $\yluego$ (def?(e, c) $\impluego$ obtener(e, c) $=$ ObtenerDeEstructura(t.diccionario, c)))}
 
   ~ 
  \tadOperacion{ObtenerDeEstructura}{lista(nodoTrie), k}{$\sigma$}{}
  \tadAxioma{ObtenerDeEstructura(l, c)}{\IF Primero(l).letra = prim(k) $\wedge$ vac\'ia?(fin(k)) THEN *Primero(l).significado ELSE {\IF (Primero(l).letra = prim(k)) THEN ObtenerDeEstructura(Primero(l).siguientes, fin(k) ELSE ObtenerDeEstructura(Fin(l), k) FI} FI}
 
\end{Representacion}

\begin{Algoritmos}


\begin{algorithm}[H]{\textbf{iVac\'io}() $\to$ $res$ : trie}
	\begin{algorithmic}
			 \State res $\gets$ <Vac\'ia(), Vac\'io()> 				\Comment $\Theta(1)$
			 \medskip
			  \Statex \underline{Complejidad:} $\Theta(1)$
	
    	\end{algorithmic}
\end{algorithm}

\begin{algorithm}[H]{\textbf{iDefinir}(\Inout{d}{trie}, \In {k}{k}, \In {s}{$\sigma$})}
	\begin{algorithmic}
	
\State AgregarRapido(d.claves, k) 						\Comment O(copy(k))
\State kAux : k $\gets$ copy(k) 						\Comment O(copy(k))
\State recorrido : itLista(nodoTrie) $\gets$ crearIT(d.diccionario) 		\Comment O(1)
\While{$\neg$ vac\'ia?(kAux)} 				 			\Comment O(8 x long(k)) = O(long(k))
	\State encontrado : bool $\gets$ false 			\Comment O(1)
	\While{HaySiguiente(recorrido) $\wedge$ $\neg$ encontrado}          \Comment O(7 x cantidad de elementos asignados a la lista que itera recorrido) pero esto est\'a acotado por constante as\'i que es O(1)
		\If{Siguiente(recorrido) = prim(kAux)} 		\Comment O(2)
		  \State encontrado $\gets$ true 		\Comment O(1)
		\EndIf
		\State Avanzar(recorrido) 		\Comment O(1)
	\EndWhile
	\If{$\neg$encontrado}      \Comment O(5)
		\State nueva : lista(nodoTrie) $\gets$ Vac\'ia() 					\Comment O(1)
		\State *aAgregar : puntero($\sigma$) $\gets$ s 			\Comment O(1)
		\State AgregarComoSiguiente(recorrido, <prim(kAux) , vac\'ia?(fin(kAux)), nueva, aAgregar>) 	 \Comment O(copy(<prim(kAux) , vac\'ia?(fin(kAux)), nueva, s>)) = O(1)
		\State recorrido $\gets$ crearIT(Siguiente(recorrido).Siguiente) 		\Comment O(1)
	\ElsIf {encontrado $\wedge$ vac\'ia?(fin(kAux))} 	\Comment O(5)
		\State Siguiente(recorrido).final $\gets$ true 					\Comment O(1)		
		\State *Siguiente(recorrido).significado $\gets$ s 					\Comment O(1)
	\ElsIf {encontrado} 					\Comment O(2)
		\State recorrido $\gets$ crearIT(Siguiente(recorrido).Siguiente) 		\Comment O(1)
	\EndIf
	\State kAux $\gets$ fin(kAux) 							\Comment O(1)
\EndWhile
	
			 \medskip
			  \Statex \underline{Complejidad:} O(long(k) + copy(k))
			  \Statex \underline{Justificaci\'on:} O(long(k) + 1 + 2xcopy(k)) = O(long(k) + copy(k))
    	\end{algorithmic}
\end{algorithm}

\begin{algorithm}[H]{\textbf{iDefinido?}(\In{d}{trie}, \In {k}{k})$\to$ $res$ : bool}
	\begin{algorithmic}
	
\State kAux : k $\gets$ copy(k) 						\Comment O(copy(k))
\State recorrido : itLista(nodoTrie) $\gets$ crearIT(d.diccionario) 		\Comment O(1)
\State res $\gets$ false 						\Comment O(1)
\While{$\neg$ vac\'ia?(kAux) $\wedge$ $\neg$res $\wedge$ HaySiguiente(recorrido)} 						\Comment O(12 x long(k)) = O(long(k))
	\State encontrado : bool $\gets$ false 		\Comment O(1)
	\While{HaySiguiente(recorrido) $\wedge$ $\neg$ encontrado}          \Comment O(7 x cantidad de elementos asignados a la lista que itera recorrido) pero esto est\'a acotado por constante as\'i que es O(1)
		\If{Siguiente(recorrido) = prim(kAux)} 		\Comment O(2)
		  \State encontrado $\gets$ true 		\Comment O(1)
		\EndIf
		\State Avanzar(recorrido) 		\Comment O(1)
	\EndWhile
	\If{$\neg$ encontrado}      \Comment O(2)
		\State res $\gets$ false 					\Comment O(1)
	\ElsIf {encontrado $\wedge$ vac\'ia?(fin(kAux))} 	\Comment O(4)
		\State res $\gets$ Siguiente(recorrido).final 					\Comment O(1)		
	\ElsIf {encontrado} 					\Comment O(2)
		\State recorrido $\gets$ crearIT(Siguiente(recorrido).Siguiente) 		\Comment O(1)
	\EndIf
	\State kAux $\gets$ fin(kAux) 							\Comment O(1)
\EndWhile
	)
			 \medskip
			 \Statex \underline{Complejidad:} O(copy(k) + long(k))
			  \Statex \underline{Justificaci\'on:} O(copy(k) + long(k) + 2) = O(copy(k) + long(k))
    	\end{algorithmic}
\end{algorithm}

\begin{algorithm}[H]{\textbf{iSignificado}(\In{d}{trie}, \In {k}{k})$\to$ $res$ : $\sigma$}
	\begin{algorithmic}
	
\State kAux : k $\gets$ copy(k) 						\Comment O(copy(k))
\State resultado : puntero($\sigma$) $\gets$ NULL   				\Comment O(1)
\State recorrido : itLista(nodoTrie) $\gets$ crearIT(d.diccionario) 		\Comment O(1)
\While{$\neg$ vac\'ia?(kAux)} 					\Comment O(6 x long(k)) = O(long(k))
	\State encontrado : bool $\gets$ false 		\Comment O(1)
	\While{HaySiguiente(recorrido) $\wedge$ $\neg$ encontrado}          \Comment O(7 x cantidad de elementos asignados a la lista que itera recorrido) pero esto est\'a acotado por constante as\'i que es O(1)
		\If{Siguiente(recorrido) = prim(kAux)} 		\Comment O(2)
		  \State encontrado $\gets$ true 		\Comment O(1)
		\EndIf
		\State Avanzar(recorrido) 		\Comment O(1)
	\EndWhile
	\If {encontrado $\wedge$ vac\'ia?(fin(kAux))} 	\Comment O(3)
		\State resultado $\gets$ Siguiente(recorrido).significado 					\Comment O(1)		
	\Else  					\Comment O(2)
		\State recorrido $\gets$ crearIT(Siguiente(recorrido).Siguiente) 		\Comment O(1)
		
	\EndIf
\State kAux $\gets$ fin(kAux) 							\Comment O(1)
\EndWhile
			\State res $\gets$ *resultado  		\Comment O(1)
			 \medskip
			  \Statex \underline{Complejidad:} O(copy(k) + long(k))
			  \Statex \underline{Justificaci\'on:} O(copy(k) + long(k) + 3) = O(copy(k) + long(k))
    	\end{algorithmic}
\end{algorithm}

\begin{algorithm}[H]{\textbf{iBorrar}(\Inout{d}{trie}, \In {k}{k})}
	\begin{algorithmic}
\State Eliminar(d.claves, k)						\Comment $O(\Sigma_{a' \in d.claves} equal(k,a')))$ %Costo de eliminar en un conjunto.
\State kAux : k $\gets$ copy(k) 						\Comment O(copy(k))
\State recorrido : itLista(nodoTrie) $\gets$ crearIT(d.diccionario) 		\Comment O(1)

\While{$\neg$ vac\'ia?(kAux)} 						\Comment O(4 x long(k)) = O(long(k))
	\State encontrado : bool $\gets$ false 		\Comment O(1)
	\While{HaySiguiente(recorrido) $\wedge$ $\neg$ encontrado}          \Comment O(7 x cantidad de elementos asignados a la lista que itera recorrido) pero esto est\'a acotado por constante as\'i que es O(1)
		\If{Siguiente(recorrido) = prim(kAux)} 		\Comment O(2)
		  \State encontrado $\gets$ true 		\Comment O(1)
		\EndIf
		\State Avanzar(recorrido) 		\Comment O(1)
	\EndWhile
	\If {vac\'ia?(fin(kAux))} 	\Comment O(2)
		\State Siguiente(recorrido).final $\gets$ false 					\Comment O(1)		
	\Else  					\Comment O(2)
		\State recorrido $\gets$ crearIT(Siguiente(recorrido).Siguiente) 		\Comment O(1)
	\EndIf
	\State kAux $\gets$ fin(kAux) 							\Comment O(1)
\EndWhile	

			 \medskip
			  \Statex \underline{Complejidad:} O($\Sigma_{a' \in d.claves} equal(k,a')$ + copy(k) + long(k))
			  \Statex \underline{Justificaci\'on:} O($\Sigma_{a' \in d.claves} equal(k,a')$ + copy(k) + long(k) + 1) = O($\Sigma_{a' \in d.claves} equal(k,a')$ + copy(k) + long(k))
    	\end{algorithmic}
\end{algorithm}

\begin{algorithm}[H]{\textbf{Claves}(\In{d}{trie})$\to$ $res$ : conj(k)}
	\begin{algorithmic}
	  \State res $\gets$ d.claves							\Comment O(1)
			 \medskip
			  \Statex \underline{Complejidad:} O(1)
    	\end{algorithmic}
\end{algorithm}

\begin{algorithm}[H]{\textbf{$\#$Claves}(\In{d}{trie})$\to$ $res$ : nat}
	\begin{algorithmic}
	  \State res $\gets$ Cardinal(d.claves)						\Comment O(1)
			 \medskip
			  \Statex \underline{Complejidad:} O(1)
    	\end{algorithmic}
\end{algorithm}


\end{Algoritmos}
\textbf{Aclaraciones:}
- Varias de las complejidades incluyen copy(k) haciendo referencia al costo de copiar la clave. Como las claves con las que vamos a usar este modulo son hostname, y por ende secuencias de caracteres, el costo de copia es igual a la longitud de la clave. Entonces estas complejidades nos quedan O(long(k) + long(k)) = O(log(k)).


\section{M�dulo Avld(nat, itLista(compu))}


\begin{Interfaz}
  

  \textbf{se explica con}: {\tadNombre{Diccionario(nat, itLista(compu))}}.

  \textbf{g�neros}:{ \TipoVariable{avld(nat, itLista(compu))}}.

  \Titulo{Operaciones b�sicas de Avld}


  \InterfazFuncion{Vac�o}{}{avld} % {nombreFunci�n}{Signatura}{Res}
  {res $\igobs$ vacio()} %POST
  [O(1)] %COMPLEJIDAD
  [Crea un AVL vac�o.] %DESCRIPCI�N
  [] %ALIASING

  \InterfazFuncion{Definido?AVL}{\In {A}{avld}, \In {id}{nat}}{bool} % {nombreFunci�n}{Signatura}{Res}
  {res $\igobs$ def?(id,A)} %POST
  [O(log(n) + log(k))] %COMPLEJIDAD
  [Indica si hay un nodo en el AVL cuyo id es el nat pasado por par�metro.] %DESCRIPCI�N
  [] %ALIASING

  \InterfazFuncion{SignificadoAVL}{\In {A}{avld}, \In {id}{nat}}{itLista(compu)} % {nombreFunci�n}{Signatura}{Res}
  [def?(id,A)] %PRE
  {res $\igobs$ obtener(id,A)} %POST
  [O(log(n) + log(k))] %COMPLEJIDAD
  [Busca el nodo del AVL cuyo id es el pasado por par�metro y devuelve su camino. ] %DESCRIPCI�N
  [res es modificable si y s\'olo si A es modificable.] %ALIASING
 
  \InterfazFuncion{DefinirAVL}{\Inout {A}{avld},\In {id}{nat},\In {camino}{itLista(compu)}}{} % {nombreFunci�n}{Signatura}{Res}
  [$A \equiv A_0$] %PRE
  {$A \igobs definir(id,camino,A_0)$} %POST
  [O(log(n) + log(k))] %COMPLEJIDAD
  [Inserta un Nodo al AVL, que tiene como id y camino los pasados por par�metro. ] %DESCRIPCI�N
  [] %ALIASING
  
%		Pre{def?(id,A) && A=A'}
%		Post{A=borrar(id,A')}
  \InterfazFuncion{BorrarAVL}{\Inout {A}{avld},\In {id}{nat}}{} % {nombreFunci�n}{Signatura}{Res}
  [def?(id,A) $\wedge A \equiv A_0$] %PRE
  {$A \igobs borrar(id,A_0)$} %POST
  [O(log(n) + log(k))] %COMPLEJIDAD
  [] %DESCRIPCI�N
  [] %ALIASING

  \InterfazFuncion{ClavesAVL}{\In {A}{avld}}{conj(nat)} % {nombreFunci�n}{Signatura}{Res}
  {$A \igobs claves(A)$} %POST
  [O(n x k)] %COMPLEJIDAD
  [Devuelve un conjunto que contiene todas las claves definidas en el AVL] %DESCRIPCI�N

\textbf{Funciones auxiliares:}

  ConstruirClaves(inout p : puntero(nodo)) $\rightarrow$ res : conj(nat)
-Dado un puntero a nodo de un avl, se agrega la id de ese nodo a un conjunto vac\'io. Despu\'es se hace lo mismo con los dos hijos del nodo y se agregan los resultados al conjunto. Con esto recorremos todo el \'arbol guardando las claves de los nodos y devolviendo las mismas como resultado en un conjunto.


  FactorDeBalanceo(inout p : puntero(nodo))$\rightarrow$ res : int
-Dado un puntero a nodo de un avl, se calcula su factor de balanceo. Esta valor num\'erico se consigue restando la altura de su hijo derecho con el izquierdo.


BalancearNodo(inout p : puntero(nodo)) $\rightarrow$ res : puntero(nodo)
-Dado un puntero a nodo de un avl, se lo balancea si es necesario. B\'asicamente se trata de si el factor de balanceo del nodo es 2 o -2 realizar las operaciones pertinentes para restablecer el invariante de balanceo del avl. 

  Balancear(inout iteradorRama : itLista(puntero(nodo))
-Dado el iterador de una lista de punteros a nodo de un avl, se busca aplicar la funci\'on BalancearNodo a cada uno de los miembros de la lista. .La lista en la que se itera representa una rama del avl invertida. Exceptuando al \'ultimo que es el de raiz. Esto ultimo se encarga la funci\'on que lo llama.

CorreguirAlturas(inout iteradorRama: itLista(puntero(nodo))
-Dado el iterador de una lista de punteros a nodo de un avl, se busca recalcular las alturas de todos los miembros de la lista. La lista en la que se itera representa una rama del avl invertida. Para cada uno se toma la altura m�xima entre su hijo izquierdo y derecho y se le suma 1(Esto es porque el propio nodo tambi�n cuenta en la altura). 

BuscarElRemplazo(inout listaABalancear : lista(puntero(nodo), in aBorrar : puntero(nodo))$\rightarrow$ res : puntero(nodo)
-Cuando se elimina un nodo que tiene hijo izquierdo y derecho, esta funci\'on se encarga de buscar un candidato para remplazar el nodo eliminado. Se toma la una lista de punteros a nodos que forman la rama en la que se estuvo recorriendo para llegar al nodo que se busca quitar, solo que sin este \'ultimo. Tambi�n se utiliza un puntero al nodo que se pretende quitar. La funci\'on adem�s va guardando los nodos que recorre en la lista de punteros a nodos ya que estos se tendr�n que recalcular las alturas y balancear. Finalmente cuando se encuentra el candidato se debe devolver, no sin antes asegurarse que no se pierdan sus hijos si es que los tiene.
 


\end{Interfaz}

\begin{Representacion}
 
  \Titulo{Representaci�n de AVL}
%Tengo que usar esta estructura porque sino tengo que hacer un iterador para moverme por el avl!
  \begin{Estructura}{avld}[avl]
    \begin{Tupla}[avl]
    	\tupItem{raiz}{puntero(nodo)}
    \end{Tupla}
		
    \begin{Tupla}[nodo]
	\tupItem{id}{nat}
	\tupItem{camino}{itLista(compu)}
	\tupItem{izq}{puntero(nodo)}
	\tupItem{der}{puntero(nodo)}
	\tupItem{altura}{nat} 
    \end{Tupla}	
  \end{Estructura}

	\textbf{Invariante de representaci�n en castellano: }
	-El puntero raiz es NULL\\
	-Si el puntero raiz no es NULL entonces la altura es 1 sii izq y der son NULL. Adem\'as es 2 sii o izq o der son NULL pero no ambos y adem\'as la suma entre la altura del hijo distinto de NULL debe ser 1. Y es 3 o mayor sii la suma de las alturas de izq y der mas 1 es de como resultado la altura del nodo padre.\\ 
	-La resta de la altura del hijo derecho con el hijo izquierdo, si ninguno es vac\'io, de cualquier nodo no puede ser mayor en modulo a 1.\\
	-Para todo nodo el hijo izquierdo tiene un id menor y el derecho mayor, si es que tiene alguno.\\
	-Al ser un \'arbol y usar punteros tenemos que pedir que cada uno de los nodos no tenga punteros derecho o izquierdo a nodos que sean apuntados por otro. Osea que no de un bucle.\\
	-Tanto el hijo izquierdo y derecho, si los hay, cumplen estas condiciones.\\
	
	
  \Abs[avl]{Avld(nat, itLista(compu))}[a]{e}{($\forall$ n : nat)(def?(n, e) $\Leftrightarrow$ ExisteNodoConClave(*a.raiz, n) $\yluego$ (def?(n, e) $\impluego$  obtener(n, e) = ObtenerAVL(*a.raiz, n)))}
  
  \tadOperacion{ExisteNodoConClave}{nodo , nat}{bool}{}
  \tadAxioma{ExisteNodoConClave(nodo, n)}{\IF nodo.id = n THEN true ELSE {\IF nodo.id < n  THEN ExisteNodoConClave(*nodo.izq, n) ELSE ExisteNodoConClave(*nodo.der, n) FI} FI}
  
  \tadOperacion{ObtenerAVL}{nodo , nat}{itLista(compu)}{}
  \tadAxioma{ObtenerAVL(nodo, n)}{\IF nodo.id = n THEN nodo.camino ELSE {\IF nodo.id < n  THEN ObtenerAVL(*nodo.izq, n) ELSE ObtenerAVL(*nodo.der, n) FI} FI}
  
  
\end{Representacion}	



\begin{Algoritmos}

%iVac�o:

\begin{algorithm}[H]{\textbf{iVac�o}() $\to$ $res$ : $avl$} 
	\begin{algorithmic}
			\State res $\gets$ <NULL> \Comment O(1)
			\medskip
			\Statex \underline{Complejidad:} O(1)
			\Statex \underline{Justificaci�n:} 
    	\end{algorithmic}
\end{algorithm}

\begin{algorithm}[H]{\textbf{iDefinido?AVL}(\In {A}{avl}, \In {i}{nat}) $\to$ $res$ : $bool$} 
	\begin{algorithmic}
			\State recorrido : puntero(nodo) $\gets$ A.raiz		\Comment O(1)
			 \State res $\gets$ false 				\Comment O(1)
			 \While{$\neg$res $\wedge$ recorrido $\neq$ NULL} 						\Comment O(2 x log(n)) = O(log(n))
			    \If{id $=$ *recorrido.id} 	 \Comment O(2)
			      \State res $\gets$ true    \Comment O(1)
			    \ElsIf{id < *recorrido.id} 			\Comment O(2)
			      \State recorrido $\gets$ *recorrido.izq 	\Comment O(1)
			    \Else 					\Comment O(2)
			      \State recorrido $\gets$ *recorrido.der 	\Comment O(1)
			    \EndIf
			 \EndWhile

			\medskip
			\Statex \underline{Complejidad:} O(log(n))
			\Statex \underline{Justificaci�n:} O(log(n) + 2) = O(log(n))
    	\end{algorithmic}
\end{algorithm}

\begin{algorithm}[H]{\textbf{SignificadoAVL}(\In {A}{avl}, \In {id}{nat}) $\to$ $res$ : itLista(compu)} 
	\begin{algorithmic}
			 \State recorrido : puntero(nodo) $\gets$ A.raiz		\Comment O(1)
			 \State llegue : bool $\gets$ false 				\Comment O(1)
			 \While{$\neg$llegue} 						\Comment O(2 x log(n)) = O(log(n))
			    \If{id $=$ *recorrido.id} 	 \Comment O(2)
			      \State llegue $\gets$ true    \Comment O(1)
			    \ElsIf{id < *recorrido.id} 			\Comment O(2)
			      \State recorrido $\gets$ *recorrido.izq 	\Comment O(1)
			    \Else 					\Comment O(2)
			      \State recorrido $\gets$ *recorrido.der 	\Comment O(1)
			    \EndIf
			 \EndWhile
			 \State res $\gets$ *recorrido.camino 		\Comment O(1)
			\medskip
			\Statex \underline{Complejidad:} O(log(n))
			\Statex \underline{Justificaci�n:} O(3 + log(n)) = O(log(n))
    	\end{algorithmic}
\end{algorithm}

\begin{algorithm}[H]{\textbf{DefinirAVL}(\Inout {A}{avl},\In {id}{nat},\In {camino}{itLista(compu)})} 
	\begin{algorithmic}
			
			\If{A.raiz $=$ NULL} 						\Comment O(2)
			  \State *A.raiz $\gets$ $\delta$<id, camino, NULL, NULL, 1>  		\Comment O(1)
			\Else   \Comment O(3 x log(n) + 4) = O(log(k))
			  \State recorrido : puntero(nodo) $\gets$ A.raiz			\Comment O(1)
			  \State llegue : bool $\gets$ false 					\Comment O(1)
			  \State necesitaBalancear : bool $\gets$ true 				\Comment O(1)
			  \State listaABalancear : lista(puntero(nodo)) $\gets$ Vac\'ia()  	\Comment O(1)
			  \While{$\neg$llegue} 						\Comment O(5 x log(n)) = O(log(n))
			      \State AgregarAdelante(listaABalancear, recorrido)   	\Comment O(copy(recorrido)) = O(1)
			      \If{id $=$ *recorrido.id} 				\Comment O(4)
				\State recorrido $\gets$ $\delta$<id, camino, *recorrido.izq, *recorrido.der, *recorrido.altura> 	\Comment O(1)
				\State llegue $\gets$ true 				\Comment O(1)
				\State necesitaBalancear $\gets$ false 			\Comment O(1)
			      \ElsIf{id < *recorrido.id} 				\Comment O(4)
				\If{*recorrido.izq $=$ NULL} 				\Comment O(3)
				  \State llegue $\gets$ true 				\Comment O(1)
				  \State *recorrido.izq $\gets$ $\delta$<id,camino,NULL,NULL,1>  \Comment O(1)
				\Else 							\Comment O(2)
				  \State recorrido $\gets$ *recorrido.izq 		\Comment O(1)
				\EndIf
			      \Else 							\Comment O(4)
				\If{*recorrido.der $=$ NULL} 				\Comment O(3)
				  \State llegue $\gets$ true 				\Comment O(1)
				  \State *recorrido.der $\gets$ $\delta$<id,camino,NULL,NULL,1>  \Comment O(1)
				\Else 							\Comment O(2)
				  \State recorrido $\gets$ *recorrido.der 	\Comment O(1)
				\EndIf
			      \EndIf
			  \EndWhile
			  \If{necesitaBalancear} 		\Comment O(log(n) + 3) = O(log(n))
			    \State iteradorRama : itLista(puntero(nodo)) $\gets$ crearIT(listaABalancear) 	\Comment O(1)
			    \State CorreguirAlturas(iteradorRama) 			\Comment O(log(n))
			    \State iteradorRama2 : itLista(puntero(nodo)) $\gets$ crearIT(listaABalancear) 	\Comment O(1)
			    \State Balancear(iteradorRama2)			\Comment O(log(n))
			    \State A.raiz $\gets$ BalancearNodo(Anterior(iteradorRama2)) 	\Comment O(1)
			  \EndIf
			\EndIf
			\medskip
			\Statex \underline{Complejidad:} O(log(n))
			\Statex \underline{Justificaci�n:} O(2 x log(n) + 4) = O(log(n))
    	\end{algorithmic}
\end{algorithm}



\begin{algorithm}[H]{\textbf{BorrarAVL}(\Inout {A}{avl},\In {id}{nat})} 
	\begin{algorithmic}
			\State listaABalancear : lista(nodo) $\gets$ Vac\'ia() 		\Comment O(1)
			\If{*A.raiz.id = id} 		\Comment O(log(n))
			  \If{*A.raiz.izq = NULL $\wedge$ *A.raiz.der = NULL} 	\Comment O(4)
			    \State A.raiz $\gets$ NULL 		\Comment O(1)
			  \ElsIf{*A.raiz.izq = NULL} 		\Comment O(2)
			    \State A.raiz $\gets$ *A.raiz.der 		\Comment O(1)
			  \ElsIf{*A.raiz.der = NULL} 		\Comment O(2)
			    \State A.raiz $\gets$ *A.raiz.izq	   \Comment O(1)  
			  \Else 		\Comment O(3 x log(n) + 8) = O(log(n))
			    \State remplazo : puntero(compu) $\gets$ BuscarElRemplazo(listaABalancear, A.raiz)  \Comment O(log(n))
			    \State *remplazo.der $\gets$ *A.raiz.der 	\Comment O(1)
			    \If{$\neg$Longitud(listaABalancear) = 1} 	\Comment O(2)
			      \State *remplazo.izq $\gets$ *A.raiz.izq 	\Comment O(1)
			    \EndIf
			    \State A.raiz $\gets$ remplazo 		\Comment O(1)
			    \State iteradorRama : itLista(puntero(nodo)) $\gets$ crearIT(listaABalancear) 	\Comment O(1)
			    \State CorreguirAlturas(iteradorRama) 			\Comment O(log(n))
			    \State iteradorRama2 : itLista(puntero(nodo)) $\gets$ crearIT(listaABalancear) 	\Comment O(1)
			    \State Balancear(iteradorRama2)			\Comment O(log(n))
			    \State A.raiz $\gets$ BalancearNodo(Anterior(iteradorRama2)) 	\Comment O(1)
			  \EndIf
			 \Else 				\Comment O(2 x log(n) + 2) = O(log(n))
			  \State recorrido : puntero(nodo) $\gets$ A.raiz 			\Comment O(1)
			  \State llegue : bool $\gets$ false 				\Comment O(1)
			  \While{$\neg$llegue} 						 \Comment O(5 x log(n)) = O(log(n))
				\State AgregarAdelante(listaABalancear, recorrido)   	\Comment O(copy(recorrido)) = O(1)
				\If{*recorrido.id $=$ id}  			\Comment O(2)
				  \State llegue $\gets$ true 			\Comment O(1)
				\ElsIf{*recorrido.id $<$ id} 			\Comment O(2)
				  \State recorrido $\gets$ *recorrido.izq 	\Comment O(1)
				\ElsIf{*recorrido.id $>$ id} 			\Comment O(2)
				  \State recorrido $\gets$ *recorrido.der 	\Comment O(1)
				\EndIf
			  \EndWhile
			  \If{*recorrido.izq = NULL $\wedge$ *recorrido.der = NULL} 	\Comment O(5)
			      \If{*listaABalancear[2].izq.id = *recorrido.id} 	 	\Comment O(2)
				\State *listaABalancear[2].izq $\gets$ NULL 		\Comment O(1)
			      \Else 							\Comment O(2)
				\State *listaABalancear[2].der $\gets$ NULL 		\Comment O(1)
			      \EndIf
			  \ElsIf{*recorrido.izq = NULL} 				\Comment O(3)
			      \If{*listaABalancear[2].izq.id = *recorrido.id} 		\Comment O(2)
				\State *listaABalancear[2].izq $\gets$ *recorrido.der 	\Comment O(1)
			      \Else 							\Comment O(2)
				\State *listaABalancear[2].der $\gets$ *recorrido.der 	\Comment O(1)
			      \EndIf
			  \ElsIf{*recorrido.der = NULL} 					\Comment O(3)
			      \If{*listaABalancear[2].izq.id = *recorrido.id} 			\Comment O(2)
				\State *listaABalancear[2].izq $\gets$ *recorrido.izq 		\Comment O(1)
			      \Else 								\Comment O(2)
				\State *listaABalancear[2].der $\gets$ *recorrido.izq 		\Comment O(1)
			      \EndIf	    
			  \Else 				\Comment O(3 x log(n) + 6) = O(log(n))
			      \State Aux $\gets$ listaABalancear[2] 		\Comment O(1)
			      \State remplazo $\gets$ BuscarElRemplazo(listaABalancear, recorrido) 	\Comment O(log(n))
			      \If{*Aux.izq.id = *recorrido.id} 						\Comment O(2)
				\State *Aux.izq $\gets$ remplazo 					\Comment O(1)
			      \Else 									\Comment O(2)
				\State *Aux.der $\gets$ remplazo 					\Comment O(1)
			      \EndIf
			      \State iteradorRama : itLista(puntero(nodo)) $\gets$ crearIT(listaABalancear) 	\Comment O(1)
			      \State CorreguirAlturas(iteradorRama) 						\Comment O(log(n))
			      \State iteradorRama2 : itLista(puntero(nodo)) $\gets$ crearIT(listaABalancear) 	\Comment O(1)
			      \State Balancear(iteradorRama2)							\Comment O(log(n))
			      \State A.raiz $\gets$ BalancearNodo(Anterior(iteradorRama2)) 			\Comment O(1)
			  \EndIf
			  
			\EndIf
			\medskip
			\Statex \underline{Complejidad:} O(log(n))
			\Statex \underline{Justificaci�n:} O(log(n) + 1) = O(log(n))
    	\end{algorithmic}
\end{algorithm}


\begin{algorithm}[H]{\textbf{ClavesAVL}(\In {A}{avl}) $\to$ $res$ : conj(nat)} 
	\begin{algorithmic}
			\State res $\gets$ ConstruirClaves(A.raiz) \Comment O(n)
			\medskip
			\Statex \underline{Complejidad:} O(n)

    	\end{algorithmic}
\end{algorithm}

\begin{algorithm}[H]{\textbf{iConstruirClaves}(\In {p}{puntero(nodo)}) $\to$ $res$ : conj(nat)} 
	\begin{algorithmic}
			\State res $\gets$ Vac\' io() 	\Comment O(1)
			\If{p $=$ NULL} 	\Comment O(3 + Cardinal(resIzq) + Cardinal(resDer) + ($2^{*(*p.izq.altura)} - 1) + (2^{*(*p.izq.altura)} - 1)$)
			  \State AgregarRapido(res, *p.id)  \Comment O(1)
			  \State resIzq : conj(nat) $\gets$ ConstruirClaves(*p.izq) 	\Comment O($(2^{*(*p.izq.altura)} - 1)$)
			  \State resDer : conj(nat) $\gets$ ConstruirClaves(*p.der) 	\Comment O($(2^{*(*p.izq.altura)} - 1)$)
			  \State iteradorIzq : itConj(compu) $\gets$ crearIT(resIzq) 	\Comment O(1)
			  \State iteradorDer : itConj(compu) $\gets$ crearIT(resDer) 	\Comment O(1)
			  
			  \While{HaySiguiente(iteradorIzq)} 	\Comment O(3 x Cardinal(resIzq)) = O(Cardinal(resIzq))
			    \State AgregarRapido(res, *Siguiente(iteradorIzq).id) 	\Comment O(copy(*Siguiente(iteradorIzq).id)) = O(1)
			    \State Avanzar(iteradorIzq) 	\Comment O(1)
			  \EndWhile
			  
			  \While{HaySiguiente(iteradorDer)} \Comment O(3 x Cardinal(resDer)) = O(Cardinal(resDer))
			    \State AgregarRapido(res, *Siguiente(iteradorDer).id) \Comment O(copy(*Siguiente(iteradorDer).id)) = O(1)
			    \State Avanzar(iteradorDer) 	\Comment O(1)
			  \EndWhile
			\EndIf
			\medskip
			\Statex \underline{Complejidad:} O(n)
			\Statex \underline{Justificaci�n:} O(4 + Cardinal(resIzq) + Cardinal(resDer) + $(2^{*(*p.izq.altura)} - 1)$ + $2^{*(*p.izq.altura))}$ = O(6 + 2 x n) = O(n). $(2^{*(*p.izq.altura)}- 1)$ es una cota superior para la cantidad de elementos del sub\'arbol izquierdo, idem derecho. Ya que la suma de los dos cardinales representan a la totalidad de elemntos del diccionario menos uno. AgregarRapido se puede usar porque la id no se repite nunca en la recurci\'on(es un \'arbol binario).
    	\end{algorithmic}
\end{algorithm}


\begin{algorithm}[H]{\textbf{iFactorDeBalanceo}(\In {p}{puntero(nodo)}) $\to$ $res$ : int} 
	\begin{algorithmic}
			\State res $\gets$ 0 	   \Comment O(1)	 
			\If{*p.izq $\neq$ NULL} 	\Comment O(2)
			  \State res $\gets$ res - *p.izq.altura 	\Comment O(1)
			\EndIf
			\If{*p.der $\neq$ NULL} 	\Comment O(2)
			  \State res $\gets$ res + *p.der.altura 	\Comment O(1)
			\EndIf
			\medskip
			\Statex \underline{Complejidad:} O(1)
			\Statex \underline{Justificaci�n:} O(2 + 2 + 1) = O(5) = O(1)
    	\end{algorithmic}
\end{algorithm}


\begin{algorithm}[H]{\textbf{iBalancearNodo}(\Inout {p}{puntero(nodo)})$\to$ $res$ : puntero(nodo)} 
	\begin{algorithmic}
			\If{FactorDeBalanceo(p) $=$ 2} 			\Comment O(12)
				\State P : puntero(nodo) $\gets$ p 	 \Comment O(1)
				\State Q : puntero(nodo) $\gets$ *P.izq  \Comment O(1)
				
				\If{FactorDeBalanceo(Q) $=$ 1} 	\Comment O(4)
				  \State *P.der $\gets$ *Q.izq 		\Comment O(1)
				  \State *Q.izq $\gets$ P 		\Comment O(1)
				  \State *P.altura $\gets$ *P.altura - 2 	\Comment O(1)
				  \State res $\gets$ Q 				\Comment O(1)
				\Else 						\Comment O(10)
				  \State R : puntero(nodo) $\gets$ *Q.izq 	\Comment O(1)
				  \State *P.der $\gets$ *R.izq 			\Comment O(1)
				  \State *Q.izq $\gets$ *R.der 			\Comment O(1)
				  \State *R.izq $\gets$ P  			\Comment O(1)
				  \State *R.der $\gets$ Q 			\Comment O(1)
				  \State *P.altura $\gets$ *P.altura - 2 	\Comment O(1)
				  \State *Q.altura $\gets$ *Q.altura - 1 	 \Comment O(1)
				  \State *R.altura $\gets$ *R.altura + 1   	 \Comment O(1)
				  \State res $\gets$ R 				\Comment O(1)
				\EndIf
			    
			    \ElsIf{FactorDeBalanceo(p) $=$ -2} 			\Comment O(12)
			      \State P : puntero(nodo) $\gets$ p 		\Comment O(1)
			      \State Q : puntero(nodo) $\gets$ *P.der  		\Comment O(1)
			       
			      \If{FactorDeBalanceo(Q) $=$ -1} 		\Comment O(5)
				  \State *P.izq $\gets$ *Q.der 			\Comment O(1)
				  \State *Q.der $\gets$ P 			\Comment O(1)
				  \State *P.altura $\gets$ *P.altura - 2 	\Comment O(1)
				  \State res $\gets$ Q 				\Comment O(1)
			      \Else 						\Comment O(10)
				  \State R : puntero(nodo) $\gets$ *Q.der 	\Comment O(1)
				  \State *P.izq $\gets$ *R.izq 	\Comment O(1)
				  \State *Q.der $\gets$ *R.der 	\Comment O(1)
				  \State *R.izq $\gets$ P 	\Comment O(1)
				  \State *R.der $\gets$ Q 	\Comment O(1)
				  \State *P.altura $\gets$ *P.altura - 2 	\Comment O(1)
				  \State *Q.altura $\gets$ *Q.altura - 1 	\Comment O(1)
				  \State *R.altura $\gets$ *R.altura + 1 	\Comment O(1)
				  \State res $\gets$ R 				\Comment O(1)
			      \EndIf 
			    
			    \EndIf
			\medskip
			\Statex \underline{Complejidad:} O(1)
			\Statex \underline{Justificaci�n:} O(12) = O(1)
    	\end{algorithmic}
\end{algorithm}

\begin{algorithm}[H]{\textbf{iCorreguirAlturas}(\Inout {iteradorRama}{itLista(puntero(nodo)})} 
	\begin{algorithmic}
			\While{HaySiguiente(iteradorRama)} 							\Comment O(8 x log(n))
			      \State auxAltura : nat $\gets$ 1 							\Comment O(1)
			      \If{*Siguiente(iteradorRama).izq $\neq$ NULL  $\wedge$ *Siguiente(iteradorRama).der $\neq$ NULL} 		\Comment O(5)
				\If{*Siguiente(iteradorRama).der.altura < *Siguiente(iteradorRama).izq} 	\Comment O(2)
				  \State auxAltura $\gets$ auxAltura + *Siguiente(iteradorRama).izq.altura 	\Comment O(1)
				\Else 										\Comment O(2)
				  \State auxAltura $\gets$ auxAltura + *Siguiente(iteradorRama).der.altura 	\Comment O(1)
				\EndIf
			      \ElsIf{*Siguiente(iteradorRama).izq $\neq$ NULL} 					\Comment O(2)
				\State auxAltura $\gets$ auxAltura + *Siguiente(iteradorRama).izq.altura 	\Comment O(1)
			      \ElsIf{*Siguiente(iteradorRama).der $\neq$ NULL} 					\Comment O(2)
				\State auxAltura $\gets$ auxAltura + *Siguiente(iteradorRama).der.altura 	\Comment O(1)
			      \EndIf
			      \State *Siguiente(iteradorRama).altura $\gets$ auxAltura  			\Comment O(1)
			      \State Avanzar(iteradorRama) 	     						\Comment O(1)
			\EndWhile
			\medskip
			\Statex \underline{Complejidad:} O(log(n))
			\Statex \underline{Justificaci�n:} O(8 x log(n)) = O(log(n)). El iteradorRama recorre una rama del \'arbol y esta mide log(n) siendo n la cantidad de nodos del avl.
    	\end{algorithmic}
\end{algorithm}

\begin{algorithm}[H]{\textbf{iBalancear}(\Inout {iteradorRama}{itLista(puntero(nodo))})} 
	\begin{algorithmic}
			\While{HaySiguiente(iteradorRama)} 							\Comment O(4 x log(n))
			      \State Avanzar(iteradorRama) 	     				\Comment O(1)
			      \If{HaySiguiente(iteradorRama)} 	\Comment O(3)
				  \If{Anterior(iteradorRama) $=$ *Siguiente(iteradorRama).izq}   	\Comment O(2)
				    \State *Siguiente(iteradorRama).izq $\gets$ BalancearNodo(Anterior(iteradorRama)) 	\Comment O(1)
				  \Else 		\Comment O(2)
				    \State *Siguiente(iteradorRama).der $\gets$ BalancearNodo(Anterior(iteradorRama)) 	\Comment O(1)
				  \EndIf
			      \EndIf
			    \EndWhile
			\medskip
			\Statex \underline{Complejidad:} O(log(n))
			\Statex \underline{Justificaci�n:} O(4 x log(n)) = O(log(n))
    	\end{algorithmic}
\end{algorithm}

\begin{algorithm}[H]{\textbf{iBuscarElRemplazo}(\Inout {listaABalancear}{lista(puntero(nodo))}, \In aBorrar{puntero(nodo)})$\to$ $res$ : puntero(nodo)} 
	\begin{algorithmic}

			 \State recorrido : puntero(nodo) $\gets$ *aBorrar.izq 		\Comment O(1)
			 \State llegue $\gets$ false 					\Comment O(1)
			 \If{*recorrido.der $\neq$ NULL} 				\Comment O(8 + log(n)) = O(log(n))
			  \State i : nat $\gets$ 1 					\Comment O(1)
			  \While{$\neg$llegue} 						\Comment O(6 x log(n)) = O(log(n))
				\State AgregarAdelante(listaABalancear, recorrido)   	\Comment O(copy(recorrido)) = O(1). Porque es un puntero.
				\If{*recorrido.der = NULL} 				\Comment O(4)
				  \State listaABalancear[2].der $\gets$ *recorrido.izq 	\Comment O(1)
				  \State listaABalancear[i] $\gets$ recorrido 		\Comment O(1)
				  \State llegue $\gets$ true 				\Comment O(1)
				\Else 							\Comment O(2)
				  \State recorrido $\gets$ *recorrido.der	        \Comment O(1)
				\EndIf
				\State i $\gets$ i + 1 					\Comment O(1)
			  \EndWhile
			\Else
			  \State AgregarAdelante(listaABalancear, recorrido)   	\Comment O(copy(recorrido)) = O(1)
			\EndIf
			\State res $\gets$ recorrido    \Comment O(1)

			\medskip
			\Statex \underline{Complejidad:} O(log(n))
			\Statex \underline{Complejidad:} O(log(n) + 11) = O(log(n))
    	\end{algorithmic}
\end{algorithm}


\end{Algoritmos}




\section{M�dulo colaPaquetes}


\begin{Interfaz}


  \textbf{se explica con}: \tadNombre{Cola de Prioridad(paquete)}.

  \textbf{g�neros}: \TipoVariable{colaPaquetes}.

  \Titulo{Operaciones b�sicas de colaPaquetes}

  \InterfazFuncion{Vac\'ia}{}{colaPaquetes}
  {$res \igobs $vac\'ia}
  [$\Theta(1)$]
  [genera un colaPaquetes vac\'ia.]

  \InterfazFuncion{Encolar}{\Inout{c}{colaPaquetes(paquete)}, \In{a}{paquete}}{}
  [$c \igobs c_0$]
  {$c \igobs encolar(c_0, a)$}
  [O(log(n))]
  [pone en la posici\'on adecuada al orden al paquete a en la colaPaquetes c.]
  
  \InterfazFuncion{Vac\'ia?}{\In{c}{colaPaquetes(paquete)}}{bool}
  {$res \igobs $vac\'ia?(c)}
  [$\Theta(1)$]
  [devuelve true si y s\'olo la colaPaquetes est\'a vac\'ia o, lo que es lo mismo, que no posee paquetes.]

  \InterfazFuncion{Pr\'oximo}{\In{c}{colaPaquetes(paquete)}}{paquete}
  [$\neg $vac\'ia?(c)]
  {$alias(res \igobs$ pr\'oximo(c))}
  [$\Theta(1)$]
  [devuelve el pr\'oximo paquete de la colaPaquetes. Este es el de mayor prioridad]
  [res es modificable si y s\'olo si d es modificable.]

  \InterfazFuncion{Desencolar}{\Inout{c}{colaPaquetes(paquete)}}{}
  [$c \igobs c_0 \wedge \neg$vac\'ia?(c)]
  {$c \igobs desencolar(c_0)$}
  [O(log(n))]
  [modifica la colaPaquetes quitando el pr\'oximo paquete y despu\'es reordenandola.]

 
\end{Interfaz}

\begin{Representacion}
  %Tiene que ser un heap minimo
  \Titulo{Representaci�n de la colaPaquetes}

  \begin{Estructura}{colaPaquetes(paquete)}[heap]
    \begin{Tupla}[heap]
      \tupItem{raiz}{puntero(paquete)}%
      \tupItem{izq}{puntero(heap)}%
      \tupItem{der}{puntero(heap)}%
       \tupItem{cantidadElementos}{nat}%
    \end{Tupla}


  \end{Estructura}


  \Rep[heap][c]{(c.cantidadElementos $=$ 0 $\Leftrightarrow$ (c.raiz $=$ NULL $\wedge$ c.izq $=$ NULL $\wedge$ c.der $=$ NULL)) $\wedge$ (c.cantidadElementos $=$ 1 $\Leftrightarrow$ (c.raiz $\neq$ NULL $\wedge$ c.izq $=$ NULL $\wedge$ c.der $=$ NULL)) $\wedge$ (c.cantidadElementos $=$ 2 $\Leftrightarrow$ (c.raiz $\neq$ NULL $\wedge$ c.izq $\neq$ NULL $\wedge$ c.der $=$ NULL)) $\wedge$ (c.cantidadElementos $>=$ 3 $\Leftrightarrow$ (c.raiz $\neq$ NULL $\wedge$ c.izq $\neq$ NULL $\wedge$ c.der $\neq$ NULL)) $\yluego$ (c.cantidadElementos $>=$ 3 $\Rightarrow$ c.cantidadElementos $=$ *c.izq.cantidadElementos + *c.der.cantidadElementos+1) $\wedge$ (c.izq $\neq$ NULL $\Rightarrow$ (*c.raiz.prioridad <= *c.izq.raiz.prioridad $\wedge$ Rep(*c.izq))) $\wedge$ (c.der $\neq$ NULL $\Rightarrow$ (*c.raiz.prioridad <= *c.der.raiz.prioridad $\wedge$ Rep(*c.der)))\\
  -Como se utilizan punteros tenemos que decir que no se permite que dos heap en un \'arbol tengan como punteros izq o der a otros iguales. En otras palabras que no se formen bucles en el \'arbol}
  
  
  \AbsFc[heap]{colaPaquetes(paquete)}[c]{e}{
      -vac\'ia?(e) $\Leftrightarrow$ c.raiz $=$ NULL $\yluego$ (vac\'ia?(e) $\impluego$  pr\'oximo(e) $=$  *c.raiz $\wedge$ desencolar(e) = funcionParaDesencolar(c)) \\
      -funcionParaDesencolar consiste en tomar el valor de el \'ultimo elemento del heap y colocarlo en la posici\'on del primero(esto es suponiendo que exista alguno a parte del primero, en caso contrario la funci\'on terminar\'ia aqui). Con esto nos desasemos del elemento que teniamos que desencolar. Ahora para que restablesca el invariante de heap tenemos que tomar el primer elemnto, que es el que agregamos, e ir bajandolo por izquieda o derecha hasta que encontremos un lugar donde no tenga hijos o los que tenga sean menores en prioridad a �l.\\
  }
 

\end{Representacion}

\begin{Algoritmos}



\begin{algorithm}[H]{\textbf{iVac\'ia}() $\to$ $res$ : heap}
	\begin{algorithmic}
			\State $res \gets$ <NULL, NULL, NULL, 0>   			\Comment O(1)
			 \medskip
			  \Statex \underline{Complejidad:} $\Theta(1)$
	
    	\end{algorithmic}
\end{algorithm}

\begin{algorithm}[H]{\textbf{iEncolar}(\Inout{c}{heap}, \In {p}{paquete})}
	\begin{algorithmic}
		\State *aPoner : puntero(paquete) $\gets$ $\delta$p 		\Comment O(1)
		\If{c.raiz $=$ NULL} 				\Comment O(3)
		  \State c.raiz $\gets$ aPoner 			\Comment O(1)
		  \State c.cantidadElementos $\gets$ 1 		\Comment O(1)
		\Else 			\Comment O(2 + log(n)) = O(log(n))
			
			\State *recorrido : puntero(heap) $\gets$ $\delta$c 		\Comment O(1)
			\State llegoAPosicion : bool $\gets$ false 		\Comment O(1)
			\While{$\neg$ llegoAPosicion} 					\Comment O(12 x log(n)) = O(log(n))
			    \State *recorrido.cantidadElementos $\gets$ *recorrido.cantidadElementos + 1
			    \If{recorrido.izq $=$ NULL} 					\Comment O(3)
			      \State llegoAPosicion $\gets$ true			     	\Comment O(1)
			      \State *nuevaRaiz : puntero(paquete) $\gets$ aPoner 				\Comment O(1)
			      
			      \If{**recorrido.raiz.prioridad > *aPoner.prioridad}        \Comment O(5)
				 \State *aux : puntero(paquete) $\gets$ *recorrido.raiz 			\Comment O(1)
				 \State *nuevoHeap : puntero(heap) $\gets$ $\delta$<aux,NULL,NULL,1> 		\Comment O(1)
				 \State *recorrido.izq $\gets$ nuevoHeap 		\Comment O(1)
				 \State *recorrido.raiz $\gets$ nuevaRaiz 		\Comment O(1)
			      \Else		   					\Comment O(3)
				 \State *nuevoHeap : puntero(heap) $\gets$ $\delta$<nuevaRaiz,NULL,NULL,1> 	\Comment O(1)
				 \State *recorrido.izq $\gets$ nuevoHeap 		\Comment O(1)
			      \EndIf
			      
			    \ElsIf{recorrido.der $=$ NULL} 					\Comment O(10)
			      \State llegoAPosicion $\gets$ true			     	\Comment O(1)
			      \State *nuevaRaiz : puntero(paquete) $\gets$ *aPoner 				\Comment O(1)
			      \If{**recorrido.raiz.prioridad > *aPoner.prioridad} 		\Comment O(8)
				 \State *aux1 : puntero(paquete)$\gets$ *recorrido.raiz 				\Comment O(1)
				 \State *aux2 : puntero(paquete) $\gets$ *recorrido.izq.raiz 			\Comment O(1)
				 \State *nuevoHeap1 : puntero(heap) $\gets$ $\delta$<aux1,NULL,NULL,1> 			\Comment O(1)
				 \State *nuevoHeap2 : puntero(heap) $\gets$ $\delta$<aux2,NULL,NULL,1> 			\Comment O(1)
				 \State *recorrido.izq $\gets$ nuevoHeap1 			\Comment O(1)
				 \State *recorrido.der $\gets$ nuevoHeap2 			\Comment O(1)
				 \State *recorrido.raiz $\gets$ nuevaRaiz 			\Comment O(1)

			      \Else		  	 					\Comment O(2)
				 \State *nuevoHeap  : puntero(heap)$\gets$ $\delta$<nuevaRaiz,NULL,NULL,1> 		\Comment O(1)
				 \State *recorrido.der $\gets$ nuevoHeap			\Comment O(1)
			      \EndIf
			      
			    \Else \Comment O(12)
			      \State nivelCompleto  : bool $\gets$ (2 x *recorrido.der.cantidadElementos) + 1 => *recorrido.izq.cantidadElementos \Comment O(1)
			      \If{**recorrido.raiz.prioridad > *aPoner.prioridad} 		\Comment O(11)
				\State *nuevaRaiz  : puntero(paquete)$\gets$ *aPoner 		\Comment O(1)
				\State *aux1  : puntero(paquete)$\gets$ *recorrido.raiz 		\Comment O(1)
				\State *aux2  : puntero(paquete)$\gets$ *recorrido.izq.raiz 	\Comment O(1)
				\State *aux3  : puntero(paquete)$\gets$ *recorrido.der.raiz 	\Comment O(1)
				\State *recorrido.raiz $\gets$ nuevaRaiz 	\Comment O(1)
				\State *recorrido.izq.raiz $\gets$ aux1 	\Comment O(1)
				\State *recorrido.der.raiz $\gets$ aux2 	\Comment O(1)
				\State aPoner $\gets$ aux3 			\Comment O(1)
			      \EndIf
			      
			      \If{nivelCompleto} 		\Comment O(2)
				  \State recorrido $\gets$ *recorrido.izq 	\Comment O(1)
			      \Else 				\Comment O(2)
				  \State recorrido $\gets$ *recorrido.der 	\Comment O(1)
			      \EndIf 
			       
			       
			    \EndIf
			  \EndWhile
			\EndIf
			 \medskip
			  \Statex \underline{Complejidad:} O(log(n))
			  \Statex \underline{Justificaci\'on:} O(log(n) + 1) = O(log(n)). Tomamos en consideraci\'on el camino m�s costoso del if.
    	\end{algorithmic}
\end{algorithm}

\begin{algorithm}[H]{\textbf{Vac\'ia?}(\In{c}{heap})$\to$ $res$ : bool}
	\begin{algorithmic}
	
			\State $res \gets$ c.cantidadElementos $=$ 0    			\Comment O(1)
			 \medskip
			  \Statex \underline{Complejidad:} O(1)
    	\end{algorithmic}
\end{algorithm}

\begin{algorithm}[H]{\textbf{iPr\'oximo}(\In{c}{heap})$\to$ $res$ : paquete}
	\begin{algorithmic}
			\State $res \gets$ *c.raiz    			\Comment O(1)
			 \medskip
			  \Statex \underline{Complejidad:} O(1)
    	\end{algorithmic}
\end{algorithm}

\begin{algorithm}[H]{\textbf{iDesencolar}(\Inout{c}{heap})}
	\begin{algorithmic}
	
		\If{c.izq $=$ NULL} 					\Comment O(3)
		  \State c.raiz $\gets$ NULL 				\Comment O(1)
		  \State c.cantidadElementos $\gets$ 0 			\Comment O(1)
		\Else 				\Comment O(2 + 2 x log(n)) = O(2 x log(n)) = O(log(n))
			  \State *recorrido  : puntero(heap)$\gets$ $\delta$c 					\Comment O(1)
			  \State final  : bool $\gets$ false 					\Comment O(1)
			  \While{$\neg$ final} 						\Comment O(4 x log(n)) = O(log(n))
			    \If{*recorrido.izq $=$ NULL} 				\Comment O(2)
			      \State final $\gets$ true 				\Comment O(1)
			    \ElsIf{*recorrido.der $=$ NULL} 						\Comment O(4)
			      \State final $\gets$ true 						\Comment O(1)
			      \State *recorrido.cantidadElementos = *recorrido.cantidadElementos - 1 	\Comment O(1)
			      \State recorrido $\gets$ *recorrido.izq 				\Comment O(1)

			     \Else 				\Comment O(4)
			      \State nivelCompleto  : bool $\gets$ (2 x *recorrido.der.cantidadElementos) + 1 => **recorrido.izq.cantidadElementos 	\Comment O(1)
			      \State *recorrido.cantidadElementos = *recorrido.cantidadElementos - 1 		\Comment O(1)
			      \If{nivelCompleto} 						\Comment O(2)
				\State recorrido $\gets$ *recorrido.der 			\Comment O(1)
			      \Else 								\Comment O(2)
				\State recorrido $\gets$ *recorrido.izq 			\Comment O(1)
			      \EndIf
			    \EndIf
			  \EndWhile
			  
			  \State c.raiz $\gets$ *recorrido.raiz 				\Comment O(1)
			  \State recorrido $\gets$ NULL 					\Comment O(1)
			  \State *recorrido2  : puntero(heap) $\gets$ $\delta$c 						\Comment O(1)
			  \State llegoAPosicion  : bool $\gets$ false 					\Comment O(1)
			  
			  \While{$\neg$ llegoAPosicion} 					\Comment O(6 x log(n)) = O(log(n))
			    \If{*recorrido2.izq $=$ NULL} 					\Comment O(2)
			      \State llegoAPosicion $\gets$ true			      \Comment O(1)
			    \ElsIf{*recorrido2.der $=$ NULL} 					\Comment O(1)
			      \State llegoAPosicion $\gets$ true 				\Comment O(1)
			      \If{***recorrido2.izq.raiz.prioridad < **recorrido2.raiz.prioridad} 	\Comment O(4)
				\State *aux  : puntero(paquete) $\gets$ *recorrido2.raiz 				\Comment O(1)
				\State *recorrido2.raiz $\gets$ **recorrido2.izq.raiz 		\Comment O(1)
				\State **recorrido2.izq.raiz  $\gets$ aux 			\Comment O(1)
			      \EndIf
			    \Else \Comment O(6)
			      \State nivelCompleto $\gets$ (2 x **recorrido2.der.cantidadElementos) + 1 => **recorrido2.izq.cantidadElementos \Comment O(1)
			      \If{**recorrido2.raiz.prioridad > ***recorrido2.izq.raiz.prioridad $\wedge$ $\neg$ nivelCompleto} 		\Comment O(5)
				\State *aux  : puntero(paquete) $\gets$ *recorrido2.raiz 				\Comment O(1)
				\State *recorrido2.raiz $\gets$ **recorrido2.izq.raiz 	\Comment O(1)
				\State **recorrido2.izq.raiz $\gets$ aux 	\Comment O(1)
				\State recorrido2 $\gets$ *recorrido2.izq	 \Comment O(1)
			      \ElsIf{**recorrido2.raiz.prioridad > ***recorrido2.der.raiz.prioridad $\wedge$ nivelCompleto} 	\Comment O(5)
				\State *aux  : puntero(paquete) $\gets$ *recorrido2.raiz 	\Comment O(1)
				\State *recorrido2.raiz $\gets$ **recorrido2.der.raiz 	\Comment O(1) 
				\State **recorrido2.der.raiz $\gets$ aux \Comment O(1)
				\State recorrido2 $\gets$ *recorrido2.der \Comment O(1)
			      \Else \Comment O(2)
				\State llegoAPosicion $\gets$ true  	\Comment O(1)
			      \EndIf
			    \EndIf
			  \EndWhile
			  \EndIf
			 \medskip
			  \Statex \underline{Complejidad:} O(log(n))
			  \Statex \underline{Justificaci\'on:} Tomamos en consideraci\'on el camino m�s costoso del if.
    	\end{algorithmic}
\end{algorithm}



\end{Algoritmos}




\end{document}